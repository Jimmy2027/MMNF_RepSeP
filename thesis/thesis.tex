% pythontex
%% (Master) Thesis template
% Template version used: v1.4
%
% Largely adapted from Adrian Nievergelt's template for the ADPS
% (lecture notes) project.


%% We use the memoir class because it offers a many easy to use features.
\documentclass[11pt,a4paper,titlepage,english]{memoir}

%% Packages
%% ========

%% LaTeX Font encoding -- DO NOT CHANGE
\usepackage[OT1]{fontenc}

\usepackage[british]{babel} % decent hyphenation, avoiding e.g. anal-ysis
\usepackage[iso]{isodate}
\usepackage{sansmath}
\usepackage{booktabs}
\usepackage{graphicx}
\usepackage{graphviz}
\usepackage{makecell}
\usepackage{minted}
\usepackage{siunitx}
\usepackage{subcaption}
\usepackage[section]{placeins}
\usepackage{amsfonts} % needed for \mathbb{} (Only works on capital letters!)
\usepackage{tikz}
\usetikzlibrary{shapes,snakes}
%\usetikzlibrary{shapes.geometric}

%\usepackage{hyperref}
\usepackage{amsmath}
\usepackage[normalem]{ulem}
% Needs to be loaded after hyperref and amsmath
%\usepackage{cleveref}

% Grey definitions.
\definecolor{dg}{gray}{0.25}
\definecolor{mg}{gray}{0.55}
\definecolor{lg}{gray}{0.73}
\definecolor{vlg}{gray}{0.9}

% new commands
\newcommand{\mg}[1]{\textcolor{mg}{\texttt{#1}}}
\newcommand{\dg}[1]{\textcolor{dg}{\texttt{#1}}}
\newcommand{\xdensity}{\textit{p}_x (\textbf{x})}
\newcommand{\zm}{\textbf{z}_m}
\newcommand{\zs}{\textbf{z}_s}

\newcommand{\udensity}{\textit{p}_u (\textbf{u})}
\newcommand{\transformer}{\tau(z_i;\textbf{h}_i)}
\newcommand{\conditioner}{c_i(\textbf{z}_{<1})}
\newcommand{\where}{\quad \text{where} \quad}
\newcommand{\xset}{\mathbb{X}}
\newcommand{\xsubset}{\mathbb{X}_s}
\newcommand{\xseti}{\mathbb{X}^{(i)}}
\newcommand{\xsetm}{\mathbb{X}_m}
\newcommand{\samplem}{\textbf{x}_m}

\newcommand{\unimodalpost}{q_{\phi_m}(\textbf{z}|\samplem)}

\newcommand{\approxdistri}{q_{\phi_i}(\textbf{z}|\xseti)}
\newcommand{\approxdistr}{q_{\phi, \psi}(\textbf{z}|\xset)}
\newcommand{\jointpost}{q_{\phi}(\textbf{z}|\xset)}
\newcommand{\subsetpost}{\tilde{q}_{\phi}(\textbf{z}|\xsubset)}
\newcommand{\iwsubsetpost}{\tilde{q}_{\phi}(\textbf{z}_k|\xsubset)}
\newcommand{\prior}{p_{\theta}(\textbf{z})}
\newcommand{\truedistri}{p_{\theta}(\textbf{z}|\xseti)}
\newcommand{\truedistr}{p_{\theta}(\textbf{z}|\textbf{X})}
\newcommand{\elbo}{\mathcal{L}(\theta, \phi; \xseti)}

\newcommand{\lmopoe}{\mathcal{L}_{MoPoE}(\theta, \phi; \xset)}
\newcommand{\eqlmopoe}{\lmopoe:= \mathbb{E}_{q_{\phi}(\textbf{z}|\mathbb{X})}[\log (p_{\theta}(\mathbb{X}|\textbf{z}))] - D_{KL}\biggl( \frac{1}{2^M} \sum _{\mathbb{X}_s \in \mathcal{P}(\mathbb{X})} \tilde{q}_{\phi}(\textbf{z}|\mathbb{X}_s)\ ||\ p_{\theta}(\textbf{z})\biggr)}

\newcommand{\eqlpgfm}{\mathcal{L}_{pgfm}(\theta, \phi; \xset):= \mathbb{E}_{\tilde{q}_{\phi_{12}}(\textbf{z}|\mathbb{X})}[\log (p_{\theta}(\mathbb{X}|\textbf{z}))] - D_{KL}\biggl(  \tilde{q}_{\phi_{12}}(\textbf{z}|\mathbb{X})\ ||\ p_{\theta}(\textbf{z})\biggr)}

\newcommand{\eqlmopgfm}{\mathcal{L}_{Mopgfm}(\theta, \phi; \xset):= \mathbb{E}_{q_{\phi}(\textbf{z}|\mathbb{X})}[\log (p_{\theta}(\mathbb{X}|\textbf{z}))] - D_{KL}\biggl( \frac{1}{2^M} \sum _{\mathbb{X}_k \in \mathcal{P}(\mathbb{X})} \tilde{q}_{\phi_k}(\textbf{z}|\mathbb{X}_k)\ ||\ p_{\theta}(\textbf{z})\biggr)}

\newcommand{\powerset}{\mathcal{P}(\xset)}
\newcommand{\pFmean}{\mathcal{M}_{f_{\psi}}}
\newcommand{\DklTrueApprox}{D_{KL} \left( \approxdistr || \truedistr \right)}
\newcommand{\Mnfi}{\mathcal{M}_{f_{\psi}}\left( q_{\phi _i}(\textbf{z}|\textbf{x}_i) \right)}




% PythonTeX
\usepackage[autoprint=false, gobble=auto, keeptemps=all, pyfuture=all]{pythontex} % create figures on-line directly from python!
\usepackage{pgf}
\input{lib/functions.py}
\begin{pythontexcustomcode}[begin]{py}
pytex.add_dependencies(
	'lib/utils.py',
	'lib/categorical.py',
	)
\end{pythontexcustomcode}
% Single-session PythonTeX codeblocks
\newcounter{pysessioncounter}
\newcommand{\sessionpy}{%
          \edef\sessionpysession{session\arabic{pysessioncounter}}%
            \stepcounter{pysessioncounter}%
              \expandafter\py\expandafter[\sessionpysession]}

% SIunitx customizations detect-all will use the current font for typesetting
\sisetup{per-mode=symbol, detect-all, range-units = single}
\newcommand\SIci[5]{\SI{#1}{#2}, {#3}CI: \SIrange{#4}{#5}{#2}}

% Fix for matplotlib PGF wonkiness which isn't interpreted correctly by pdflatex
\DeclareUnicodeCharacter{2212}{-}

%% Input encoding 'utf8'. In some cases you might need 'utf8x' for
%% extra symbols. Not all editors, especially on Windows, are UTF-8
%% capable, so you may want to use 'latin1' instead.
\usepackage[utf8]{inputenc}

%% This changes default fonts for both text and math mode to use Herman Zapfs
%% excellent Palatino font.  Do not change this.
\usepackage[sc]{mathpazo}

%% The AMS-LaTeX extensions for mathematical typesetting.  Do not
%% remove.
\usepackage{amsmath,amssymb,amsfonts,mathrsfs}

%% NTheorem is a reimplementation of the AMS Theorem package. This
%% will allow us to typeset theorems like examples, proofs and
%% similar.  Do not remove.
%% NOTE: Must be loaded AFTER amsmath, or the \qed placement will
%% break
\usepackage[amsmath,thmmarks]{ntheorem}

%% This allows you to add .pdf files. It is used to add the
%% declaration of originality.
\usepackage{pdfpages}

%BIBLIOGRAPHY
\usepackage[backend=bibtex,style=authoryear,natbib=true]{biblatex}

%% Some more packages that you may want to use.  Have a look at the
%% file, and consult the package docs for each.
\input{thesis/extrapackages}

%% Our layout configuration.  DO NOT CHANGE.
\input{thesis/layoutsetup}

%% Theorem environments.  You will have to adapt this for a German
%% thesis.
\input{thesis/theoremsetup}

%% Helpful macros.
%% Custom commands
%% ===============

%% Fixed/scaling delimiter examples (see mathtools documentation)
\DeclarePairedDelimiter\abs{\lvert}{\rvert}
\DeclarePairedDelimiter\norm{\lVert}{\rVert}

%% Use the alternative epsilon per default and define the old one as \oldepsilon
\let\oldepsilon\epsilon
\renewcommand{\epsilon}{\ensuremath\varepsilon}

%% Also set the alternate phi as default.
\let\oldphi\phi
\renewcommand{\phi}{\ensuremath{\varphi}}


%% Make document internal hyperlinks wherever possible. (TOC, references)
%% This MUST be loaded after varioref, which is loaded in 'extrapackages'
%% above.  We just load it last to be safe.
\usepackage[linkcolor=black,colorlinks=true,citecolor=black,filecolor=black]{hyperref}
\addbibresource{bib.bib}
\usepackage{cleveref}

% pythontex dependencies
\begin{pythontexcustomcode}[begin]{py}
    DOC_STYLE="thesis/main.conf"
    pytex.add_dependencies(
    'thesis/scripts/plots/epoch_comparison_lr.py',
    'thesis/scripts/plots/epoch_comparison_gen.py',
    'thesis/scripts/plots/nbr_mods_comparison_gen.py',
    'thesis/scripts/plots/nbr_mods_comparison_lr.py',
    'thesis/scripts/plots/utils.py',
    'thesis/scripts/tikz_graphs/rand_gen_comp_polymnist.py',
    'thesis/scripts/tikz_graphs/mimic_lat_pa_example.py',
    'thesis/scripts/gen_eval_tab.py',
    'thesis/scripts/params_tab_polymnist.py',
    'thesis/scripts/params_tab_mimic.py',
    'scripts_/mopoe_graph.py',
    'scripts_/mogfm_graph.py',
    'scripts_/mopgfm_graph.py',
    'scripts_/mofop_graph.py',
    'lib/utils.py',
    'data/thesis/gen_eval.csv',
    DOC_STYLE
    )
\end{pythontexcustomcode}

%% Document information
%% ====================

\title{Multi Modal Generative Learning\\ utilizing Normalizing Flows \\}
\author{Hendrik J. Klug}
\thesistype{Master Thesis}
\advisors{Advisors: Prof.\ Dr.\ Julia Vogt, MSc.\ Thomas M. Sutter}
\department{Department of Computer Science}
\date{}

\begin{document}

    \frontmatter

%% Title page is autogenerated from document information above.  DO
%% NOT CHANGE.
    \begin{titlingpage}
        \calccentering{\unitlength}
        \begin{adjustwidth*}{\unitlength-24pt}{-\unitlength-24pt}
            \maketitle
        \end{adjustwidth*}
    \end{titlingpage}

%% The abstract of your thesis.  Edit the file as needed.
    \begin{abstract}
%  Multimodal data naturally grants self-supervision in the form of shared information connecting the different data types.
    Multi modal, generative models are able to learn underlying generative factors of multiple data types without the need for supervision.
    Existing methods use a fixed, pre-selected aggregation function to merge the learned representation of each modality into a joint posterior distribution.
    Here, we generalise previous work by implementing the aggregation over modalities using a trainable generalized $f$-means.
    We show that this more flexible way to fuse the information between modalities improves the ability of the model to learn a meaningful joint posterior approximation and to generate coherent samples across data types.
%  We show that this more flexible way to fuse the information between modalities improves the ability of the model to generate coherent samples across modalities.
\end{abstract}


%% TOC with the proper setup, do not change.
    \cleartorecto
    \tableofcontents
    \mainmatter

    \chapter{Introduction}
% in recent years, the field of deep learning has seen a shift from uni modal learning towards multi modal learning.
%motivation for multi modal learning
The availability of multiple data types provides a rich source of information and holds promise for learning representations that generalise well across multiple modalities \citep{baltrusaitis_multimodal_2019}.
Similar to how humans learn and extract information from their surroundings using an aggregation of their senses, a machine learning model can learn from multiple data types.
Multimodal data naturally grants self-supervision in the form of shared information connecting the different data types.
It also serves as an inherent regularization which forces the model to learn more robust features from the data, since these features need to be connected between modalities.
This may lead to more interpretable features for humans since they also infer from multiple modalities.
A model that can generate any of the learned modalities, given any subset of modalities can be used for translation between modalities for example, such as image captioning.
It can also find applications in the medical domain, where the model could generate, conditioned on images and medical data of a patient, a text describing the medical condition of a patient.
%Self-supervised training paradigms are especially useful in the medical domain since there labeled data is expensive to acquire and thus very scarce.
% examples: (\citep{dorent_hetero-modal_2019}, \citep{calixto_latent_2019})

% motivation for generative models
However, the understanding of different modalities and the interplay between data types are non-trivial research questions and longstanding goals in machine learning research \citep{ngiam_multimodal_nodate}.
While fully supervised approaches have been applied successfully \citep{karpathy_deep_2015,tsai_learning_2018}, the labeling of multiple data types remains time-consuming and expensive.
Therefore, models that efficiently learn from multiple data types in a self-supervised fashion are much more widely applicable for real world problems.
In the medical domain, for example, self-supervised training paradigms are especially useful since there labeled data is expensive to acquire and thus very scarce.
%Therefore, it requires models that efficiently learn from multiple data types in a self-supervised fashion.
Generative models represent a natural way to learn underlying generative factors of the data, in a self-supervised fashion.



Self-supervised, multi modal generative models have been applied to toy datasets \citep{poe, shi_variational_2019, sutter_generalized_2020} and real world data \citep{klug_multimodal_nodate}, however results have shown that current methods are not able to aggregate well enough over the modalities to generate coherent samples.
For the model to generate coherent samples, it needs to extract and fuse information from the multiple data types.
An image captioning model for example, needs to extract information from the image and generate text from it when generating the caption for an image of a green apple.
Captions such as "A red apple." or "A yellow truck." would not be coherent with the image of a green apple.

In previous work, the aggregation over modalities is done with multiple, fixed, pre-selected methods, each coming with advantages and disadvantages.
Here, we generalise previous work by implementing the aggregation over modalities using a generic function with trainable parameters.
We show that this more flexible way to fuse the information between modalities improves the ability of the model to generate coherent samples across data types.
    \chapter{Related Work}
\section{Generative Modeling}
Generative adversarial networks \citep[GANs]{goodfellow_generative_2014} and variational autoencoders (VAEs, \cref{subsec:vae}) are the two most popular methods for generative modeling.
Both attempt to model the distribution over the data, however while for the VAEs, the resulting posterior approximation is defined explicitly, the learned posterior of GANs can not be evaluated directly.
GANs are made of two models that are trained simultaneously, a generative model G that captures the data distribution, and a discriminative model D that estimates the probability that a sample came from the training data rather than G.
The joint optimization of both models D and G can be tricky in practice and GANs are known to suffer from mode collapse since the objective does not require the learned representation to contain all modes of the data.
For images of animals for example, the generator G could learn to generate only images of brown, short haired dogs, so well that the discriminator D will not be able to distinguish them from the true data.
Mode collapse does not happen in VAEs since their objective explicitly requires their learned representation to contain all modes of the data.
Also, since the learned posterior distribution of VAEs can be evaluated explicitly, additional constraints can be added to the objective to push the posterior distribution to have specific characteristics and it can be used for downstream tasks like clustering or classification.
In this work, we focus on VAEs and give a more in depth introduction in section \cref{subsec:vae}.

\section{Multi Modal Generative Modeling}

There have been a wide range of approaches for multi-modal generative modeling, however most fall short of expressing the complete range of behaviour that we expect in this setting.

\paragraph{Modality Translation}
Most prior approaches to generative modelling with multi modal data have targeted modality translation, where the model learns to generate one modality conditioned on another one.
In this case input an output modalities of the model are not interchangeable.
Modality translation has been proposed both as VAE based \citep{pu2016variational, pandey2017variational}, as well as GAN based, for domain translation of images \citep{ledig2017photo, liu2019few}.
% todo
However, we expect our method to be able to generate any modality given any subset of modalities which extends translation between modalities.
It would be possible to train $2^M -1$ modality translation network pairs for $M$ modalities, but this is intractable in practice.

\paragraph{Joint approximation}
Other prior work has targeted to directly model the joint distribution over the data.
The joint multi modal VAE (JMVAE) from \citep{suzuki2016joint} learns a joint posterior distribution using a joint inference network.
To handle missing data at test time, inference networks need to be trained for every subset of modalities.
While feasible for two modalities, this setup quickly becomes intractable with more data types.
Similarly, the multimodal factorisation model (MFM) from \citep{tsai2018learning} explicitly defines a joint inference network on top of uni modal encoders, however additional decoder networks are needed to generate missing modalities.

These approaches typically do not scale well with the number of modalities since they require additional modelling components for each combination of modalities.
The MVAE from \citep{poe} marked an improvement over previous methods in this regard, proposing to model the joint posterior as a product of experts (POE) over the marginal posteriors, enabling cross-modal generation at test-time without requiring additional inference networks and multi-stage training regimes.
Since then, other methods have emerged, each proposing another aggregation function over the marginal posteriors.
We refer to \cref{subsec:Multi Modal VAEs} for a more in depth introduction to the MVAE and other methods that build on it.

Next to the aggregation function with which the uni modal posteriors are merged, other methods have been proposed to improve multi modal VAEs (mmVAEs).
In \citep{daun_disent}, the authors propose to split the latent space into modality specific and shared information in order to disentangle \citep{burgess_understanding_2018} them in a purely self-supervised manner.
The aggregation of modalities should only happen over the shared information and thus it makes sense to separate it from the modality specific information in order to simplify the aggregation.
For this, the authors add a new term to the mmVAE objective, which disentangles the shared representations with the modality specific representations and encourages mutual information between representations that contain shared information.
This has been shown to improve the conditional generation of missing modalities, however the results from \citep{sutter_multimodal_2020} point out that independent of that separation, the generation coherence differs between different merging functions.
The goal of this work is solely to improve the merging function, which is why we forgo this method even though we expect the separation of shared and modality specific information to improve our results.

    \section{Background}
In this section, we introduce concepts and previous work on VAEs, self supervised multi modal generative learning and normalizing flows, on which this work builds on.

\subsection{Variational Autoencoder}
The VAE, first introduced by \cite{kingma_auto-encoding_2014} and \cite{rezende_stochastic_2014}, consists of an encoder network and a decoder network.
In contrast to a typical auto encoder network, the VAE is trained such that it's learned representation has the structure of a prior distribution.
The most popular choice for a prior is the standard Gaussian distribution $\mathcal{N}(0,\textbf{I})$, which we also use in this work.
The latent representation being a distribution, the decoder part can generate unseen data by sampling from it.
The model is trained such that it maximizes the log-likelihood ($\log p(x)$) of the data by maximizing the Evidence Lower BOund (ELBO):

\begin{equation}
    \label{vaeelbo}
    \begin{split}
        \log p(x) &= \log \int p(x,z) dz\\
        &=  \log \int p(x,z) \frac{q(z|x)}{q(z|x)}dz\\
        &\geq \mathbb{E} _{q(z|x)}[\log \frac{p(x|z)p(z)}{q(z|x)}]\\
        &= \mathbb{E} _{q(z|x)}[\log p(x|z)] - \mathbb{E} _{q(z|x)}[\log \frac{q(z|x)}{p(z)}]\\
        &= \mathbb{E} _{q(z|x)}[\log p(x|z)] - D_{KL}\left( q(z|x)\ ||\ p(z)\right)
    \end{split}
\end{equation}

The ELBO consists of two parts: the reconstruction loss which pushes the generated samples to resemble the real data and a regularization term which forces the latent representation to be structured like the prior.
In \cite{beta_vae}, the authors introduce the hyperparameter $\beta$, which allows to weight the regularization term in the VAE objective:
\begin{equation}
    \label{eq:vaeelbo}
    \mathcal{L}_{ELBO} = \mathbb{E} _{q(z|x)}[\log p(x|z)] - \beta D_{KL}\left( q(z|x)\ ||\ p(z)\right)
\end{equation}

A lower $\beta$ gives the model more freedom in learning the latent representation, while a higher $\beta$ forces the model to learn a latent distribution that is disentangled, like the prior.
"Disentangled" here means that each dimension in the learned latent representation is independent of each other, and represents a latent factor that corresponds to a different attribute in the data.
In images of animals for example, one dimension in the latent representation could represent the color of the fur, while another might correspond to the color of the eyes.
Both the color of the fur and the color of the eyes are independent, and so should be the corresponding latent variables.


%todo benefits of disentanglement? no shared information between dimension -> maximum information in latent representation?

%It is widely believed that disentanglement leads to a better latent representation
Disentanglement is a popular objective in representation learning and has been addressed in recent works \parencite{chen_isolating_2019, locatello_challenging_2019}.

\subsection{Multi Modal VAEs}
In order for the VAE model to learn a representation which captures the underlying factors of multiple modalities, several adaptations to the objective in \cref{vaeelbo} need to be made.
The first approach that scales with the number of modalities, allows for a coherent joint generation over all modalities and cross-generation across individual modalities, the MVAE, was introduced in \cite{poe}.
The MVAE makes the assumption that the joint posterior is a product of uni modal posteriors, also called a Product-of-Experts (PoE) \parencite{hinton_training_2002}:

\begin{equation}
    p(z|x_1,\ldots,x_M) \propto \prod ^M _{i=1} q(z|x_i)
\end{equation}

The PoE has the advantage of aggregating information across any subset of uni modal posteriors which allows for missing modalities.
However, the product of experts does not train the individual inference networks and they don't learn to handle missing data at test time.
To address this issue, the MVAE requires a sub-sampling of uni modal log-likelihoods, which no longer guarantees a valid lower bound on the joint log-likelihood \parencite{wu_multimodal_2019}.

Another approach was proposed with the MMVAE in \cite{shi2019variational}, which models the joint posterior as a mixture of uni modal posteriors, i.e. a mixture of experts (MoE):

\begin{equation}
    p(z|x_1,\ldots,x_M) = \frac{1}{M} \sum _{i=1} q(z|x_i)
\end{equation}

The MoE has the advantage of optimizing each inference network individually, however it does not merge the information between posteriors since only uni modal posteriors are considered during training.

Both the advantage of the PoE, which results in a good approximation of the joint distribution and the MoE which optimizes each uni modal posterior individually are combined in the MoPoE \parencite{thomas_gener-ELBO}.
The MoPoE-VAE takes advantage of both methods by merging the uni modal posteriors into $2^M$ subsets, which are then combined with a MoE (see \cref{mopoeGraph}).

\begin{figure}[h!]
    \centering
    \resizebox{0.9\textwidth}{!}{%
        \py{pytex_printonly(script='scripts_/mopoe_graph.py', data = '')}
    }
    \caption{\textbf{The MoPoE makes use of the PoE to create $2^M$ subsets, which are then merged with a MoE.} Here $M=2$, the empty subset is not shown. On the left side are the two input modalities from the polymnist dataset (see \cref{polymnist}), on the right side are the generated samples. In the header of each generated sample is described from which subset the decoder sampled for the generation (left side of the $\rightarrow$) and which modality was generated (right side of the $\rightarrow$).}
    \label{mopoeGraph}
\end{figure}

Similar to the MoE, the MoPoE models the joint posterior as a mixture, however of subsets instead of uni modal posteriors.
For multi modal data $\mathbb{X}$ with M modalities, and $2^M$ subsets of modalities $\mathbb{X}_s \in \mathbb{X}$, the objective of the MoPoE, which is an evidence lower bound (ELBO) on the joint log-likelihood $\log p_{\theta}(\xset)$, can be written as follows:

\begin{equation}
    \eqlmopoe
\end{equation}

with $\jointpost$ the joint posterior:
\begin{equation}
    \jointpost = \frac{1}{2^M} \sum _{\xsubset \in \powerset} \subsetpost
\end{equation}

and $\subsetpost$ the posterior approximation of subset $\xsubset$:
\begin{equation}
    \label{eq:subsetmopoe}
    \subsetpost=PoE(\{q_{\phi_j}(\textbf{z}|\textbf{x}_j) \forall \textbf{x}_j \in \xsubset\}) \propto \prod _{\textbf{x}_j \in \xsubset}q_{\phi_j}(\textbf{z}|\textbf{x}_j)
\end{equation}

For gaussian posteriors, the PoE in \cref{eq:subsetmopoe} can be computed in closed form.

In this work, the MoPoE is taken as the current state of the art for scalable, self supervised, multi modal generative models and used as baseline.

\subsection{Importance Weighted Autoencoder}
It has been shown that the tightness of the ELBO in \cref{eq:vaeelbo} can be improved by sampling multiple samples form the posterior at each step \parencite{burda_importance_2016}, which results in the following lower bound:
\begin{equation}
    \label{eq:iwelbo}
    \log p(x) \geq \mathbb{E}_{z_1,\ldots,z_K \sim q_{\phi}(z|x)}\left[ \log \frac{1}{K} \sum ^K _{k=1} \frac{p_{\theta}(x|z_k)p_{\theta}(z)}{q_{\phi}(z_k| x)} \right]\\
    := \mathcal{L}_K
\end{equation}

\Cref{eq:iwelbo} yields useful properties summarized in \parencite{nowozin_debiasing_2018}, namely that one recovers the ELBO for $K=1$, $\mathcal{L}_K$ approaches the true $\log p(x)$ for larger Ks ($\lim _{K \rightarrow \inf} \mathcal{L}_K = \log p(x)$) and $\mathcal{L}_1, \ldots, \mathcal{L}_K$ provide stochastic monotonicity ($\mathcal{L}_E = \mathcal{L}_1 \leq \mathcal{L}_2 \leq \ldots \leq \log p(x)$).
The MMVAE from \parencite{shi2019variational} adapts this for multi modal data:
\begin{equation}
    \label{eq:iwelbommvae}
    \mathcal{L}^{MoE}_K(\textbf{x}_{1:M}) =
    \frac{1}{M} \sum _{m=1} ^M \mathbb{E}_{z_m^{1:K} \sim q_{\phi_m}(z|x_m)}\left[ \log \frac{1}{K} \sum ^K _{k=1} \frac{p_{\theta}(x_{1:M}|z_m^k)p_{\theta}(z^k_m)}{q_{\phi}(z_m^k| x_{1:M})} \right]
\end{equation}
which is a valid lower bound of the multi modal log likelihood $\log p(\xset)$.

In our work, we make use of this importance sampling training paradigm to improve the tightness of the ELBO and to approximate the KL-divergence between the posterior and the prior by comparing samples from both (see \cref{sec:methods})).

\subsection{Normalizing Flows}
Normalizing flows \parencite{papamakarios_normalizing_2019} represent an approach for defining invertible and differentiable transformations of probability distributions.
They are widely used for generative modeling \citep[\textbf{GLOW}, \textbf{Real NVP}]{kingma_glow_2018, dinh_density_2017} and variational inference \parencite{rezende_variational_2016, berg_sylvester_2019}.
In this work, we make use of normalizing flows both as a simple parameterizable invertible function for the $f$-mean, as well as a transformation of the joint posterior into an arbitrary complexity in order to improve its ability to capture the underlying factors of multiple modalities.

In generative modeling, normalizing flows are used to learn a diffeomorphic mapping $T$ from images to a prior, like Gaussian noise.
Since $T$ is invertible, one can then transform samples from the prior into new images with $T^{-1}$.

In practice, flow-based models are typically constructed by implementing a diffeomorphic transformation T (or $T^{-1}$) with a neural network.
Because invertible and differentiable transforms are composable, complex transformations can be built by composing multiple instances of simpler ones: $T=T_K \circ \cdots \circ T_1$.
The density of the transformed posterior $\tilde{q_{\phi}}$ can easily be obtained with the change of variable formula \parencite{bogachev2007measure}:
\begin{equation}
    \tilde{q_{\phi}} = T(q_{\phi}) \where q_{\phi} \sim p_{q_{\phi}}(q_{\phi}) = \mathcal{N}(\mu_{\phi}, \sigma^2_{\phi})
\end{equation}
\begin{equation}
    \label{eq:changeofvariables}
    p_{\tilde{q}_{\phi}}(\tilde{q}_{\phi}) = p_{q_{\phi}}(q_{\phi})|\det J_T(q_{\phi})|^{-1}
\end{equation}








    \section{Methods}
\label{sec:methods}
As introduced herein, we are working with a multi modal VAE (mmVAE), which learns a joint distribution that contains the combined information of each learned uni modal latent distribution.
In order to generalize previous methods and to increase the flexibility of the combination of the modalities, we implement the fusion of the uni modal latent distributions with a mixture of generalized $f$-mean.
Instead of merging the uni modal posteriors into subset-posteriors with a PoE, like is done in the \mg{MoPoE}, we merge them with a trainable $f_{\psi}$-mean, with parameters $\psi$.
The main difficulty in this approach comes from the fact that the $f$-mean of the uni modal distributions follows an unknown distribution.
While this makes the joint distribution more flexible, this also makes the computation of the regularization term in the ELBO, the KL-divergence, more difficult to compute.
In fact, if the density of the joint distribution is unknown, it is impossible to compute the KL-divergence in closed form.

An intuitive alternative would be to find an upper bound of the KL-divergence which can be computed in closed from, such that it can be minimized in order to minimize the true divergence:
\begin{equation}
    \label{eq:kldivbound}
    D_{KL}^{\prime} \geq D_{KL}(\mathcal{M}_f(\{\unimodalpost\ \forall\ \xsetm \in \xsubset\})) =  D_{KL}\left(f^{-1}\left(\sum _{\xsetm \in \xsubset} \frac{f(\unimodalpost)}{|\xsubset|}\right)\ ||\ \prior\right)
\end{equation}

Using the change of variable formula (\cref{eq:changeofvariables}), the $f$-mean can be rewritten as follows:
\begin{equation}
    \mathcal{M}_f = f^{-1}(Q)|J_{f^{-1}}(Q)|
\end{equation}
with
\begin{equation}
    Q= \sum _{\xsetm \in \xsubset} \frac{\unimodalpost|J_f(\unimodalpost)|}{|\xsubset|}
\end{equation}

Here Q is a sum of random variables, which can be rewritten as chained convolutions \footnote{\url{https://en.wikipedia.org/wiki/Sum_of_normally_distributed_random_variables}} and is hard to evaluate.

Instead, we propose three workarounds to the computation of the KL-divergence in \cref{eq:kldivbound}.
\begin{enumerate}

    \item For one, \cref{eq:kldivbound} can be simplified by skipping the backwards transformation $f^{-1}$.
    This leads to a mixture of transformed posteriors, which divergence can be bounded using \cref{lemma:DklLowerBound} from \parencite{sutter_multimodal_2020}.

    We then get an upper bound that can be minimized:
    \begin{equation}
        D_{KL}\left(\sum _{\xsetm \in \xsubset} \frac{f_{\psi}(\unimodalpost)}{|\xsubset|}\ ||\ \prior\right) \leq \frac{1}{|\xsubset|} \sum  _{\xsetm \in \xsubset} D_{KL} \left(f_{\psi} (\unimodalpost))||\ \prior \right) \quad (\cref{lemma:DklLowerBound})
    \end{equation}
% todo write all method names as grey ?
    We implement this in the Mixture of flow of product of experts (\mg{MofoPoE}) model, which is described in \cref{subsec:mofopoe}.

% todo beschreib mofop method and sag vorteile und nachteile von dieser Methods: man macht die einzelnen posterior besser aber man macht das mergen der information nicht flexibler
    % todo sag dass weil das nicht so gut funktioniert, this hints that the true problem is not the merging method of the information but the form of the posterior.

    \item Another way to simplify the Kl-divergence in \cref{eq:kldivbound} is to force the output of the $f$-mean to be a Gaussian distribution.
    This can be done by, instead of mixing the posteriors which follow a normal distribution, mixing their parameters $\mu_s$ and $\sigma_s$.
    The joint posterior is then described as follows:
    \begin{equation}
        \label{eq:qjointmopgfm}
        q_{\phi, joint} \sim \mathcal{N}\left(  f_{\mu}^{-1}(\sum _{\xsetm \in \xsubset} \frac{f_{\mu}(\mu_s)}{|\xsubset|}),\ f_{\sigma}^{-1}(\sum  _{\xsetm \in \xsubset} \frac{f_{\sigma}(\sigma_s^2)}{|\xsubset|})\right)
    \end{equation}

    This is implemented as the mixture of parameter generalized $f$-mean (\mg{MopgfM}) and described in \cref{subsec:mopgfm}.

    \item The sum of random variables in the $f$-mean (\cref{eq:kldivbound}) is hard to evaluate since the transformed uni modal posteriors ($\unimodalpost$) follow an unknown distribution.
    It is however possible to steer the normalizing flow $f_{\psi}$ to map towards $\unimodalpost$ a normal distribution, such that the sum of random variables can be evaluated.
    This normal distribution can be amortized by making it dependent on the input.
    We implement this as the \mg{MogfM\_amortized} method, described in \cref{subsubsec:mogfm_amortized}.

    \item The third option is to instead of computing the KL-divergence in closed form, one can approximate it by sampling from the posterior, i.e. comparing k samples from the posterior with k samples from the prior.
    This method is similar to the importance weighted VAE \parencite[iwVAE]{burda_importance_2016}, however since the Kl-divergence is not computed is closed form but approximated with the K samples, a higher number of K is required to obtain good results.
    This method is implemented as the importance weighted mixture of generalized $f$-mean (iwMogfM) and described in %todo.
\end{enumerate}

\subsection{Models}
Here we describe the models that implement the three methods introduced above and enumerate their advantages and disadvantages.

\subsubsection{MofoPoE}\label{subsec:mofopoe}
The \mg{MofoPoE} builds on the \mg{MoPoE} by transforming the subset posteriors $\subsetpost$ with a series of F invertible transformations with trainable parameters $\psi$:
\begin{equation}
    z_{F,S} =f_{\psi}(z_{0,S} \sim \subsetpost) = f_F \circ \ldots \circ f_2 \circ f_1(z_{0,S} \sim \subsetpost)
\end{equation}

The density of the resulting transformed subset posterior can be evaluated with the change of variables formula (\cref{eq:changeofvariables}):
\begin{equation}
    \label{eq:changeofvariables_}
    \ln f(\subsetpost) = \ln q_\phi (z_0|\xsubset) - \sum _{i=1} ^{F}\ln \left|  \det \frac{df_i}{dz_{i-1}}\right|
\end{equation}

Here $f(\subsetpost)$ can follow any distribution is thus more flexible than the gaussian subset posterior in the \mg{MoPoE} model.
A flow chart depiction of the \mg{MofoPoE} is shown in \cref{fig:mofopoe}.
Effectively, during a forward pass, the reparameterisation happens at the subset posteriors.
I.e. a sample is taken from each subset posterior, transformed with a normalizing flow $f$ and then mixed with a MoE.

The resulting objective can be written as follows, by slightly modifying the \mg{MoPoE} objective from \cref{eq:mopoe_}:

\begin{equation}
    \begin{split}
        \mathcal{L}_{\mg{MofoPoE}}(\theta, \phi, \psi; \xset) &=  \mathbb{E}_{q_{\phi}(\textbf{z}|\mathbb{X})}[\log (p_{\theta}(\mathbb{X}|\textbf{z}))] - \frac{1}{2^M} \sum _{\mathbb{X}_s \in \mathcal{P}(\mathbb{X})} D_{KL}\biggl( \tilde{q}_{\phi}(\textbf{z}|\mathbb{X}_s)\ ||\ p_{\theta}(\textbf{z})\biggr)\\
        &= \mathbb{E}_{q_{\phi}(\textbf{z}|\mathbb{X})}[\log (p_{\theta}(\mathbb{X}|\textbf{z}))] - \frac{1}{2^M} \sum _{\mathbb{X}_s \in \mathcal{P}(\mathbb{X})} D_{KL}\biggl( f_{\psi} (\unimodalpost))\ ||\ p_{\theta}(\textbf{z})\biggr)
    \end{split}
\end{equation}

The KL-divergences between the transformed subset posteriors and the prior can be evaluated as follows using \cref{eq:changeofvariables_}:

\begin{equation}
    \begin{split}
        D_{KL}\biggl( f_{\psi} (\unimodalpost))\ ||\ p_{\theta}(\textbf{z})\biggr) &= \mathbb{E}_{f_{\psi}(q_{\phi_m})} \left[ \log  f_{\psi} (q_{\phi_m}(f_{\psi}(z)|x_m)) - \log p_{\theta}(f_{\psi}(z))   \right]\\
        &= \mathbb{E}_{q_{\phi_m}} \left[ \log  q_{\phi_m}(z|x_m) - \log \det J_{f_{\psi}} - \log p_{\theta}(f_{\psi}(z))   \right]\\
    \end{split}
\end{equation}

We use the \mg{MoFoPoE} method in comparison to the other methods that make use of the inverse transform $f^{-1}$ to evaluate if the merging of information between unimodal posteriors can be improved by simply making the subsets more flexible. % todo

One advantage of the \mg{MoFoPoE} method is that since the inverse of the flow transformation is not needed, implementations of normalizing flows can be used were the (% todo schreib das vielleicht lieber als NAchteil bei gfm models?)

\begin{figure}[h!]
    \centering
    \resizebox{0.9\textwidth}{!}{%
        \py{pytex_printonly(script='scripts_/mofop_graph.py', data = '')}
    }
    \caption{\textbf{Fowchart depicting the \mg{MofoPoE} method.} The \mg{MofoPoE} creates more expressive subset posteriors by transforming the PoE posteriors with a series of invertible transformations.}
    \label{fig:mofopoe}
\end{figure}

\subsubsection{MopgfM}\label{subsec:mopgfm}
% the main disadvantage here is that this method requires many parameters (needs to flows) but has not much flexibility
The \mg{Mopgfm} mixes the mean and the standard deviation of the unimodal posteriors, in order to obtain a normal distribution that depends on each of the uni modal posteriors (see \cref{eq:qjointmopgfm}).
This is a generalisation of the PoE method since a product of gaussian experts is itself Gaussian with mean $\mu_{PoE} = (\sum _i \mu _i V_i)(\sum _i V_i)^{-1}$ and covariance $V_{PoE}= (\sum _i V_i)^{-1}$ where $\mu _i, V_i$ are the parameters of the $i$-th Gaussian.
Without loss of generality, it can be assumed that $f_{\mu}^{-1}(\sum _{\xsetm \in \xsubset} \frac{f_{\mu}(\mu_s)}{|\xsubset|}) = \mu_{PoE}$ and $f_{\sigma}^{-1}(\sum  _{\xsetm \in \xsubset} \frac{f_{\sigma}(\sigma_s^2)}{|\xsubset|}) = V_{PoE}$.
The main advantage of this method is that since it is a generalisation of the PoE, it gives more flexibility to the modality fusion.
However, this comes at the cost that the expressiveness of the joint distribution is limited by being a Gaussian, and since the transformations are applied on the parameters of the uni modal distributions, transparency of the resulting transformation is lost.
It is hard, if not impossible, to translate \cref{eq:qjointmopgfm} into the following equation:
\begin{equation}
    q_{\phi, joint} = T(\{q_{\phi _m}(z|x_m) \forall x_m \in \xset\})
\end{equation}
with T a well defined transformation.

\begin{figure}[h!]
    \centering
    \resizebox{0.9\textwidth}{!}{%
        \py{pytex_printonly(script='scripts_/mopgfm_graph.py', data = '')}
    }
\end{figure}

\subsubsection{MogfM\_amortized}\label{subsubsec:mogfm_amortized}
For the \mg{MogfM\_amortized} method, we introduce a new loss $\mathcal{L}_2$ that pushes $f_{\psi}$ to map the uni modal posteriors to an amortized prior distribution, i.e. such that:

\begin{equation}
    \label{eq:amortizedprior}
    f_{\psi}(\unimodalpost) \sim \mathcal{N}(f_{\psi}(\mu_m), \textbf{I})
\end{equation}

Then, the density of the sum of random variables $\textbf{G}_f$ can easily be evaluated with:
\begin{equation}
    \textbf{G}_f(\textbf{z}|\textbf{x}_{1:|\xsubset|}) =\sum _{\xsetm \in \xsubset} \frac{f(\unimodalpost)}{|\xsubset|} \sim \mathcal{N} \left(  \sum _{m \in \xsubset} \frac{f(\mu_m)}{|\xsubset|}, \frac{1}{\sqrt{|\xsubset|}}  \cdot \textbf{I} \right)
\end{equation}

\Cref{eq:amortizedprior} can be achieved by minimizing the KL-divergence between the transformed uni modal posteriors and the amortized prior:
\begin{equation}
    \begin{split}
        \mathcal{L}_2 &= \sum _{\xsetm \in \xset} D_{KL}\left( f(\unimodalpost)\ ||\ \mathcal{N}(f(\mu_m), \textbf{I}) \right)\\
        &= \sum _{\xsetm \in \xset} D_{KL}\left( f(\unimodalpost)\ ||\ p_{\theta_m}(\textbf{z}) \right)\\
        &=  \sum _{\xsetm \in \xset} \mathbb{E}_{f(\unimodalpost)} [\log f(\unimodalpost) - \log p_{\theta_m}(\textbf{z})]\\
        &=  \sum _{\xsetm \in \xset} \mathbb{E}_{z_m \sim \unimodalpost} [\log q_{\phi_m}(z_m|\textbf{x}_M) - \log \det J_f  - \log p_{\theta_m}(f(z_m))]\\
    \end{split}
\end{equation}

The ELBO can then be evaluated as following:
\begin{equation}
    \begin{split}
        \mathcal{L}_1 &=  \mathbb{E}_{q_{\phi}(\textbf{z}|\mathbb{X})}[\log (p_{\theta}(\mathbb{X}|\textbf{z}))] -  \frac{1}{2^M} \sum _{\mathbb{X}_s \in \mathcal{P}(\mathbb{X})} D_{KL}\biggl( \tilde{q}_{\phi}(\textbf{z}|\mathbb{X}_s)\ ||\ p_{\theta}(\textbf{z})\biggr)\\
        &= \mathbb{E}_{q_{\phi}(\textbf{z}|\mathbb{X})}[\log (p_{\theta}(\mathbb{X}|\textbf{z}))] - \frac{1}{2^M} \sum _{\mathbb{X}_s \in \mathcal{P}(\mathbb{X})} \mathbb{E}_{\tilde{q}_{\phi}(\textbf{z}|\mathbb{X}_s)}[\log \tilde{q}_{\phi}(\textbf{z}|\mathbb{X}_s) - \log p_{\theta}(\textbf{z}) ]\\
        &= \mathbb{E}_{q_{\phi}(\textbf{z}|\mathbb{X})}[\log (p_{\theta}(\mathbb{X}|\textbf{z}))] - \frac{1}{2^M} \sum _{\mathbb{X}_s \in \mathcal{P}(\mathbb{X})} \mathbb{E}_{\textbf{G}_f(\textbf{z}|\textbf{x}_{1:|\xsubset|})}[\log \textbf{G}_f(\textbf{z}|\textbf{x}_{1:|\xsubset|}) + \log \det J_{f^{-1}}- \log p_{\theta}(\textbf{z}) ]
    \end{split}
\end{equation}

The total loss is then:
\begin{equation}
    \begin{split}
        &\mathcal{L} = \mathcal{L}_2 + \mathcal{L}_2\\
        &= \mathbb{E}_{q_{\phi}(\textbf{z}|\mathbb{X})}[\log (p_{\theta}(\mathbb{X}|\textbf{z}))] - \frac{1}{2^M} \sum _{\mathbb{X}_s \in \mathcal{P}(\mathbb{X})} \mathbb{E}_{\textbf{G}_f(\textbf{z}|\textbf{x}_{1:|\xsubset|})}[\log \textbf{G}_f(\textbf{z}|\textbf{x}_{1:|\xsubset|}) + \log \det J_{f^{-1}}- \log p_{\theta}(\textbf{z}) ]\\
        &+ \sum _{\xsetm \in \xset} \mathbb{E}_{z_m \sim \unimodalpost} [\log q_{\phi_m}(z_m|\textbf{x}_M) - \log \det J_f  - \log p_{\theta_m}(f(z_m))]
    \end{split}
\end{equation}

The resulting joint posterior $\textbf{G}_f(\textbf{z}|\textbf{x}_{1:|\xsubset|}) =\sum _{\xsetm \in \xsubset} \frac{f(\unimodalpost)}{|\xsubset|}$ of the \mg{MogfM\_amortized} method can follow any distribution can thus be more expressive than the joint posterior in the \mg{MopgfM} or \mg{MoPoE} methods.
The main disadvantage of this method is that the KL-divergence term in $\mathcal{L}_1$ can only be evaluated when the flow $f$ has already learned to map the uni modal posteriors towards the amortized priors.
In parctice, this makes it very hard to tune the two loss functions to each other.

\subsubsection{iwMogfM}
Like the \mg{MoPoE}, the iwMogfM creates the joint posterior by creating $2^M$ subsets from the uni modal posteriors and then mixing them with a mixture of experts.
However, instead of using a PoE to create the subsets, it uses an $f$-mean.
The iwMogfM also makes use of the importance sampling method from the iwVAE, by sampling K samples from the posterior.
To derive the resulting objective, we rewrite the objective from the \mg{MoPoE} for K importance samples in a first step:
\begin{equation}
    \begin{split}
        \mathcal{L}^{mopoe}_1 &= \mathbb{E}_{\jointpost} \left[ \log \frac{p_{\theta}(\xset, \textbf{z})}{\jointpost} \right]\\
        &=\frac{1}{|\powerset|} \sum _{\xsubset \in \powerset} \mathbb{E}_{\subsetpost} \left[ \log \frac{p_{\theta}(\xsubset, \textbf{z})}{\subsetpost} \right]\\
        &=\frac{1}{|\powerset|} \sum _{\xsubset \in \powerset} \mathbb{E}_{z_s \sim \subsetpost} \left[ \log \frac{p_{\theta}(\xsubset, \textbf{z}_s)}{\tilde{q}_{\phi}(\textbf{z}_s|\xsubset)} \right]\\
        &\leq \frac{1}{|\powerset|} \sum _{\xsubset \in \powerset} \mathbb{E}_{z^{1:K}_s \sim \subsetpost} \left[ \log \frac{1}{K} \sum _{k=1}^K \frac{p_{\theta}(\xsubset, \textbf{z}^k _s)}{\tilde{q}_{\phi}(\textbf{z}^k _s|\xsubset)} \right] = \mathcal{L}^{mopoe}_K
    \end{split}
\end{equation}

Using the fact that the logarithm is concave and Jensens inequality, $\mathcal{L}^{mopoe}_K$ can be rewritten as follows:
\begin{equation}
    \begin{split}
        &\frac{1}{|\powerset|} \sum _{\xsubset \in \powerset} \mathbb{E}_{z^{1:K}_s \sim \subsetpost} \left[ \log \frac{1}{K} \sum _{k=1}^K \frac{p_{\theta}(\xsubset, \textbf{z}^k _s)}{\tilde{q}_{\phi}(\textbf{z}^k _s|\xsubset)} \right]\\
%
        &\geq \frac{1}{|\powerset|} \sum _{\xsubset \in \powerset} \mathbb{E}_{z^{1:K}_s \sim \subsetpost} \left[ \frac{1}{K} \sum _{k=1}^K \log  \frac{p_{\theta}(\xsubset, \textbf{z}^k _s)}{\tilde{q}_{\phi}(\textbf{z}^k _s|\xsubset)} \right]\\
%
        &= \frac{1}{|\powerset|} \sum _{\xsubset \in \powerset} \mathbb{E}_{z^{1:K}_s \sim \subsetpost} \left[ \frac{1}{K} \sum _{k=1}^K \log p_\theta (\xsubset|\textbf{z}^k _s) -\log  \frac{\tilde{q}_{\phi}(\textbf{z}^k _s|\xsubset)}{p_{\theta}(\textbf{z}^k _s)} \right]\\
%
        &= \frac{1}{|\powerset|} \sum _{\xsubset \in \powerset} \mathcal{R}^{1:K}_s - D^{1:K}_s
    \end{split}
\end{equation}

where $\mathcal{R}$ is the reconstruction loss and D the Kl-divergence between the subset posterior and the prior.
The subset posteriors are obtained with an $f$-mean of the uni modal posteriors:
\begin{equation}
    \subsetpost = f^{-1}\left(\sum _{\xsetm \in \xsubset} \frac{f(\unimodalpost)}{|\xsubset|}\right)
\end{equation}

Since the density of the subset posteriors is hard to evaluate, $D^{1:K}_s$ is calculated by comparing K samples from $\subsetpost$ with K samples from the prior $\prior$:
\begin{equation}
    \begin{split}
        D^{1:K}_s &=  \mathbb{E}_{z^{1:K}_s \sim \subsetpost} \left[ \frac{1}{K} \sum _{k=1}^K \log \tilde{q}_{\phi}(\textbf{z}^k _s|\xsubset) - \log p_{\theta}(\textbf{z}^k _s) \right]\\
        %
        &=     \mathbb{E}_{\{z^{1:K}_m \sim \unimodalpost \forall \samplem \in \xsubset\}} \left[ \frac{1}{K} \sum _{k=1}^K \log f^{-1}\left(\sum _{\samplem \in \xsubset} \frac{f(q_{\phi_m}(\textbf{z}_m^k|\samplem))}{|\xsubset|}\right) - \log p_{\theta}(\textbf{z}^k _s) \right] \label{eq:samplestep}\\
        %
%        &\approx \sum ^{\text{batch size}} \frac{1}{K} \sum _{k=1}^K f^{-1}\left(\sum _{\xsetm \in \xsubset} \frac{f(q_{\phi_m}(\textbf{z}_m^k|\xsetm))}{|\xsubset|}\right)(  \log f^{-1}\left(\sum _{\xsetm \in \xsubset} \frac{f(q_{\phi_m}(\textbf{z}_m^k|\xsetm))}{|\xsubset|}\right) - \log p_{\theta}(\textbf{z}^k _s))\\
        %
        &\approx \sum ^{\text{batch size}} \frac{1}{K} \sum _{k=1}^K f^{-1}\left(\sum _{\xsetm \in \xsubset} \frac{f(\mu_m+\sigma_m\ast \epsilon)}{|\xsubset|}\right)(  \log f^{-1}\left(\sum _{\xsetm \in \xsubset} \frac{f(\mu_m+\sigma_m\ast \epsilon)}{|\xsubset|}\right) - \log \epsilon), \epsilon \sim \mathcal{N}(0,\textbf{I})
    \end{split}
\end{equation}

%In \cref{eq:samplestep}, the sampling from the gaussian uni modal posteriors is done with the reparameterisation trick \citep{rezende_stochastic_2014}:
%\begin{equation}
%
%\end{equation}


%\begin{figure}[h!]
%    \centering
%    \resizebox{0.99\textwidth}{!}{%
%        \py{pytex_printonly(script='scripts_/mogfm_graph.py', data = '')}
%    }
%    \caption{\textbf{The iwMogfM makes use of the $f$-mean to create $2^M$ subsets, which are then merged with a MoE.} Here $M=2$, the empty subset is not shown. On the left side are the two input modalities from the polymnist dataset (see \cref{polymnist}), on the right side are the generated samples. In the header of each generated sample is described from which subset the decoder sampled for the generation (left side of the $\rightarrow$) and which modality was generated (right side of the $\rightarrow$).}
%    \label{iwmogfmGraph}
%\end{figure}




    \chapter{Experiments}
In this section we describe the experimental setup that was used in order to compare our methods to each other as well as to the \mg{MVAE}, the \mg{MMVAE} and the \mg{MoPoE} methods.


\section{Datasets}
We evaluate on three datasets, each providing different difficulties in order to filter out advantages and disadvantages of our methods.

\subsection{PolyMNIST} \label{polymnist}
The PolyMNIST dataset, first introduced in \citep{sutter_multimodal_2020}, consists of MNIST digits overlayed over a random part of a certain background image.
The modality specific information of each sample in this dataset is defined by the background image and the shared information by the digit.
In this case the modality specific information is harder to learn than the shared information (for the modality specific information the model has to have learned the set of possible backgrounds and styles of handwriting while the shared information is simply the set of digits).
Examples from the PolyMNIST dataset are shown in \cref{fig:PolyMNIST}.
In total there are 60,000 tuples of training examples and 10,000 tuples of test examples.
The PolyMNIST dataset is useful to study how the number of modalities impacts the performance of multi modal methods, since an abritrary amount of modalities can easily be generated.
We present our results for the PolyMNIST dataset in \cref{subsec: results polymnist}, for which we trained each method for 5 times for 500 epochs.
All results are presented as averages over the 5 runs, accompanied with the standard deviations.
If the number of modalities is not explicitely specified, the model was trained with three modalities from the PolyMNIST dataset.

\begin{figure}[h!]
    \centering
    \includegraphics[width=0.9\textwidth]{data/thesis/polymnist_example}
    \caption{The PolyMNIST dataset consists of sets of MNIST
    digits where each set consists of M images
    with the same digit label but different backgrounds
    and different styles of hand writing for M different modalities.}
    \label{fig:PolyMNIST}
\end{figure}

\subsection{MIMIC-CXR-JPG}

\begin{figure}[h!]
    \centering
    \includegraphics[width=0.9\textwidth]{data/static/mimic_dataset_sample}
    \caption{The PolyMNIST dataset consists of sets of MNIST
    digits where each set consists of M images
    with the same digit label but different backgrounds
    and different styles of hand writing for M different modalities.}
    \label{fig:PolyMNIST}
\end{figure}
%%%%%%%%%%%%%%%%%%%%%%%%%%%%%%%%%%%%%%%%%%%%%%%%%%%%%%%%%%%%%%%%%%%%%%%%%%%%%%%%%%%%%%%%%%%%%%%%%%%%%%%%%%%%%%%%%%%%%%%%%%%%%%%%%%%%%%%%%%%%%%%%%%%%%%%%%%%%%%%%%%%%%%%%%%%%%%%%%%%%

\section{Metrics}
In order to compare the proposed methods in a meaningful manner, we make use of three metrics that each quantifies the performance of a different aspect of the mmVAE.
Namely, we compare the quality of the learned latent representation, the coherence of the generated samples and the quality of the generated samples, as described in the follwing sections.

\subsection{Evaluation of the Latent Representation}
To evaluate if the different mmVAEs are able to extract characteristic information and compress it in the latent representation in a meaningful manner, we evaluate the separability of the latent space via linear classifiers.
If the classifier can separate the latent space into the corresponding classes, we conclude that the posterior approximations are meaningful.
One classifier for each class and for each latent space is trained on the 1000 encoded samples from the training set and tested on the test set.
Note that this can be seen as a variant of the disentanglement metric from \cite{beta_vae} where each class is a different generative factor.
If the dimensions of latent representation are independent and interpretable, there will be less variance in the samples belonging to the same class and thus make them separable from the rest with low capacity classifiers.
It has been shown in \cite{locatello_challenging_2019} that this disentanglement metric correlates with other disentanglement metrics.

\subsection{Evaluation of the generation coherence}
\label{subsubsec:gen_coh}
To evaluate if the method is able to separate the shared information from the modality specific information, we verify that all generated tuples belong to the same class using pretrained classifiers.
For conditional generation, the conditionally generated samples have to be coherent to the input samples.
The coherence accuracy is the ratio of coherent samples divided by the number of generated samples.
For every data type, we train a neural network classifier in a supervised way and the architecture is identical to the encoder except from the last layer.

\subsection{Evaluation of the generation quality}
\label{subsubsec:gen_qual}
To evaluate the quality of the generated samples, we make use of the precision-recall score from \cite{precision_recall_distributions}.
The Precision and Recall for Disitributions (prd) metric is similar to the Fréchet Inception Distance (FID) \citep{heusel_gans_2017}, but disentangles the quality of generated samples from the coverage of the target distribution.
The prd metric reduces the problem of comparing a distribution Q (the distribution of generated samples) to a reference distribution P (the distribution of true images) into a one dimensional problem by applying a pre-trained classifier trained on natural images and to compare \^{P} and \^{Q} at a feature level.
The embeddings are then clustered such that the histogram over the cluster assignments can be meaningfully compared.
Here we compute the prd score by taking the area under the precision-recall curve.

%%%%%%%%%%%%%%%%%%%%%%%%%%%%%%%%%%%%%%%%%%%%%%%%%%%%%%%%%%%%%%%%%%%%%%%%%%%%%%%%%%%%%%%%%%%%%%%%%%%%%%%%%%%%%%%%%%%%%%%%%%%%%%%%%%%%%%%%%%%%%%%%%%%%%%%%%%%%%%%%%%%%%%%%%%%%%%%%%%%%

\section{Hyperparameter Selection}
\label{sec:Hyperparameter Selection}
We select three hyperparameters for the standard mmVAE models (\mg{MoPoE}, \mg{MoE}, \mg{PoE}) that we optimize for our experiments:

\begin{itemize}
    \item The dimension of the latent representation (the bottleneck of the VAE).
    A higher dimensional latent representation gives the model more freedom to separate the different classes and can contain more information in general.
    However, for a too large latent representation, the encoder is not constrained to extract only the most informative features of the input such that the latent representation will contain much information that is non-informative for the decoder.
    \item The learning rate for the stochastic optimization of the parameters, using the Adam optimizer \citep{kingma_adam_2017}.
    For a low learning rate, the objective will take a very long time to converge and for a too high learning rate it might oscillate around a local minimum and never converge.
    \item The $\beta$ in the modified ELBO from \cref{eq:vaeelbo}, described in \cref{subsec:vae}
\end{itemize}

Since the choice for these parameters is non trivial, we optimize them using the hyperparameter optimization framework \dg{Optuna} \citep{akiba_optuna_2019}.
As objective, we use a weighted average of the generation coherence metric (\cref{subsubsec:gen_coh}) and the area Under the precision-recall curve (prd-score, \cref{subsubsec:gen_qual}, where a higher weight is given to the prd-score since its values are generally lower than those of the generation cohrence metric.
The results for the \mg{MoPoE} method can be seen in \cref{fig:mopoe hyperopt}.

For our methods that make use of normalizing flows, we add three additional hyperparameters:
\begin{itemize}
    \item The number of chained transformations that make the flow (Nbr Flows).
    \item The number of coupling block layers per transformation (Nbr Coupling Block layers).
    \item The number of parameters of each coupling block layer Ccoupling Dim).
\end{itemize}

For the optimization of those, we fixed the dimension of the latent representation to 1280 and the learning rate to $5e-4$.
The results can be seen in \cref{fig:mopgfm hyperopt}.
The parameters used for each method in our experiments are shown in \cref{tab:params}.


\section{General Setup}
All parameter values for the experiments studied in \cref{chap:results} can be found in \cref{tab:params}.
We chose to use few normalizing flows to reduce training time and because that yielded more stable results.
We adapted the parameters of the \mg{PoE} and the \mg{MoE} method to match those selected for the \mg{MoPoE} with the hyperoptimization.
Only the dimension of the latent representation (class dim) of the \mg{PoE} was reduced since the performance of the \mg{PoE} dropped significantly with a higher dimension.

\py{
    pytex_tab(
    script='thesis/scripts/params_tab.py',
    options_pre='\\centering \\resizebox{\\textwidth}{!}{',
    options_post='}',
    caption='Parameters used for the models evaluated on the PolyMNIST dataset.',
    label='params'
    )
}


%todo


\section{Reproducibility}
Advances in scientific research are contingent on reproducibility and verifiability of previous work.
To ensure this, we make the framework used to train all models evaluated in this work available as an open source python package \citep{mmvae_github}, tested with continuous integration using \citep{travis} and kept up to date with \citep{dependabot}.
We publish this thesis as a reproducible self publishing document \citep[\href{https://github.com/TheChymera/RepSeP}{RepSeP}]{repsep} made available on GitHub \citep{mmnf_repsep}.
Using \LaTeX and PythonTeX \citep{pytex}, we make all steps described herein easily reexecutable and extendable.
    \section{Results}
% First compare mopoe, mopgfm, mogfm
% Then compare iwmopoe, iwmogfm, mogfm

% try other datasets also (mimic,..)
% Train 5 times: mopoe, mopgfm, mogfm

% also compare training times


My results.


\begin{sansmath}
    \py{pytex_subfigs(
        [
            {'script':'thesis/scripts/plots/epoch_comparison_lr.py', 'label':'vccv','conf':'thesis/4*2.conf', 'options_pre':'{.48\\textwidth}',
        'options_pre_caption':'\\vspace{-1.5em}\\',
        'options_post':'\\vspace{1em}',
        'caption':''
        ,},
            {'script':'thesis/scripts/plots/epoch_comparison_gen.py', 'label':'sccv','conf':'thesis/4*2.conf', 'options_pre':'{.48\\textwidth}',
        'options_pre_caption':'\\vspace{-1.5em}\\',
        'options_post':'\\vspace{1em}',
        'caption':''
        ,},
            {'script':'thesis/scripts/plots/nbr_mods_comparison_lr.py', 'label':'vcfb','conf':'thesis/4*2.conf', 'options_pre':'{.48\\textwidth}',
        'options_pre_caption':'\\vspace{-1.5em}\\',
        'options_post':'\\vspace{1em}',
        'caption':''
        ,},
            {'script':'thesis/scripts/plots/nbr_mods_comparison_gen.py', 'label':'scfb','conf':'thesis/4*2.conf', 'options_pre':'{.48\\textwidth}',
        'options_pre_caption':'\\vspace{-1.5em}\\',
        'options_post':'\\vspace{1em}',
        'caption':''
        ,},
        ],
        caption='
        ',
        label='fig:vc',
        )}
\end{sansmath}

\py{
    pytex_tab(
    script='thesis/scripts/gen_eval_tab.py',
    options_pre='\\centering \\resizebox{0.7\\textwidth}{!}{',
    options_post='}',
    caption='Generation coherence for the Test set.
    ',
    )
}

\py{
    pytex_tab(
    script='thesis/scripts/lr_eval_tab.py',
    options_pre='\\centering \\resizebox{0.99\\textwidth}{!}{',
    options_post='}',
    caption='Linear classification accuracy of latent representations for the Test set.\\
    ',
    )
}

\py{
    pytex_tab(
    script='thesis/scripts/prd_tab.py',
    options_pre='\\centering \\resizebox{0.7\\textwidth}{!}{',
    options_post='}',
    caption='Area under the Precision and Recall curve of the PRD score \citep{precision_recall_distributions}.\\
    ',
    )
}
    % todo say that it would be interesting to compare how increasing capacity of encoder/decoder model would increase performance in comparison with increasing number of flows for flow methods
\section{Conclusion \& Discussion}



    \printbibliography
    \includepdf[pages={-}]{thesis/declaration-originality.pdf}


    \appendix

%\chapter{Qualitative comparison of generated PolyMNIST samples}
\input{thesis/gen_comp_polymnist}


\begin{figure}[h!]
    \centering
    \resizebox{0.7\textwidth}{!}{%
        \py{pytex_printonly(script='thesis/scripts/tikz_graphs/rand_gen_comp_polymnist.py', data = '')}
    }
    \caption{\textbf{Comparison of randomly generated samples between methods.} The samples are generated by sampling from the prior and decoding them with a randomly selected decoder from the modalities $m_0$, $m_1$, $m_2$.
    % todo give the random generation quality scores achieved for each model
    }
\end{figure}

\chapter{Qualitative comparison of generated Mimic-CXR samples}
\input{thesis/gen_comp_mimic}

    \backmatter


\end{document}
