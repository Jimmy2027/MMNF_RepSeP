% pythontex
%% (Master) Thesis template
% Template version used: v1.4
%
% Largely adapted from Adrian Nievergelt's template for the ADPS
% (lecture notes) project.


%% We use the memoir class because it offers a many easy to use features.
\documentclass[11pt,a4paper,titlepage,english]{memoir}

%% Packages
%% ========

%% LaTeX Font encoding -- DO NOT CHANGE
\usepackage[OT1]{fontenc}

\usepackage[british]{babel} % decent hyphenation, avoiding e.g. anal-ysis
\usepackage[iso]{isodate}
\usepackage{sansmath}
\usepackage{booktabs}
\usepackage{graphicx}
\usepackage{graphviz}
\usepackage{makecell}
\usepackage{minted}
\usepackage{siunitx}
\usepackage{subcaption}
\usepackage[section]{placeins}
\usepackage{amsfonts} % needed for \mathbb{} (Only works on capital letters!)
\usepackage{tikz}
\usetikzlibrary{shapes,snakes}
%\usetikzlibrary{shapes.geometric}

\usepackage{hyperref}
\usepackage{amsmath}
% Needs to be loaded after hyperref and amsmath
\usepackage{cleveref}

% new commands
\newcommand{\xdensity}{\textit{p}_x (\textbf{x})}
\newcommand{\udensity}{\textit{p}_u (\textbf{u})}
\newcommand{\transformer}{\tau(z_i;\textbf{h}_i)}
\newcommand{\conditioner}{c_i(\textbf{z}_{<1})}
\newcommand{\where}{\quad \text{where} \quad}
\newcommand{\xset}{\mathbb{X}}
\newcommand{\xseti}{\mathbb{X}^{(i)}}
\newcommand{\approxdistri}{q_{\phi_i}(\textbf{z}|\xseti)}
\newcommand{\approxdistr}{q_{\phi, \psi}(\textbf{z}|\xset)}
\newcommand{\truedistri}{p_{\theta}(\textbf{z}|\xseti)}
\newcommand{\truedistr}{p_{\theta}(\textbf{z}|\textbf{X})}
\newcommand{\elbo}{\mathcal{L}(\theta, \phi; \xseti)}
\newcommand{\xsubset}{\mathbb{X}_k}
\newcommand{\lmopoe}{\mathcal{L}_{MoPoE}(\theta, \phi; \xset)}
\newcommand{\powerset}{\mathcal{P}(\xset)}
\newcommand{\pFmean}{\mathcal{M}_{f_{\psi}}}
\newcommand{\DklTrueApprox}{D_{KL} \left( \approxdistr || \truedistr \right)}
\newcommand{\Mnfi}{\mathcal{M}_{f_{\psi}}\left( q_{\phi _i}(\textbf{z}|\textbf{x}_i) \right)}




% PythonTeX
\usepackage[autoprint=false, gobble=auto, keeptemps=all, pyfuture=all]{pythontex} % create figures on-line directly from python!
\usepackage{pgf}
\begin{pythontexcustomcode}[begin]{py}
import os, sys
almost_this_path = os.path.abspath(os.path.dirname(__file__))
this_path_base, _ = os.path.split(almost_this_path)
this_path = os.path.join(this_path_base,"pythontex")

from pylab import gcf
import matplotlib
import matplotlib.pyplot as plt

pytex.add_dependencies(os.path.join(this_path,'matplotlibrc.conf'))

plt.style.use(os.path.join(this_path,'matplotlibrc.conf'))

# Set the prefix used for figure labels
fig_label_prefix = 'fig'
# Track figure numbers to create unique auto-generated names
fig_count = 0

def figure_by_path(figure_path,textheight_frac=1,caption=None,label=None):
    latex_code = "\\begin{figure}\n"
    latex_code += "\\centering\\includegraphics[width={textheight_frac}\\textheight]{{{figure_path}}}\n".format(textheight_frac=textheight_frac,figure_path=figure_path)
    latex_code += "\\vspace{-2.5em}\n"
    latex_code += "\\caption{{{caption}}}\n".format(caption=caption)
    latex_code += "\\label{{fig:{label}}}\n".format(label=label)
    latex_code += "\\end{figure}\n"
    return latex_code

def save_fig(name='', legend=False, fig=None, ext='.pgf', fig_width=1, fig_height=1):
    '''
    Save the current figure (or `fig`) to file using `plt.save_fig()`.
    If called with no arguments, automatically generate a unique filename.
    Return the filename.
    '''
    # Get name (without extension) and extension
    if not name:
        global fig_count
        # Need underscores or other delimiters between `input_*` variables
        # to ensure uniqueness
        name = 'auto_fig_{}-{}'.format(pytex.id, fig_count)
        fig_count += 1
    else:
        if len(name) > 4 and name[:-4] in ['.pgf', '.svg', '.png', '.jpg']:
            name, ext = name.rsplit('.', 1)

    # Get current figure if figure isn't specified
    if not fig:
        fig = gcf()
    fig.set_size_inches(fig_width,fig_height)
    fig.savefig(name + ext)
    fig.clf()
    return name

def latex_environment(name, content='', option=''):
    '''
    Simple helper function to write the `\begin...\end` LaTeX block.
    '''
    return '\\vspace{-0.25cm}\\begin{%s}%s\n%s\n\\end{%s}' % (name, option, content, name)

def latex_figure(name=None, caption='', label='', width=1):
    ''''
    Auto wrap `name` in a LaTeX figure environment.
    Width is a fraction of `\textwidth`.
    '''
    if not name:
        name = save_fig()
    content = '\\centering\n'
    content += '\\makeatletter\\let\\input@path\\Ginput@path\\makeatother\n'
    content += '\\input{%s.pgf}\n' % name
    if not label:
        label = name
    if caption and not caption.rstrip().endswith('.'):
        caption += '.'
    if caption:
        # `\label` needs to be in `\caption` to avoid issues in some cases
        content += "\\caption{%s\\label{%s:%s}}\n" % (caption, fig_label_prefix, label)
    return latex_environment('figure', content, '[htp]')

pytex.bio_fignum = 0
#global pytex # try without this line
def bio_fig(gdd, fname=None, caption=None, label=None):
#        global pytex # and this one, should work
        if fname is None:
            fname = 'pythontex-files-pres/biopython_fig_{0}-{1}.pdf'.format(pytex.id, pytex.bio_fignum)
        gdd.write(fname, "PDF")
        template = '''
    \\begin{{figure}}
    \\centering
    \\includegraphics{{{fname}}}
    \\caption{{ {label} {caption} }}
    \\end{{figure}}
    '''
        if caption is None:
            caption = ''
        if label is None:
            label = ''
        else:
            if not label.startswith('fig:'):
                label = 'fig:' + label
            label = '\\label{{{0}}}'.format(label)
        template = template.format(fname=fname.rsplit('.', 1)[0], label=label, caption=caption)
        print(template)
        pytex.add_created(fname)
        pytex.bio_fignum += 1
        return template
\end{pythontexcustomcode}
\begin{pythontexcustomcode}[end]{py}
\end{pythontexcustomcode}

\begin{pythontexcustomcode}[begin]{py}
pytex.add_dependencies(
	'lib/utils.py',
	'lib/categorical.py',
	)
\end{pythontexcustomcode}
% Single-session PythonTeX codeblocks
\newcounter{pysessioncounter}
\newcommand{\sessionpy}{%
          \edef\sessionpysession{session\arabic{pysessioncounter}}%
            \stepcounter{pysessioncounter}%
              \expandafter\py\expandafter[\sessionpysession]}

% SIunitx customizations detect-all will use the current font for typesetting
\sisetup{per-mode=symbol, detect-all, range-units = single}
\newcommand\SIci[5]{\SI{#1}{#2}, {#3}CI: \SIrange{#4}{#5}{#2}}

% Fix for matplotlib PGF wonkiness which isn't interpreted correctly by pdflatex
\DeclareUnicodeCharacter{2212}{-}

%% Input encoding 'utf8'. In some cases you might need 'utf8x' for
%% extra symbols. Not all editors, especially on Windows, are UTF-8
%% capable, so you may want to use 'latin1' instead.
\usepackage[utf8]{inputenc}

%% This changes default fonts for both text and math mode to use Herman Zapfs
%% excellent Palatino font.  Do not change this.
\usepackage[sc]{mathpazo}

%% The AMS-LaTeX extensions for mathematical typesetting.  Do not
%% remove.
\usepackage{amsmath,amssymb,amsfonts,mathrsfs}

%% NTheorem is a reimplementation of the AMS Theorem package. This
%% will allow us to typeset theorems like examples, proofs and
%% similar.  Do not remove.
%% NOTE: Must be loaded AFTER amsmath, or the \qed placement will
%% break
\usepackage[amsmath,thmmarks]{ntheorem}

%% This allows you to add .pdf files. It is used to add the
%% declaration of originality.
\usepackage{pdfpages}

%BIBLIOGRAPHY
\usepackage[backend=bibtex,style=authoryear,natbib=true]{biblatex}

%% Some more packages that you may want to use.  Have a look at the
%% file, and consult the package docs for each.
%% See the TeXed file for more explanations

% my stuff
\usepackage{color}

%% [OPT] Multi-rowed cells in tabulars
%\usepackage{multirow}

%% [REC] Intelligent cross reference package. This allows for nice
%% combined references that include the reference and a hint to where
%% to look for it.
\usepackage{varioref}

%% [OPT] Easily changeable quotes with \enquote{Text}
%\usepackage[german=swiss]{csquotes}

%% [REC] Format dates and time depending on locale
\usepackage{datetime}

%% [OPT] Provides a \cancel{} command to stroke through mathematics.
%\usepackage{cancel}

%% [NEED] This allows for additional typesetting tools in mathmode.
%% See its excellent documentation.
\usepackage{mathtools}

%% [ADV] Conditional commands
%\usepackage{ifthen}

%% [OPT] Manual large braces or other delimiters.
%\usepackage{bigdelim, bigstrut}

%% [REC] Alternate vector arrows. Use the command \vv{} to get scaled
%% vector arrows.
\usepackage[h]{esvect}

%% [NEED] Some extensions to tabulars and array environments.
\usepackage{array}

%% [OPT] Postscript support via pstricks graphics package. Very
%% diverse applications.
%\usepackage{pstricks,pst-all}

%% [?] This seems to allow us to define some additional counters.
%\usepackage{etex}

%% [ADV] XY-Pic to typeset some matrix-style graphics
%\usepackage[all]{xy}

%% [OPT] This is needed to generate an index at the end of the
%% document.
%\usepackage{makeidx}

%% [OPT] Fancy package for source code listings.  The template text
%% needs it for some LaTeX snippets; remove/adapt the \lstset when you
%% remove the template content.
\usepackage{listings}
\lstset{language=TeX,basicstyle={\normalfont\ttfamily}}

%% [REC] Fancy character protrusion.  Must be loaded after all fonts.
\usepackage[activate]{pdfcprot}

%% [REC] Nicer tables.  Read the excellent documentation.
\usepackage{booktabs}


%% Our layout configuration.  DO NOT CHANGE.
%% Memoir layout setup

%% NOTE: You are strongly advised not to change any of them unless you
%% know what you are doing.  These settings strongly interact in the
%% final look of the document.

% Dependencies
\usepackage{thesis/ETHlogo}

% Turn extra space before chapter headings off.
\setlength{\beforechapskip}{0pt}

\nonzeroparskip
\parindent=0pt
\defaultlists

% Chapter style redefinition
\makeatletter

\if@twoside
  \pagestyle{Ruled}
  \copypagestyle{chapter}{Ruled}
\else
  \pagestyle{ruled}
  \copypagestyle{chapter}{ruled}
\fi
\makeoddhead{chapter}{}{}{}
\makeevenhead{chapter}{}{}{}
\makeheadrule{chapter}{\textwidth}{0pt}
\copypagestyle{abstract}{empty}

\makechapterstyle{bianchimod}{%
  \chapterstyle{default}
  \renewcommand*{\chapnamefont}{\normalfont\Large\sffamily}
  \renewcommand*{\chapnumfont}{\normalfont\Large\sffamily}
  \renewcommand*{\printchaptername}{%
    \chapnamefont\centering\@chapapp}
  \renewcommand*{\printchapternum}{\chapnumfont {\thechapter}}
  \renewcommand*{\chaptitlefont}{\normalfont\huge\sffamily}
  \renewcommand*{\printchaptertitle}[1]{%
    \hrule\vskip\onelineskip \centering \chaptitlefont\textbf{\vphantom{gyM}##1}\par}
  \renewcommand*{\afterchaptertitle}{\vskip\onelineskip \hrule\vskip
    \afterchapskip}
  \renewcommand*{\printchapternonum}{%
    \vphantom{\chapnumfont {9}}\afterchapternum}}

% Use the newly defined style
\chapterstyle{bianchimod}

\setsecheadstyle{\Large\bfseries\sffamily}
\setsubsecheadstyle{\large\bfseries\sffamily}
\setsubsubsecheadstyle{\bfseries\sffamily}
\setparaheadstyle{\normalsize\bfseries\sffamily}
\setsubparaheadstyle{\normalsize\itshape\sffamily}
\setsubparaindent{0pt}

% Set captions to a more separated style for clearness
\captionnamefont{\sffamily\bfseries\footnotesize}
\captiontitlefont{\sffamily\footnotesize}
\setlength{\intextsep}{16pt}
\setlength{\belowcaptionskip}{1pt}

% Set section and TOC numbering depth to subsection
\setsecnumdepth{subsection}
\settocdepth{subsection}

%% Titlepage adjustments
\pretitle{\vspace{0pt plus 0.7fill}\begin{center}\HUGE\sffamily\bfseries}
\posttitle{\end{center}\par}
\preauthor{\par\begin{center}\let\and\\\Large\sffamily}
\postauthor{\end{center}}
\predate{\par\begin{center}\Large\sffamily}
\postdate{\end{center}}

\def\@advisors{}
\newcommand{\advisors}[1]{\def\@advisors{#1}}
\def\@department{}
\newcommand{\department}[1]{\def\@department{#1}}
\def\@thesistype{}
\newcommand{\thesistype}[1]{\def\@thesistype{#1}}

\renewcommand{\maketitlehooka}{\noindent\ETHlogo[2in]}

\renewcommand{\maketitlehookb}{\vspace{1in}%
  \par\begin{center}\Large\sffamily\@thesistype\end{center}}

\renewcommand{\maketitlehookd}{%
  \vfill\par
  \begin{flushright}
    \sffamily
    \@advisors\par
    \@department, ETH Z\"urich
  \end{flushright}
}

\checkandfixthelayout

\setlength{\droptitle}{-48pt}

\makeatother

% This defines how theorems should look. Best leave as is.
\theoremstyle{plain}
\setlength\theorempostskipamount{0pt}

%%% Local Variables:
%%% mode: latex
%%% TeX-master: "thesis"
%%% End:


%% Theorem environments.  You will have to adapt this for a German
%% thesis.
%% Theorem-like environments

%% This can be changed according to language. You can comment out the ones you
%% don't need.

\numberwithin{equation}{chapter}

%% German theorems
%\newtheorem{satz}{Satz}[chapter]
%\newtheorem{beispiel}[satz]{Beispiel}
%\newtheorem{bemerkung}[satz]{Bemerkung}
%\newtheorem{korrolar}[satz]{Korrolar}
%\newtheorem{definition}[satz]{Definition}
%\newtheorem{lemma}[satz]{Lemma}
%\newtheorem{proposition}[satz]{Proposition}

%% English variants
\newtheorem{theorem}{Theorem}[chapter]
\newtheorem{example}[theorem]{Example}
\newtheorem{remark}[theorem]{Remark}
\newtheorem{corollary}[theorem]{Corollary}
\newtheorem{definition}[theorem]{Definition}
\newtheorem{lemma}[theorem]{Lemma}
\newtheorem{proposition}[theorem]{Proposition}

%% Proof environment with a small square as a "qed" symbol
\theoremstyle{nonumberplain}
\theorembodyfont{\normalfont}
\theoremsymbol{\ensuremath{\square}}
\newtheorem{proof}{Proof}
%\newtheorem{beweis}{Beweis}


%% Helpful macros.
%% Custom commands
%% ===============

%% Fixed/scaling delimiter examples (see mathtools documentation)
\DeclarePairedDelimiter\abs{\lvert}{\rvert}
\DeclarePairedDelimiter\norm{\lVert}{\rVert}

%% Use the alternative epsilon per default and define the old one as \oldepsilon
\let\oldepsilon\epsilon
\renewcommand{\epsilon}{\ensuremath\varepsilon}

%% Also set the alternate phi as default.
\let\oldphi\phi
\renewcommand{\phi}{\ensuremath{\varphi}}


%% Make document internal hyperlinks wherever possible. (TOC, references)
%% This MUST be loaded after varioref, which is loaded in 'extrapackages'
%% above.  We just load it last to be safe.
\usepackage[linkcolor=black,colorlinks=true,citecolor=black,filecolor=black]{hyperref}
\addbibresource{bib.bib}
\usepackage{cleveref}

% pythontex dependencies
\begin{pythontexcustomcode}[begin]{py}
    DOC_STYLE="thesis/main.conf"
    pytex.add_dependencies(
    'thesis/scripts/plots/epoch_comparison_lr.py',
    'thesis/scripts/plots/epoch_comparison_gen.py',
    'thesis/scripts/plots/nbr_mods_comparison_gen.py',
    'thesis/scripts/plots/nbr_mods_comparison_lr.py',
    'thesis/scripts/plots/utils.py',
    'thesis/scripts/tikz_graphs/rand_gen_comp_polymnist.py',
    'thesis/scripts/tikz_graphs/mimic_lat_pa_example.py',
    'thesis/scripts/gen_eval_tab.py',
    'thesis/scripts/params_tab_polymnist.py',
    'thesis/scripts/params_tab_mimic.py',
    'scripts_/mopoe_graph.py',
    'scripts_/mogfm_graph.py',
    'scripts_/mopgfm_graph.py',
    'scripts_/mofop_graph.py',
    'lib/utils.py',
    'data/thesis/gen_eval.csv',
    DOC_STYLE
    )
\end{pythontexcustomcode}

%% Document information
%% ====================

\title{Multi Modal Generative Learning\\ utilizing Normalizing Flows \\}
\author{Hendrik J. Klug}
\thesistype{Master Thesis}
\advisors{Advisors: Prof.\ Dr.\ Julia Vogt, MSc.\ Thomas M. Sutter}
\department{Department of Computer Science}
\date{}

\begin{document}

    \frontmatter

%% Title page is autogenerated from document information above.  DO
%% NOT CHANGE.
    \begin{titlingpage}
        \calccentering{\unitlength}
        \begin{adjustwidth*}{\unitlength-24pt}{-\unitlength-24pt}
            \maketitle
        \end{adjustwidth*}
    \end{titlingpage}

%% The abstract of your thesis.  Edit the file as needed.
    \begin{abstract}
  This example thesis briefly shows the main features of our thesis
  style, and how to use it for your purposes.
\end{abstract}


%% TOC with the proper setup, do not change.
    \cleartorecto
    \tableofcontents
    \mainmatter

    \chapter{Introduction}
% in recent years, the field of deep learning has seen a shift from uni modal learning towards multi modal learning.
%motivation for multi modal learning
The availability of multiple data types provides a rich source of information and holds promise for learning representations that generalise well across multiple modalities \citep{baltrusaitis_multimodal_2019}.
Similar to how humans learn and extract information from their surroundings using an aggregation of their senses, a machine learning model can learn from multiple data types.
Multimodal data naturally grants self-supervision in the form of shared information connecting the different data types.
It also serves as an inherent regularization which forces the model to learn more robust features from the data, since these features need to be connected between modalities.
This may lead to more interpretable features for humans since they also infer from multiple modalities.
A model that can generate any of the learned modalities, given any subset of modalities can be used for translation between modalities for example, such as image captioning.
It can also find applications in the medical domain, where the model could generate, conditioned on images and medical data of a patient, a text describing the medical condition of a patient.
%Self-supervised training paradigms are especially useful in the medical domain since there labeled data is expensive to acquire and thus very scarce.
% examples: (\citep{dorent_hetero-modal_2019}, \citep{calixto_latent_2019})

% motivation for generative models
However, the understanding of different modalities and the interplay between data types are non-trivial research questions and longstanding goals in machine learning research \citep{ngiam_multimodal_nodate}.
While fully supervised approaches have been applied successfully \citep{karpathy_deep_2015,tsai_learning_2018}, the labeling of multiple data types remains time-consuming and expensive.
Therefore, models that efficiently learn from multiple data types in a self-supervised fashion are much more widely applicable for real world problems.
In the medical domain, for example, self-supervised training paradigms are especially useful since there labeled data is expensive to acquire and thus very scarce.
%Therefore, it requires models that efficiently learn from multiple data types in a self-supervised fashion.
Generative models represent a natural way to learn underlying generative factors of the data, in a self-supervised fashion.



Self-supervised, multi modal generative models have been applied to toy datasets \citep{poe, shi_variational_2019, sutter_generalized_2020} and real world data \citep{klug_multimodal_nodate}, however results have shown that current methods are not able to aggregate well enough over the modalities to generate coherent samples.
For the model to generate coherent samples, it needs to extract and fuse information from the multiple data types.
An image captioning model for example, needs to extract information from the image and generate text from it when generating the caption for an image of a green apple.
Captions such as "A red apple." or "A yellow truck." would not be coherent with the image of a green apple.

In previous work, the aggregation over modalities is done with multiple, fixed, pre-selected methods, each coming with advantages and disadvantages.
Here, we generalise previous work by implementing the aggregation over modalities using a generic function with trainable parameters.
We show that this more flexible way to fuse the information between modalities improves the ability of the model to generate coherent samples across data types.
    \chapter{Related Work}
\section{Generative Modeling}
Generative adversarial networks \citep[GANs]{goodfellow_generative_2014} and variational autoencoders (VAEs, \cref{subsec:vae}) are the two most popular methods for generative modeling.
Both attempt to model the distribution over the data, however while for the VAEs, the resulting posterior approximation is defined explicitly, the learned posterior of GANs can not be evaluated directly.
GANs are made of two models that are trained simultaneously, a generative model G that captures the data distribution, and a discriminative model D that estimates the probability that a sample came from the training data rather than G.
The joint optimization of both models D and G can be tricky in practice and GANs are known to suffer from mode collapse since the objective does not require the learned representation to contain all modes of the data.
For images of animals for example, the generator G could learn to generate only images of brown, short haired dogs, so well that the discriminator D will not be able to distinguish them from the true data.
Mode collapse does not happen in VAEs since their objective explicitly requires their learned representation to contain all modes of the data.
Also, since the learned posterior distribution of VAEs can be evaluated explicitly, additional constraints can be added to the objective to push the posterior distribution to have specific characteristics and it can be used for downstream tasks like clustering or classification.
In this work, we focus on VAEs and give a more in depth introduction in section \cref{subsec:vae}.

\section{Multi Modal Generative Modeling}

There have been a wide range of approaches for multi-modal generative modeling, however most fall short of expressing the complete range of behaviour that we expect in this setting.

\paragraph{Modality Translation}
Most prior approaches to generative modelling with multi modal data have targeted modality translation, where the model learns to generate one modality conditioned on another one.
In this case input an output modalities of the model are not interchangeable.
Modality translation has been proposed both as VAE based \citep{pu2016variational, pandey2017variational}, as well as GAN based, for domain translation of images \citep{ledig2017photo, liu2019few}.
However, we expect our method to be able to generate any modality given any subset of modalities which extends translation between modalities.
It would be possible to train $2^M -1$ modality translation network pairs for $M$ modalities, but this is intractable in practice.

\paragraph{Joint approximation}
Other prior work has targeted to directly model the joint distribution over the data.
The joint multi modal VAE (JMVAE) from \citep{suzuki2016joint} learns a joint posterior distribution using a joint inference network.
To handle missing data at test time, inference networks need to be trained for every subset of modalities.
While feasible for two modalities, this setup quickly becomes intractable with more data types.
Similarly, the multimodal factorisation model (MFM) from \citep{tsai2018learning} explicitly defines a joint inference network on top of uni modal encoders, however additional decoder networks are needed to generate missing modalities.

These approaches typically do not scale well with the number of modalities since they require additional modelling components for each combination of modalities.
The MVAE from \citep{poe} marked an improvement over previous methods in this regard, proposing to model the joint posterior as a product of experts (POE) over the marginal posteriors, enabling cross-modal generation at test-time without requiring additional inference networks and multi-stage training regimes.
Since then, other methods have emerged, each proposing another aggregation function over the marginal posteriors.
We refer to \cref{subsec:Multi Modal VAEs} for a more in depth introduction to the MVAE and other methods that build on it.

Next to the aggregation function with which the uni modal posteriors are merged, other methods have been proposed to improve multi modal VAEs (mmVAEs).
In \citep{daun_disent}, the authors propose to split the latent space into modality specific and shared information in order to disentangle \citep{burgess_understanding_2018} them in a purely self-supervised manner.
The aggregation of modalities should only happen over the shared information and thus it makes sense to separate it from the modality specific information in order to simplify the aggregation.
For this, the authors add a new term to the mmVAE objective, which disentangles the shared representations with the modality specific representations and encourages mutual information between representations that contain shared information.
This has been shown to improve the conditional generation of missing modalities, however the results from \citep{sutter_multimodal_2020} point out that independent of that separation, the generation coherence differs between different merging functions.
The goal of this work is solely to improve the merging function, which is why we forgo this method even though we expect the separation of shared and modality specific information to improve our results.

    \section{Background}
Some background.
    \section{Methods}
\label{sec:methods}
As introduced herein, we are working with a multi modal VAE (mmVAE), which learns a joint distribution that contains the combined information of each learned uni modal latent distribution.
In order to generalize previous methods and to increase the flexibility of the combination of the modalities, we implement the fusion of the uni modal latent distributions with a mixture of generalized $f$-mean.
Instead of merging the uni modal posteriors into subset-posteriors with a PoE, like is done in the \mg{MoPoE}, we merge them with a trainable $f_{\psi}$-mean, with parameters $\psi$.
The main difficulty in this approach comes from the fact that the $f$-mean of the uni modal distributions follows an unknown distribution.
While this makes the joint distribution more flexible, this also makes the computation of the regularization term in the ELBO, the KL-divergence, more difficult to compute.
In fact, if the density of the joint distribution is unknown, it is impossible to compute the KL-divergence in closed form.

An intuitive alternative would be to find an upper bound of the KL-divergence which can be computed in closed from, such that it can be minimized in order to minimize the true divergence:
\begin{equation}
    \label{eq:kldivbound}
    D_{KL}^{\prime} \geq D_{KL}(\mathcal{M}_f(\{\unimodalpost\ \forall\ \xsetm \in \xsubset\})) =  D_{KL}\left(f^{-1}\left(\sum _{\xsetm \in \xsubset} \frac{f(\unimodalpost)}{|\xsubset|}\right)\ ||\ \prior\right)
\end{equation}

Using the change of variable formula (\cref{eq:changeofvariables}), the $f$-mean can be rewritten as follows:
\begin{equation}
    \mathcal{M}_f = f^{-1}(Q)|J_{f^{-1}}(Q)|
\end{equation}
with
\begin{equation}
    Q= \sum _{\xsetm \in \xsubset} \frac{\unimodalpost|J_f(\unimodalpost)|}{|\xsubset|}
\end{equation}

Here Q is a sum of random variables, which can be rewritten as chained convolutions \footnote{\url{https://en.wikipedia.org/wiki/Sum_of_normally_distributed_random_variables}} and is hard to evaluate.

Instead, we propose three workarounds to the computation of the KL-divergence in \cref{eq:kldivbound}.
\begin{enumerate}

    \item For one, \cref{eq:kldivbound} can be simplified by skipping the backwards transformation $f^{-1}$.
    This leads to a mixture of transformed posteriors, which divergence can be bounded using \cref{lemma:DklLowerBound} from \parencite{sutter_multimodal_2020}.

    We then get an upper bound that can be minimized:
    \begin{equation}
        D_{KL}\left(\sum _{\xsetm \in \xsubset} \frac{f_{\psi}(\unimodalpost)}{|\xsubset|}\ ||\ \prior\right) \leq \frac{1}{|\xsubset|} \sum  _{\xsetm \in \xsubset} D_{KL} \left(f_{\psi} (\unimodalpost))||\ \prior \right) \quad (\cref{lemma:DklLowerBound})
    \end{equation}
% todo write all method names as grey ?
    We implement this in the Mixture of flow of product of experts (\mg{MofoPoE}) model, which is described in \cref{subsec:mofopoe}.

% todo beschreib mofop method and sag vorteile und nachteile von dieser Methods: man macht die einzelnen posterior besser aber man macht das mergen der information nicht flexibler
    % todo sag dass weil das nicht so gut funktioniert, this hints that the true problem is not the merging method of the information but the form of the posterior.

    \item Another way to simplify the Kl-divergence in \cref{eq:kldivbound} is to force the output of the $f$-mean to be a Gaussian distribution.
    This can be done by, instead of mixing the posteriors which follow a normal distribution, mixing their parameters $\mu_s$ and $\sigma_s$.
    The joint posterior is then described as follows:
    \begin{equation}
        \label{eq:qjointmopgfm}
        q_{\phi, joint} \sim \mathcal{N}\left(  f_{\mu}^{-1}(\sum _{\xsetm \in \xsubset} \frac{f_{\mu}(\mu_s)}{|\xsubset|}),\ f_{\sigma}^{-1}(\sum  _{\xsetm \in \xsubset} \frac{f_{\sigma}(\sigma_s^2)}{|\xsubset|})\right)
    \end{equation}

    This is implemented as the mixture of parameter generalized $f$-mean (\mg{MopgfM}) and described in \cref{subsec:mopgfm}.

    \item The sum of random variables in the $f$-mean (\cref{eq:kldivbound}) is hard to evaluate since the transformed uni modal posteriors ($\unimodalpost$) follow an unknown distribution.
    It is however possible to steer the normalizing flow $f_{\psi}$ to map towards $\unimodalpost$ a normal distribution, such that the sum of random variables can be evaluated.
    This normal distribution can be amortized by making it dependent on the input.
    We implement this as the \mg{MogfM\_amortized} method, described in \cref{subsubsec:mogfm_amortized}.

    \item The third option is to instead of computing the KL-divergence in closed form, one can approximate it by sampling from the posterior, i.e. comparing k samples from the posterior with k samples from the prior.
    This method is similar to the importance weighted VAE \parencite[iwVAE]{burda_importance_2016}, however since the Kl-divergence is not computed is closed form but approximated with the K samples, a higher number of K is required to obtain good results.
    This method is implemented as the importance weighted mixture of generalized $f$-mean (iwMogfM) and described in %todo.
\end{enumerate}

\subsection{Models}
Here we describe the models that implement the three methods introduced above and enumerate their advantages and disadvantages.

\subsubsection{MofoPoE}\label{subsec:mofopoe}
The \mg{MofoPoE} builds on the \mg{MoPoE} by transforming the subset posteriors $\subsetpost$ with a series of F invertible transformations with trainable parameters $\psi$:
\begin{equation}
    z_{F,S} =f_{\psi}(z_{0,S} \sim \subsetpost) = f_F \circ \ldots \circ f_2 \circ f_1(z_{0,S} \sim \subsetpost)
\end{equation}

The density of the resulting transformed subset posterior can be evaluated with the change of variables formula (\cref{eq:changeofvariables}):
\begin{equation}
    \label{eq:changeofvariables_}
    \ln f(\subsetpost) = \ln q_\phi (z_0|\xsubset) - \sum _{i=1} ^{F}\ln \left|  \det \frac{df_i}{dz_{i-1}}\right|
\end{equation}

Here $f(\subsetpost)$ can follow any distribution is thus more flexible than the gaussian subset posterior in the \mg{MoPoE} model.
A flow chart depiction of the \mg{MofoPoE} is shown in \cref{fig:mofopoe}.
Effectively, during a forward pass, the reparameterisation happens at the subset posteriors.
I.e. a sample is taken from each subset posterior, transformed with a normalizing flow $f$ and then mixed with a MoE.

The resulting objective can be written as follows, by slightly modifying the \mg{MoPoE} objective from \cref{eq:mopoe_}:

\begin{equation}
    \begin{split}
        \mathcal{L}_{\mg{MofoPoE}}(\theta, \phi, \psi; \xset) &=  \mathbb{E}_{q_{\phi}(\textbf{z}|\mathbb{X})}[\log (p_{\theta}(\mathbb{X}|\textbf{z}))] - \frac{1}{2^M} \sum _{\mathbb{X}_s \in \mathcal{P}(\mathbb{X})} D_{KL}\biggl( \tilde{q}_{\phi}(\textbf{z}|\mathbb{X}_s)\ ||\ p_{\theta}(\textbf{z})\biggr)\\
        &= \mathbb{E}_{q_{\phi}(\textbf{z}|\mathbb{X})}[\log (p_{\theta}(\mathbb{X}|\textbf{z}))] - \frac{1}{2^M} \sum _{\mathbb{X}_s \in \mathcal{P}(\mathbb{X})} D_{KL}\biggl( f_{\psi} (\unimodalpost))\ ||\ p_{\theta}(\textbf{z})\biggr)
    \end{split}
\end{equation}

The KL-divergences between the transformed subset posteriors and the prior can be evaluated as follows using \cref{eq:changeofvariables_}:

\begin{equation}
    \begin{split}
        D_{KL}\biggl( f_{\psi} (\unimodalpost))\ ||\ p_{\theta}(\textbf{z})\biggr) &= \mathbb{E}_{f_{\psi}(q_{\phi_m})} \left[ \log  f_{\psi} (q_{\phi_m}(f_{\psi}(z)|x_m)) - \log p_{\theta}(f_{\psi}(z))   \right]\\
        &= \mathbb{E}_{q_{\phi_m}} \left[ \log  q_{\phi_m}(z|x_m) - \log \det J_{f_{\psi}} - \log p_{\theta}(f_{\psi}(z))   \right]\\
    \end{split}
\end{equation}

We use the \mg{MoFoPoE} method in comparison to the other methods that make use of the inverse transform $f^{-1}$ to evaluate if the merging of information between unimodal posteriors can be improved by simply making the subsets more flexible. % todo

One advantage of the \mg{MoFoPoE} method is that since the inverse of the flow transformation is not needed, implementations of normalizing flows can be used were the (% todo schreib das vielleicht lieber als NAchteil bei gfm models?)

%\begin{figure}[h!]
%    \centering
%    \resizebox{0.9\textwidth}{!}{%
%        \py{pytex_printonly(script='scripts_/mofop_graph.py', data = '')}
%    }
%    \caption{\textbf{Fowchart depicting the \mg{MofoPoE} method.} The \mg{MofoPoE} creates more expressive subset posteriors by transforming the PoE posteriors with a series of invertible transformations.}
%    \label{fig:mofopoe}
%\end{figure}

\subsubsection{MopgfM}\label{subsec:mopgfm}
% the main disadvantage here is that this method requires many parameters (needs to flows) but has not much flexibility
The \mg{Mopgfm} mixes the mean and the standard deviation of the unimodal posteriors, in order to obtain a normal distribution that depends on each of the uni modal posteriors (see \cref{eq:qjointmopgfm}).
This is a generalisation of the PoE method since a product of gaussian experts is itself Gaussian with mean $\mu_{PoE} = (\sum _i \mu _i V_i)(\sum _i V_i)^{-1}$ and covariance $V_{PoE}= (\sum _i V_i)^{-1}$ where $\mu _i, V_i$ are the parameters of the $i$-th Gaussian.
Without loss of generality, it can be assumed that $f_{\mu}^{-1}(\sum _{\xsetm \in \xsubset} \frac{f_{\mu}(\mu_s)}{|\xsubset|}) = \mu_{PoE}$ and $f_{\sigma}^{-1}(\sum  _{\xsetm \in \xsubset} \frac{f_{\sigma}(\sigma_s^2)}{|\xsubset|}) = V_{PoE}$.
The main advantage of this method is that since it is a generalisation of the PoE, it gives more flexibility to the modality fusion.
However, this comes at the cost that the expressiveness of the joint distribution is limited by being a Gaussian, and since the transformations are applied on the parameters of the uni modal distributions, transparency of the resulting transformation is lost.
It is hard, if not impossible, to translate \cref{eq:qjointmopgfm} into the following equation:
\begin{equation}
    q_{\phi, joint} = T(\{q_{\phi _m}(z|x_m) \forall x_m \in \xset\})
\end{equation}
with T a well defined transformation.

\begin{figure}[h!]
    \centering
    \resizebox{0.9\textwidth}{!}{%
        \py{pytex_printonly(script='scripts_/mopgfm_graph.py', data = '')}
    }
\end{figure}

\subsubsection{MogfM\_amortized}\label{subsubsec:mogfm_amortized}
For the \mg{MogfM\_amortized} method, we introduce a new loss $\mathcal{L}_2$ that pushes $f_{\psi}$ to map the uni modal posteriors to an amortized prior distribution, i.e. such that:

\begin{equation}
    \label{eq:amortizedprior}
    f_{\psi}(\unimodalpost) \sim \mathcal{N}(f_{\psi}(\mu_m), \textbf{I})
\end{equation}

Then, the density of the sum of random variables $\textbf{G}_f$ can easily be evaluated with:
\begin{equation}
    \textbf{G}_f(\textbf{z}|\textbf{x}_{1:|\xsubset|}) =\sum _{\xsetm \in \xsubset} \frac{f(\unimodalpost)}{|\xsubset|} \sim \mathcal{N} \left(  \sum _{m \in \xsubset} \frac{f(\mu_m)}{|\xsubset|}, \frac{1}{\sqrt{|\xsubset|}}  \cdot \textbf{I} \right)
\end{equation}

\Cref{eq:amortizedprior} can be achieved by minimizing the KL-divergence between the transformed uni modal posteriors and the amortized prior:
\begin{equation}
    \begin{split}
        \mathcal{L}_2 &= \sum _{\xsetm \in \xset} D_{KL}\left( f(\unimodalpost)\ ||\ \mathcal{N}(f(\mu_m), \textbf{I}) \right)\\
        &= \sum _{\xsetm \in \xset} D_{KL}\left( f(\unimodalpost)\ ||\ p_{\theta_m}(\textbf{z}) \right)\\
        &=  \sum _{\xsetm \in \xset} \mathbb{E}_{f(\unimodalpost)} [\log f(\unimodalpost) - \log p_{\theta_m}(\textbf{z})]\\
        &=  \sum _{\xsetm \in \xset} \mathbb{E}_{z_m \sim \unimodalpost} [\log q_{\phi_m}(z_m|\textbf{x}_M) - \log \det J_f  - \log p_{\theta_m}(f(z_m))]\\
    \end{split}
\end{equation}

The ELBO can then be evaluated as following:
\begin{equation}
    \begin{split}
        \mathcal{L}_1 &=  \mathbb{E}_{q_{\phi}(\textbf{z}|\mathbb{X})}[\log (p_{\theta}(\mathbb{X}|\textbf{z}))] -  \frac{1}{2^M} \sum _{\mathbb{X}_s \in \mathcal{P}(\mathbb{X})} D_{KL}\biggl( \tilde{q}_{\phi}(\textbf{z}|\mathbb{X}_s)\ ||\ p_{\theta}(\textbf{z})\biggr)\\
        &= \mathbb{E}_{q_{\phi}(\textbf{z}|\mathbb{X})}[\log (p_{\theta}(\mathbb{X}|\textbf{z}))] - \frac{1}{2^M} \sum _{\mathbb{X}_s \in \mathcal{P}(\mathbb{X})} \mathbb{E}_{\tilde{q}_{\phi}(\textbf{z}|\mathbb{X}_s)}[\log \tilde{q}_{\phi}(\textbf{z}|\mathbb{X}_s) - \log p_{\theta}(\textbf{z}) ]\\
        &= \mathbb{E}_{q_{\phi}(\textbf{z}|\mathbb{X})}[\log (p_{\theta}(\mathbb{X}|\textbf{z}))] - \frac{1}{2^M} \sum _{\mathbb{X}_s \in \mathcal{P}(\mathbb{X})} \mathbb{E}_{\textbf{G}_f(\textbf{z}|\textbf{x}_{1:|\xsubset|})}[\log \textbf{G}_f(\textbf{z}|\textbf{x}_{1:|\xsubset|}) + \log \det J_{f^{-1}}- \log p_{\theta}(\textbf{z}) ]
    \end{split}
\end{equation}

The total loss is then:
\begin{equation}
    \begin{split}
        &\mathcal{L} = \mathcal{L}_2 + \mathcal{L}_2\\
        &= \mathbb{E}_{q_{\phi}(\textbf{z}|\mathbb{X})}[\log (p_{\theta}(\mathbb{X}|\textbf{z}))] - \frac{1}{2^M} \sum _{\mathbb{X}_s \in \mathcal{P}(\mathbb{X})} \mathbb{E}_{\textbf{G}_f(\textbf{z}|\textbf{x}_{1:|\xsubset|})}[\log \textbf{G}_f(\textbf{z}|\textbf{x}_{1:|\xsubset|}) + \log \det J_{f^{-1}}- \log p_{\theta}(\textbf{z}) ]\\
        &+ \sum _{\xsetm \in \xset} \mathbb{E}_{z_m \sim \unimodalpost} [\log q_{\phi_m}(z_m|\textbf{x}_M) - \log \det J_f  - \log p_{\theta_m}(f(z_m))]
    \end{split}
\end{equation}

The resulting joint posterior $\textbf{G}_f(\textbf{z}|\textbf{x}_{1:|\xsubset|}) =\sum _{\xsetm \in \xsubset} \frac{f(\unimodalpost)}{|\xsubset|}$ of the \mg{MogfM\_amortized} method can follow any distribution can thus be more expressive than the joint posterior in the \mg{MopgfM} or \mg{MoPoE} methods.
The main disadvantage of this method is that the KL-divergence term in $\mathcal{L}_1$ can only be evaluated when the flow $f$ has already learned to map the uni modal posteriors towards the amortized priors.
In parctice, this makes it very hard to tune the two loss functions to each other.

\subsubsection{iwMogfM}
Like the \mg{MoPoE}, the iwMogfM creates the joint posterior by creating $2^M$ subsets from the uni modal posteriors and then mixing them with a mixture of experts.
However, instead of using a PoE to create the subsets, it uses an $f$-mean.
The iwMogfM also makes use of the importance sampling method from the iwVAE, by sampling K samples from the posterior.
To derive the resulting objective, we rewrite the objective from the \mg{MoPoE} for K importance samples in a first step:
\begin{equation}
    \begin{split}
        \mathcal{L}^{mopoe}_1 &= \mathbb{E}_{\jointpost} \left[ \log \frac{p_{\theta}(\xset, \textbf{z})}{\jointpost} \right]\\
        &=\frac{1}{|\powerset|} \sum _{\xsubset \in \powerset} \mathbb{E}_{\subsetpost} \left[ \log \frac{p_{\theta}(\xsubset, \textbf{z})}{\subsetpost} \right]\\
        &=\frac{1}{|\powerset|} \sum _{\xsubset \in \powerset} \mathbb{E}_{z_s \sim \subsetpost} \left[ \log \frac{p_{\theta}(\xsubset, \textbf{z}_s)}{\tilde{q}_{\phi}(\textbf{z}_s|\xsubset)} \right]\\
        &\leq \frac{1}{|\powerset|} \sum _{\xsubset \in \powerset} \mathbb{E}_{z^{1:K}_s \sim \subsetpost} \left[ \log \frac{1}{K} \sum _{k=1}^K \frac{p_{\theta}(\xsubset, \textbf{z}^k _s)}{\tilde{q}_{\phi}(\textbf{z}^k _s|\xsubset)} \right] = \mathcal{L}^{mopoe}_K
    \end{split}
\end{equation}

Using the fact that the logarithm is concave and Jensens inequality, $\mathcal{L}^{mopoe}_K$ can be rewritten as follows:
\begin{equation}
    \begin{split}
        &\frac{1}{|\powerset|} \sum _{\xsubset \in \powerset} \mathbb{E}_{z^{1:K}_s \sim \subsetpost} \left[ \log \frac{1}{K} \sum _{k=1}^K \frac{p_{\theta}(\xsubset, \textbf{z}^k _s)}{\tilde{q}_{\phi}(\textbf{z}^k _s|\xsubset)} \right]\\
%
        &\geq \frac{1}{|\powerset|} \sum _{\xsubset \in \powerset} \mathbb{E}_{z^{1:K}_s \sim \subsetpost} \left[ \frac{1}{K} \sum _{k=1}^K \log  \frac{p_{\theta}(\xsubset, \textbf{z}^k _s)}{\tilde{q}_{\phi}(\textbf{z}^k _s|\xsubset)} \right]\\
%
        &= \frac{1}{|\powerset|} \sum _{\xsubset \in \powerset} \mathbb{E}_{z^{1:K}_s \sim \subsetpost} \left[ \frac{1}{K} \sum _{k=1}^K \log p_\theta (\xsubset|\textbf{z}^k _s) -\log  \frac{\tilde{q}_{\phi}(\textbf{z}^k _s|\xsubset)}{p_{\theta}(\textbf{z}^k _s)} \right]\\
%
        &= \frac{1}{|\powerset|} \sum _{\xsubset \in \powerset} \mathcal{R}^{1:K}_s - D^{1:K}_s
    \end{split}
\end{equation}

where $\mathcal{R}$ is the reconstruction loss and D the Kl-divergence between the subset posterior and the prior.
The subset posteriors are obtained with an $f$-mean of the uni modal posteriors:
\begin{equation}
    \subsetpost = f^{-1}\left(\sum _{\xsetm \in \xsubset} \frac{f(\unimodalpost)}{|\xsubset|}\right)
\end{equation}

Since the density of the subset posteriors is hard to evaluate, $D^{1:K}_s$ is calculated by comparing K samples from $\subsetpost$ with K samples from the prior $\prior$:
\begin{equation}
    \begin{split}
        D^{1:K}_s &=  \mathbb{E}_{z^{1:K}_s \sim \subsetpost} \left[ \frac{1}{K} \sum _{k=1}^K \log \tilde{q}_{\phi}(\textbf{z}^k _s|\xsubset) - \log p_{\theta}(\textbf{z}^k _s) \right]\\
        %
        &=     \mathbb{E}_{\{z^{1:K}_m \sim \unimodalpost \forall \samplem \in \xsubset\}} \left[ \frac{1}{K} \sum _{k=1}^K \log f^{-1}\left(\sum _{\samplem \in \xsubset} \frac{f(q_{\phi_m}(\textbf{z}_m^k|\samplem))}{|\xsubset|}\right) - \log p_{\theta}(\textbf{z}^k _s) \right] \label{eq:samplestep}\\
        %
%        &\approx \sum ^{\text{batch size}} \frac{1}{K} \sum _{k=1}^K f^{-1}\left(\sum _{\xsetm \in \xsubset} \frac{f(q_{\phi_m}(\textbf{z}_m^k|\xsetm))}{|\xsubset|}\right)(  \log f^{-1}\left(\sum _{\xsetm \in \xsubset} \frac{f(q_{\phi_m}(\textbf{z}_m^k|\xsetm))}{|\xsubset|}\right) - \log p_{\theta}(\textbf{z}^k _s))\\
        %
        &\approx \sum ^{\text{batch size}} \frac{1}{K} \sum _{k=1}^K f^{-1}\left(\sum _{\xsetm \in \xsubset} \frac{f(\mu_m+\sigma_m\ast \epsilon)}{|\xsubset|}\right)(  \log f^{-1}\left(\sum _{\xsetm \in \xsubset} \frac{f(\mu_m+\sigma_m\ast \epsilon)}{|\xsubset|}\right) - \log \epsilon), \epsilon \sim \mathcal{N}(0,\textbf{I})
    \end{split}
\end{equation}

%In \cref{eq:samplestep}, the sampling from the gaussian uni modal posteriors is done with the reparameterisation trick \citep{rezende_stochastic_2014}:
%\begin{equation}
%
%\end{equation}


%\begin{figure}[h!]
%    \centering
%    \resizebox{0.99\textwidth}{!}{%
%        \py{pytex_printonly(script='scripts_/mogfm_graph.py', data = '')}
%    }
%    \caption{\textbf{The iwMogfM makes use of the $f$-mean to create $2^M$ subsets, which are then merged with a MoE.} Here $M=2$, the empty subset is not shown. On the left side are the two input modalities from the polymnist dataset (see \cref{polymnist}), on the right side are the generated samples. In the header of each generated sample is described from which subset the decoder sampled for the generation (left side of the $\rightarrow$) and which modality was generated (right side of the $\rightarrow$).}
%    \label{iwmogfmGraph}
%\end{figure}




    \chapter{Experiments}
In this section we describe the experimental setup that was used in order to compare our methods to each other as well as to the \mg{MVAE}, the \mg{MMVAE} and the \mg{MoPoE} methods.


\section{Datasets}
We evaluate on three datasets, each providing different difficulties in order to filter out advantages and disadvantages of our methods.

\subsection{PolyMNIST} \label{polymnist}
The PolyMNIST dataset, first introduced in \citep{sutter_generalized_2020}, consists of MNIST digits overlayed over a random part of a certain background image.
The modality specific information of each sample in this dataset is defined by the background image and the shared information by the digit.
In this case the modality specific information is harder to learn than the shared information (for the modality specific information the model has to have learned the set of possible backgrounds and styles of handwriting while the shared information is simply the set of digits).
Examples from the PolyMNIST dataset are shown in \cref{fig:PolyMNIST}.
In total there are 60,000 tuples of training examples and 10,000 tuples of test examples.
The PolyMNIST dataset is useful to study how the number of modalities impacts the performance of multi modal methods, since an abritrary amount of modalities can easily be generated.

\begin{figure}[h!]
    \centering
    \includegraphics[width=0.9\textwidth]{data/thesis/polymnist_example}
    \caption{The PolyMNIST dataset consists of sets of MNIST
    digits where each set consists of M images
    with the same digit label but different backgrounds
    and different styles of hand writing for M different modalities.}
    \label{fig:PolyMNIST}
\end{figure}

\subsection{MIMIC-CXR Database}
The MIMIC-CXR Database \citep{johnson2019mimic} is a large publicly available dataset of chest radiographs with free-text radiology reports containing 377,110 images corresponding to 227,835 radiographic studies performed at the Beth Israel Deaconess Medical Center in Boston, MA.
In this work, three modalities were extracted from the database: frontal and lateral chest radiographs together with their corresponding text reports (\cref{fig:mimic}).
Only datapoints where all three modalities are present were selected.
Every sample is labeled with one or more of the following categories: 'Atelectasis', 'Cardiomegaly', 'Consolidation', 'Edema', 'Enlarged Cardiomediastinum', 'Fracture', 'Lung Lesion', 'Lung Opacity', 'Pleural Effusion', 'Pleural Other', 'Pneumonia', 'Pneumothorax', 'Support Devices'.
For our purposes, all images were resized to (128, 128).

\paragraph{Text preprocessing}
Every word that occurs at least 3 times in all the text reports is mapped to an index.
Using this mapping each sentence is encoded into a sequence of indices.
All sentences with a word count above 128 are truncated and all sentences consisting of less words are padded with a padding token "$<pad>$" such that all text samples are of equal length (128 words).

The MIMIC-CXR dataset is extremely challenging since both the modality specific and shared information present small details that are hard to learn.
In particular, the pathologies represent only a small fraction of the images such that they are hard to distinguish, even for human experts.
Also, the shared information between modalities is different between the image modalities and the image and text modalities together.
The shared information between images contains information about the patient such as the posture and size, that is not contained in the text modality.
The MIMIC-CXR dataset provides a good representation of real world data with all the challenges that come with it, such as unevenly represented classes and different shared information between modalities.


\begin{figure}[h!]
    \centering
    \includegraphics[width=0.8\textwidth]{data/static/mimic_dataset_sample}
    \vspace{-1cm}
    \caption{An example from the MIMIC-CXR dataset is shown: the frontal view image together with the corresponding lateral view image and the text report.}
    \label{fig:mimic}
\end{figure}

%%%%%%%%%%%%%%%%%%%%%%%%%%%%%%%%%%%%%%%%%%%%%%%%%%%%%%%%%%%%%%%%%%%%%%%%%%%%%%%%%%%%%%%%%%%%%%%%%%%%%%%%%%%%%%%%%%%%%%%%%%%%%%%%%%%%%%%%%%%%%%%%%%%%%%%%%%%%%%%%%%%%%%%%%%%%%%%%%%%%


\section{Metrics}
In order to compare the proposed methods in a meaningful manner, we make use of three metrics that each quantifies the performance of a different aspect of mmVAEs.
Namely, we compare the quality of the learned latent representation, the coherence of the generated samples and the quality of the generated samples, as described in the follwing sections.

\subsection{Evaluation of the Latent Representation} \label{subsec:lr metric}
To evaluate if the different mmVAEs are able to extract characteristic information and compress it in the latent representation in a meaningful manner, we evaluate the separability of the latent space via linear classifiers.
If the classifier can separate the latent space into the corresponding classes, we conclude that the posterior approximations are meaningful.
One classifier for each class and for each latent space is trained on 1000 encoded samples from the training set and tested on the test set.
Note that this can be seen as a variant of the disentanglement metric from \citep{beta_vae} where each class is a different generative factor.
If the dimensions of latent representation are independent and interpretable, there will be less variance in the samples belonging to the same class and thus make them separable from the rest with low capacity classifiers.
It has been shown in \citep{locatello_challenging_2019} that this disentanglement metric correlates with other disentanglement metrics.

\subsection{Evaluation of the Generation Coherence}
\label{subsubsec:gen_coh}
To evaluate if the method is able to separate the shared information from the modality specific information, we verify that all generated tuples belong to the same class using pretrained classifiers.
For conditional generation, the conditionally generated samples have to be coherent to the input samples.
The coherence accuracy is the ratio of coherent samples divided by the number of generated samples.
For every data type, we train a neural network classifier in a supervised way and the architecture is identical to the encoder except from the last layer.

When comparing the coherence accuracy for methods trained on only one modality, the coherence is evaluated by verifying if the generated sample belongs to the same class than the input sample.
We compare the coherence accuracy for the generation of missing modalities, reconstruction of modalities and randomly generated samples.

\subsection{Evaluation of the Generation Quality}
\label{subsubsec:gen_qual}
To evaluate the quality of the generated samples, we make use of the precision-recall score from \citep{precision_recall_distributions}.
The Precision and Recall for Disitributions (prd) metric is similar to the Fréchet Inception Distance (FID) \citep{heusel_gans_2017}, but disentangles the quality of generated samples from the coverage of the target distribution.
The prd metric reduces the problem of comparing a distribution Q (the distribution of generated samples) to a reference distribution P (the distribution of true images) into a one dimensional problem by applying a pre-trained classifier trained on natural images and to compare \^{P} and \^{Q} at a feature level.
The embeddings are then clustered such that the histogram over the cluster assignments can be meaningfully compared.
Here we compute the prd score by taking the area under the precision-recall curve.

%%%%%%%%%%%%%%%%%%%%%%%%%%%%%%%%%%%%%%%%%%%%%%%%%%%%%%%%%%%%%%%%%%%%%%%%%%%%%%%%%%%%%%%%%%%%%%%%%%%%%%%%%%%%%%%%%%%%%%%%%%%%%%%%%%%%%%%%%%%%%%%%%%%%%%%%%%%%%%%%%%%%%%%%%%%%%%%%%%%%


\section{Comparison across different number of importance samples}
As introduced in \cref{subsec:iwvae}, the tightness of the ELBO in \cref{eq:vaeelbo} can be improved by sampling multiple importance samples from the posterior at each step \parencite{burda_importance_2016}.
To test if the advantage of our more flexible aggregation over modalities using the generalized $f$-mean can be overcome by taking more importance samples, we compare the \mg{mopoe} and the \mg{mopgfm} methods using the importance weighted training paradigm from \parencite{burda_importance_2016}, with different number of importance samples.
The results are shown in \cref{subsec:iw_comp}.

%%%%%%%%%%%%%%%%%%%%%%%%%%%%%%%%%%%%%%%%%%%%%%%%%%%%%%%%%%%%%%%%%%%%%%%%%%%%%%%%%%%%%%%%%%%%%%%%%%%%%%%%%%%%%%%%%%%%%%%%%%%%%%%%%%%%%%%%%%%%%%%%%%%%%%%%%%%%%%%%%%%%%%%%%%%%%%%%%%%%


\section{Hyperparameter Selection}
\label{sec:Hyperparameter Selection}
We select three hyperparameters for the standard mmVAE models (\mg{MoPoE}, \mg{MoE}, \mg{PoE}) that we optimize for our experiments:

\begin{itemize}
    \item The dimension of the latent representation (the bottleneck of the VAE).
    A higher dimensional latent representation gives the model more freedom to separate the different classes and can contain more information in general.
    However, for a too large latent representation, the encoder is not constrained to extract only the most informative features of the input such that the latent representation will contain much information that is non-informative for the decoder.
    \item The learning rate for the stochastic optimization of the parameters, using the Adam optimizer \citep{kingma_adam_2017}.
    For a low learning rate, the objective will take a very long time to converge and for a too high learning rate it might oscillate around a local minimum and never converge.
    \item The $\beta$ in the modified ELBO from \cref{eq:vaeelbo}, described in \cref{subsec:vae}
\end{itemize}

Since the choice for these parameters is non trivial, we optimize them using the hyperparameter optimization framework \dg{Optuna} \citep{akiba_optuna_2019}.
As objective, we use a weighted average of the generation coherence metric (\cref{subsubsec:gen_coh}) and the area Under the precision-recall curve (prd-score, \cref{subsubsec:gen_qual}, where a higher weight is given to the prd-score since its values are generally lower than those of the generation coherence metric.
The results for the \mg{MoPoE} method can be seen in \cref{fig:mopoe hyperopt}.

For our methods that make use of normalizing flows, we add three additional hyperparameters:
\begin{itemize}
    \item The number of chained transformations with which the normalizing flow is constructed (Nbr Flows).
    \item The number of coupling block layers per transformation (Nbr Coupling Block layers).
    \item The number of parameters of each coupling block layer Coupling Dim).
\end{itemize}

For the optimization of those, we fixed the dimension of the latent representation and the learning rate according to what gave the best results for the \mg{MoPoE} method.
Namely, a latent representation of dimension 1280 and a learning rate of $5e-4$.
The results can be seen in \cref{fig:mopgfm hyperopt1} and \cref{fig:mopgfm hyperopt2}.


\section{General Setup}

\subsection{PolyMNIST}
We present our results for the PolyMNIST dataset in \cref{subsec: results polymnist}, for which we trained each method 2 times for 500 epochs.
All results are presented as averages over the 2 runs, accompanied with the standard deviations.
If the number of modalities is not explicitely specified, the model was trained with three modalities from the PolyMNIST dataset.
All parameter values for the experiments on the PolyMNIST dataset studied in \cref{subsec: results polymnist} can be found in \cref{tab:params_poly}.
To reduce training time we chose to use a small number of chained transformations (Nbr Flows) for all normalizing flow methods.
A lower number of flows also yielded more stable results.
We adapted the parameters of the \mg{PoE} and the \mg{MoE} method to match those selected for the \mg{MoPoE} using the hyperoptimization.
Only the dimension of the latent representation (class dim) of the \mg{PoE} was reduced since the performance of the \mg{PoE} dropped significantly with a higher dimension.
All methods are trained with one importance sample from the joint posterior if not specified otherwise, except for the \mg{iwmogfm}, which is trained with two importance samples (\cref{subsubsec:iwMogfM}).
We use the same network architecture that was used in \citep{sutter_generalized_2020}, a simple 3 layer convolutional network as encoder and decoder.

All parameters for the \mg{iwmogfm} and \mg{mogfm\_amortized} methods have been chosen without any hyperoptimization.
For both methods, we found it difficult to find the optimal $\beta$, but found that both are able to learn meaningful representations and yield good generative results, without the KL-divergence as regularisation.
We set $\beta$ to 0 for the \mg{mogfm\_amortized} and to 0.001 for \mg{iwmogfm}.
A very low $\beta$ yielded better results for the \mg{iwmogfm} method than $\beta = 0$.



\py{
    pytex_tab(
    script='thesis/scripts/params_tab_polymnist.py',
    options_pre='\\centering \\resizebox{\\textwidth}{!}{',
    options_post='}',
    caption='Parameters used for the models evaluated on the PolyMNIST dataset.',
    label='params_poly'
    )
}

\subsection{MIMIC-CXR}
We present our results for the MIMIC-CXR dataset in \cref{sec:mimic res}, for which each method was trained once for 150 epochs.
All parameter for the methods evaluated on the MIMIC-CXR dataset were selected to match those used in \citep{klug_multimodal_nodate}.
We chose to weight every modality equally in the reconstruction loss.
A table with all parameters for every method evaluated on the MIMIC-CXR dataset is shown in \cref{tab:params_mimic}.
We use the same ResNet \citep{he2016deep} type architecture for the encoder and decoder, with 5 residual layers for the image modalities and 6 residual layers for the text modality.
We refer to the published codebase for more details on the implementation of the models \citep{mmvae_github}.

\py{
    pytex_tab(
    script='thesis/scripts/params_tab_mimic.py',
    options_pre='\\centering \\resizebox{\\textwidth}{!}{',
    options_post='}',
    caption='Parameters used for the models evaluated on the MIMIC-CXR dataset.',
    label='params_mimic'
    )
}

%todo


\section{Reproducibility}
Advances in scientific research are contingent on reproducibility and verifiability of previous work.
To this end, we make the framework used to train all models evaluated in this work available as an open source python package \citep{mmvae_github}, tested with continuous integration using \citep{travis} and kept up to date with \citep{dependabot}.
We publish this thesis as a reproducible self publishing document \citep[\href{https://github.com/TheChymera/RepSeP}{RepSeP}]{repsep} made available on GitHub \citep{mmnf_repsep}.
All data used to produce this document, including the trained models are made available on Zenodo \citep{mmnfdataset}.
Using \LaTeX\ and PythonTeX \citep{pytex}, we make all steps described herein easily reexecutable and extendable.
It is thus easy to reproduce all figures using different parameters for each method for future work.

    \chapter{Results}
\label{chap:results}
% First compare mopoe, mopgfm, mogfm
% Then compare iwmopoe, iwmogfm, mogfm

% try other datasets also (mimic,..)
% Train 5 times: mopoe, mopgfm, mogfm

% also compare training times


\section{Hyperoptimization Results}

The results for the optimization of the hyperparameters described in \cref{sec:Hyperparameter Selection} can be seen in \cref{fig:mopoe hyperopt} for the \mg{MoPoE} method and \cref{fig:mopgfm hyperopt} for the \mg{Mopgfm} method.
Note that every figure in \cref{fig:mopoe hyperopt} and \cref{fig:mopgfm hyperopt} represents results in function of a parameter, however all other parameters are not fixed and might vary for every point.

\paragraph{MoPoE Results} Descriptively, we find that the \mg{MoPoE} performs best on the PolyMNIST dataset with a learning rate $\approx 5e-4$ and a latent dimension of 1280.
The performance of the \mg{MoPoE} seems to be robust to a change of $\beta$ in the range of 1.1 to 2.1.

\paragraph{MopgfM Results} The optimal number of coupling layers appears to be 8 with the best number of dimensions being 64.
\Cref{subfig:mopgfm_nbr_flows} shows that better scores are achieved with a higher number of chained transformations, however more flow transformations also lead to more variance in the resulting score.
In practice, we have also experienced that models with a high number of normalizing flows can provide better performance but are more unstable.
The \mg{Mopgfm} seems to perform best with a $\beta$ between 1.5 and 2.4.

Overall the hyperoptimization results show that while the \mg{MoPoE} presents results that are much more stable (from \cref{subfig:mopoe_lr_rate}, one can infer that the only true variance in the objective value is due to a high learning rate), the highest achieved scores are lover than those achieved by the \mg{Mopgfm} method.


\begin{figure}
    \centering
    \begin{subfigure}[b]{0.49\textwidth}
        \centering
        \includegraphics[width=\textwidth]{data/static/mopoe_lr_rate}
        \caption{Results shown in function of the learning rate}
        \label{subfig:mopoe_lr_rate}
    \end{subfigure}
    \hfill
    \begin{subfigure}[b]{0.49\textwidth}
        \centering
        \includegraphics[width=\textwidth]{data/static/mopoe_class_dim}
        \caption{Results shown in function of the dimension of the latent representation}
    \end{subfigure}
    \hfill
    \begin{subfigure}[b]{0.5\textwidth}
        \centering
        \includegraphics[width=\textwidth]{data/static/mopoe_beta}
        \caption{Results shown in function of $\beta$}
    \end{subfigure}
    \caption{Hyperoptimization run results for the \mg{MoPoE} method. Every subfigure presents results in function of one parameter, with all other parameters varying.}
    \label{fig:mopoe hyperopt}
\end{figure}

\begin{figure}
    \centering
    \begin{subfigure}[b]{0.49\textwidth}
        \centering
        \includegraphics[width=\textwidth]{data/static/mopgfm_nbr_cup_layers}
        \caption{Results shown in function of the number of coupling layers in each flow}
    \end{subfigure}
    \hfill
    \begin{subfigure}[b]{0.49\textwidth}
        \centering
        \includegraphics[width=\textwidth]{data/static/mopgfm_coupling_dim}
        \caption{Results shown in function of the couling layer dimension}
    \end{subfigure}
    \hfill
    \begin{subfigure}[b]{0.49\textwidth}
        \centering
        \includegraphics[width=\textwidth]{data/static/mopgfm_nbr_flows}
        \caption{Results shown in function of the number of flows}
        \label{subfig:mopgfm_nbr_flows}
    \end{subfigure}
    \hfill
    \begin{subfigure}[b]{0.49\textwidth}
        \centering
        \includegraphics[width=\textwidth]{data/static/mopgfm_beta}
        \caption{Results shown in function of $\beta$}
    \end{subfigure}
    \caption{Hyperoptimization run results for the \mg{Mopgfm} method. Every subfigure presents results in function of one parameter, with all other parameters varying.}
    \label{fig:mopgfm hyperopt}
\end{figure}


\section{PolyMNIST} \label{subsec: results polymnist}

\subsection{Evaluation of the Latent Representation}

\paragraph{Evaluation over epochs}
Evaluating the separability of the latent representation (\cref{subsec:lr metric}) for models trained on 3 modalities, we find that the \mg{mofop}, the \mg{mopgfm} and the \mg{mopoe} perform similarly, yielding on average a linear classification accuracy of \py{boilerplate.get_lr_score(method='mofop')}, \py{boilerplate.get_lr_score(method='mopgfm')} and \py{boilerplate.get_lr_score(method='mopoe')} respectively for all subsets after 400 training epochs (see \cref{fig:ep comp lr}).
The two methods that do not regularize the latent representation with the KL-divergence (\mg{iwmogfm}, \mg{mogfm\_amortized}) perform worse than those that do, except for the \mg{moe} and \mg{poe} methods.
The two latter methods have the worst performance overall.

\begin{sansmath}
    \py{pytex_fig('thesis/scripts/plots/epoch_comparison_lr.py',
        conf='thesis/main.conf',
        label='ep comp lr',
        caption='
        \\textbf{Linear classification accuracy for different epochs over the test set, averaged over all subsets.}
        All methods were trained with 3 modalities.
        ',
        )}
\end{sansmath}

\paragraph{Evaluation across subset posteriors}
\Cref{tab:lr eval} compares the classification accuracies of linear classifiers trained on each subset posterior.
Overall, we see that the classification accuracy improves when more modalities make up the latent representation which shows that all methods are able to aggregate the modalities.
In particular, we find that the \mg{iwmogfm} method has the best performance when all modalities are given.
Comparatively, the \mg{mopgfm} is able to optimize the uni modal posteriors better than the \mg{mopoe} and the \mg{mofop}, yielding an average accuracy of \py{boilerplate.get_unimodal_lr_score(method = 'mopgfm')} compared to \py{boilerplate.get_unimodal_lr_score(method = 'mopoe')} and \py{boilerplate.get_unimodal_lr_score(method = 'mofop')}.
Our results show that the $m0$ modality is the hardest modality to learn from and as expected the \mg{poe} struggles the most to optimize for it.
It has the lowest accuracy on the subset containing only the $m0$ modality but compensates with the other modalities in the multi modal subsets.
Similarly, both the \mg{iwmogfm} and \mg{mogfm\_amortized} yield their lowest score on the $m0$ subset, while their performance improves significantly on the multi modal subsets.

\py{
    pytex_tab(
    script='thesis/scripts/lr_eval_tab.py',
    options_pre='\\centering \\resizebox{0.99\\textwidth}{!}{',
    options_post='}',
    caption='Linear classification accuracy of all subset posteriors for the test set.\\
    ',
    label='lr eval',
    )
}

\paragraph{Scalability with the number of modalities}
\Cref{fig:nbr mods comp lr} shows a comparison of how well each method scales with number of modalities it is trained on, using the linear classification metric (\cref{subsec:lr metric}).
Again, we see that the \mg{mofop}, the \mg{mopoe} and the \mg{mopgfm} scale equally well with the number of modalities, the latter yielding a slightly better score for 1 modality.

\begin{sansmath}
    \py{pytex_fig('thesis/scripts/plots/nbr_mods_comparison_lr.py',
        conf='thesis/main.conf',
        label='nbr mods comp lr',
        caption='
        \\textbf{Linear classification accuracy for models trained with different number of modalities, averaged over all subsets.}
        All methods were trained for 500 epochs.
        ',
        )}
\end{sansmath}

\subsection{Evaluation of the Generation Coherence}

\paragraph{Evaluation over epochs}
Evaluating the generation coherence (\cref{subsubsec:gen_coh}), we find that the \mg{mogfm\_amortized} and the \mg{iwmogfm} perform the best overall, yielding an accuracy of $\approx 0.88$ after only 100 epochs (\cref{fig:ep comp gen}).
However, the performance of both methods does not improve after 100 epochs such that after 500 epochs, it almost matches that of the \mg{mopgfm} and \mg{mofop}.
Overall, all methods making use of normalizing flow yield higher scores than the baseline methods.

\begin{sansmath}
    \py{pytex_fig('thesis/scripts/plots/epoch_comparison_gen.py',
        conf='thesis/main.conf',
        label='ep comp gen',
        caption='
        \\textbf{Generation classification accuracy for different epochs over the test set, averaged over all combinations of input modalites and all output modalities.}
        All methods were trained with 3 modalities.
        ',
        )}
\end{sansmath}

\paragraph{Comparison across missing modalities, reconstruction and random generation}
For the generation coherence accuracy of missing modalities the \mg{mopgfm} performs the best, followed by the \mg{mogfm\_amortized} and \mg{iwmogfm} methods.
For the reconstruction of modalities, both the \mg{mogfm\_amortized} and \mg{iwmogfm} methods perform the best, followed by the \mg{mopgfm} and \mg{mofop} methods.
For the generation of random samples, the \mg{moe} provides a much higher coherence accuracy score than all other methods, implying that the \mg{moe} learns a joint posterior that corresponds better to the prior than the other methods.
Since the \mg{iwmogfm} and \mg{mogfm\_amortized} were not trained with a regularization term in the objective that pushes their joint posterior to match the prior, their decoder networks do not recognize samples from the latter which explains the low accuracy for randomly generated images.
Interestingly, the coherence accuracy of randomly generated samples with the \mg{mofop} method is very low, suggesting that a higher regularization parameter $\beta$ might be needed.
\py{
    pytex_tab(
    script='thesis/scripts/gen_eval_tab.py',
    options_pre='\\centering \\resizebox{0.7\\textwidth}{!}{',
    options_post='}',
    caption='Generation coherence for the Test set.
    ',
    )
}

\paragraph{Scalability with the number of modalities}
Overall, the \mg{mopgfm}, the \mg{mopoe} and the \mg{moe} methods scale equally well with the number of modalities, the \mg{mopgfm} yielding better performance than the \mg{mopoe}, which itself performs better than the \mg{moe} (\cref{fig:nbr mods comp gen}).
For models trained on one modality, the coherence score is evaluated as self coherence only (\cref{subsubsec:gen_coh}), which is an easier task than coherence across generated samples.
This explains the slight dip in performance for all methods trained with 2 modalities.


\begin{sansmath}
    \py{pytex_fig('thesis/scripts/plots/nbr_mods_comparison_gen.py',
        conf='thesis/main.conf',
        label='nbr mods comp gen',
        caption='
        \\textbf{Generation classification accuracy for models trained with different number of modalities.}
        The average over all classification accuracies is taken, across all possible combinations of input modalities and all output modalities, for three modalities from the PolyMNIST dataset. All models were trained for 500 epochs.
        ',
        )}
\end{sansmath}

\subsection{Evaluation of the Ggeneration Quality}
Evaluating the generation quality (\cref{subsubsec:gen_qual}), we find that overall the methods making use of the generalized $f$-mean perform the best.
The \mg{mopgfm} yields the best prd score for the generation of missing modalities, while both the \mg{mogfm\_amortized} and \mg{iwmogfm} perform best on the reconstruction of modalities.
Interestingly, the \mg{mopoe} method provides prd scores with a higher variance than the other methods.
The \mg{poe} yields the best prd score for randomly generated samples followed by the \mg{mopoe}, however a qualitative evaluation of randomly generated samples in \cref{fig:rand gen polymnist} shows that while the \mg{mopoe} generates the digits well, there is not much variance in the backgrounds.
The modalitiy specific information (the background) of the randomly generated images from the \mg{mopoe} actually seem to only correspond to an average of all pixels of the background image correspond to each modality.
The same can be seen for the randomly generated images of the \mg{mopgfm} and the \mg{moe}.
The background has a much higher variance for the \mg{poe} method and a slightly higher variance for the \mg{mofop} method.
Examples of generated samples for each method can be found in \cref{chap:gen ex polymnist}.
\py{
    pytex_tab(
    script='thesis/scripts/prd_tab.py',
    options_pre='\\centering \\resizebox{0.7\\textwidth}{!}{',
    options_post='}',
    caption='Area under the Precision and Recall curve of the PRD score \citep{precision_recall_distributions}.\\
    ',
    )
}

%%%%%%%%%%%%%%%%%%%%%%%%%%%%%%%%%%%%%%%%%%%%%%%%%%%%%%%%%%%%%%%%%%%%%%%%%%%%%%%%%%%%%%%%%%%%%%%%%%%%%%%%%%%%%%%%%%%%%%%%%%%%%%%%%%%%%%%%%%%%%%%%%%%%%%%%%%%%%%%%%%%%%%%%


\section{Mimic-CXR}
%todo
% lzeig dass generieren von missing modalities besonders schlecht ist
% zeig dass die Modelle auch bei reconstruction details nicht generieren können (nicht genug kapazität):
% vergeleiche qualitativ examamples, vlt such ein beispiel mit offensichtlicher krankheit
    \chapter{Conclusion \& Discussion}
We have implemented and tested new methods that provide a more flexible way to aggregate over multiple modalities in multi modal VAEs, using the generalized $f$-mean.
Evaluating three metrics on the PolyMNIST dataset has shown that these methods improve results, especially for the coherence and the quality of the generated samples.
This indicates that the generalized $f$-mean is able to better merge the information from each modality into a joint distribution than previous, fixed aggregation functions.
However, a study of how well the methods scale with the number of modalities has shown that the methods utilizing normalizing flows scale less than those that do not.
We hypothesise that this comes from the fact that each modalitiy is transformed with the same normalizing flow, such that with more modalities, the task of the flow to learn a meaningful mapping for each modality becomes increasingly difficult.
We argue that this can be compensated with a higher amount of chained transformations, but which comes at a higher computational cost.

\paragraph{MofoP \& MopgfM}
As introduced in \cref{subsec:mofopoe}, the \mg{mofop} builds on the \mg{mopoe} by transforming each subset posterior approximation with a normalizing flow.
While providing a more flexible joint posterior approximation, this does not make the aggregation over modalities more flexible, since the subset distributions are obtained with PoEs and the joint distribution with a MoE over subsets.
We implemented and tested this method in comparison to our methods that utilize the generalized $f$-mean, to evaluate if the improved performance of those is due to a more flexible joint posterior distribution or a more flexible aggregation over modalities.
The \mg{mopgfm} provides a good comparison for this matter, since it utilizes the generalized $f$-mean, but uses a normal distributed posterior approximation.
It thus has a flexible aggregation over modalities but does not have a more flexible joint posterior distribution.
A comparison between the \mg{mofop} and the \mg{mopgfm} has shown that in general the \mg{mopgfm} performs only slightly better than the \mg{mofop}, indicating that both a more flexible joint posterior distribution and a more flexible aggregation function are able to improve results on the PolyMNIST dataset.
This shows that transforming the subset posterior approximation of the \mg{mopgfm} with normalizing flows to obtain a more flexible joint posterior distribution should further improve its results.
Of course, this comes at the cost of increased computational cost and training time.

\paragraph{mogfm\_amortized \& iwmogfm}
The \mg{mogfm\_amortized} and the \mg{iwmofgm} provide a way to obtain both, a more flexible aggregation function and a more flexible joint posterior approximation.
The two methods make use of a modified objective to steer the joint posterior approximation towards a distribution that can be evaluated, but we have found this to be too unstable in practice.
However, our results have shown that both methods are able to learn a good joint posterior distribution, even without the KL-divergence as regularization term in the objective (i.e. with $\beta = 0$).
While this results in a very high generation coherence and quality, this also results in a less structured joint posterior distribution since both methods yield lower linear classification accuracies (\cref{subsec:lr metric}).
In addition, since the joint posterior distribution of both methods cannot be evaluated explicitly, one cannot generate new data by sampling from it.
Overall the \mg{mogfm\_amortized} and the \mg{iwmofgm} provide very promising results and it would be interesting to evaluate in a more extensive study, if the weight of the regularization term in the objective can be adapted such that the learned posterior distribution of both methods matches a prior distribution.


A qualitative evaluation on the challenging MIMIC-CXR dataset shows that the methods are not able to extract meaningful information from the three provided modalitities.
Independent of a more flexible joint posterior distribution and a more flexible aggregation over modalities, the generated samples are extremely blurry and fail to show details in both the modality specific and shared information.
We argue that further adaptations to the training paradigm are needed to capture small details in real world datasets.
Especially for medical images where the shared information between the modalities are pathologies that are sometimes hardly recognizable, even for human experts.
In \citep{dorent_hetero-modal_2019}, the authors show with their modified MVAE model, that aggregating over the modalities on multiple scales provides high quality results for the segmentation of brain tumours.
This could be adapted for our more flexible aggregation function in future work.

Overall, we have shown that the generalized $f$-mean provides a great tool to improve the objective of multi modal VAEs.
In future work, it would be interesting to evaluate theoretical properties of the more flexible aggregation function and how it impacts the tightness of the modified ELBO.



    \printbibliography
%    \includepdf[pages={-}]{thesis/declaration-originality.pdf}


    \appendix

%\section{Supplementary}
%\begin{figure}\centering\resizebox{0.9\textwidth}{!}{\py{boilerplate.make_cond_gen_fig(which='m0__m0',methods=['mopoe', 'mopgfm', 'moe', 'poe', 'mofop', 'iwmogfm_amortized', 'iwmogfm'])}}\end{figure}



\begin{figure}\centering\resizebox{0.9\textwidth}{!}{\py{boilerplate.make_cond_gen_fig(which='m0__m1',methods=['mopoe', 'mopgfm', 'moe', 'poe', 'mofop', 'iwmogfm_amortized', 'iwmogfm'])}}\end{figure}



\begin{figure}\centering\resizebox{0.9\textwidth}{!}{\py{boilerplate.make_cond_gen_fig(which='m0__m2',methods=['mopoe', 'mopgfm', 'moe', 'poe', 'mofop', 'iwmogfm_amortized', 'iwmogfm'])}}\end{figure}



\begin{figure}\centering\resizebox{0.9\textwidth}{!}{\py{boilerplate.make_cond_gen_fig(which='m1__m0',methods=['mopoe', 'mopgfm', 'moe', 'poe', 'mofop', 'iwmogfm_amortized', 'iwmogfm'])}}\end{figure}



\begin{figure}\centering\resizebox{0.9\textwidth}{!}{\py{boilerplate.make_cond_gen_fig(which='m1__m1',methods=['mopoe', 'mopgfm', 'moe', 'poe', 'mofop', 'iwmogfm_amortized', 'iwmogfm'])}}\end{figure}



\begin{figure}\centering\resizebox{0.9\textwidth}{!}{\py{boilerplate.make_cond_gen_fig(which='m1__m2',methods=['mopoe', 'mopgfm', 'moe', 'poe', 'mofop', 'iwmogfm_amortized', 'iwmogfm'])}}\end{figure}



\begin{figure}\centering\resizebox{0.9\textwidth}{!}{\py{boilerplate.make_cond_gen_fig(which='m2__m0',methods=['mopoe', 'mopgfm', 'moe', 'poe', 'mofop', 'iwmogfm_amortized', 'iwmogfm'])}}\end{figure}



\begin{figure}\centering\resizebox{0.9\textwidth}{!}{\py{boilerplate.make_cond_gen_fig(which='m2__m1',methods=['mopoe', 'mopgfm', 'moe', 'poe', 'mofop', 'iwmogfm_amortized', 'iwmogfm'])}}\end{figure}



\begin{figure}\centering\resizebox{0.9\textwidth}{!}{\py{boilerplate.make_cond_gen_fig(which='m2__m2',methods=['mopoe', 'mopgfm', 'moe', 'poe', 'mofop', 'iwmogfm_amortized', 'iwmogfm'])}}\end{figure}



\begin{figure}\centering\resizebox{0.9\textwidth}{!}{\py{boilerplate.make_cond_gen_fig(which='m0_m1__m0',methods=['mopoe', 'mopgfm', 'moe', 'poe', 'mofop', 'iwmogfm_amortized', 'iwmogfm'])}}\end{figure}



\begin{figure}\centering\resizebox{0.9\textwidth}{!}{\py{boilerplate.make_cond_gen_fig(which='m0_m1__m1',methods=['mopoe', 'mopgfm', 'moe', 'poe', 'mofop', 'iwmogfm_amortized', 'iwmogfm'])}}\end{figure}



\begin{figure}\centering\resizebox{0.9\textwidth}{!}{\py{boilerplate.make_cond_gen_fig(which='m0_m1__m2',methods=['mopoe', 'mopgfm', 'moe', 'poe', 'mofop', 'iwmogfm_amortized', 'iwmogfm'])}}\end{figure}



\begin{figure}\centering\resizebox{0.9\textwidth}{!}{\py{boilerplate.make_cond_gen_fig(which='m0_m2__m0',methods=['mopoe', 'mopgfm', 'moe', 'poe', 'mofop', 'iwmogfm_amortized', 'iwmogfm'])}}\end{figure}



\begin{figure}\centering\resizebox{0.9\textwidth}{!}{\py{boilerplate.make_cond_gen_fig(which='m0_m2__m1',methods=['mopoe', 'mopgfm', 'moe', 'poe', 'mofop', 'iwmogfm_amortized', 'iwmogfm'])}}\end{figure}



\begin{figure}\centering\resizebox{0.9\textwidth}{!}{\py{boilerplate.make_cond_gen_fig(which='m0_m2__m2',methods=['mopoe', 'mopgfm', 'moe', 'poe', 'mofop', 'iwmogfm_amortized', 'iwmogfm'])}}\end{figure}



\begin{figure}\centering\resizebox{0.9\textwidth}{!}{\py{boilerplate.make_cond_gen_fig(which='m1_m2__m0',methods=['mopoe', 'mopgfm', 'moe', 'poe', 'mofop', 'iwmogfm_amortized', 'iwmogfm'])}}\end{figure}



\begin{figure}\centering\resizebox{0.9\textwidth}{!}{\py{boilerplate.make_cond_gen_fig(which='m1_m2__m1',methods=['mopoe', 'mopgfm', 'moe', 'poe', 'mofop', 'iwmogfm_amortized', 'iwmogfm'])}}\end{figure}



\begin{figure}\centering\resizebox{0.9\textwidth}{!}{\py{boilerplate.make_cond_gen_fig(which='m1_m2__m2',methods=['mopoe', 'mopgfm', 'moe', 'poe', 'mofop', 'iwmogfm_amortized', 'iwmogfm'])}}\end{figure}



\begin{figure}\centering\resizebox{0.9\textwidth}{!}{\py{boilerplate.make_cond_gen_fig(which='m0_m1_m2__m0',methods=['mopoe', 'mopgfm', 'moe', 'poe', 'mofop', 'iwmogfm_amortized', 'iwmogfm'])}}\end{figure}



\begin{figure}\centering\resizebox{0.9\textwidth}{!}{\py{boilerplate.make_cond_gen_fig(which='m0_m1_m2__m1',methods=['mopoe', 'mopgfm', 'moe', 'poe', 'mofop', 'iwmogfm_amortized', 'iwmogfm'])}}\end{figure}



\begin{figure}\centering\resizebox{0.9\textwidth}{!}{\py{boilerplate.make_cond_gen_fig(which='m0_m1_m2__m2',methods=['mopoe', 'mopgfm', 'moe', 'poe', 'mofop', 'iwmogfm_amortized', 'iwmogfm'])}}\end{figure}





\begin{figure}[h!]
    \centering
    \resizebox{0.7\textwidth}{!}{%
        \py{pytex_printonly(script='thesis/scripts/tikz_graphs/rand_gen_comp_polymnist.py', data = '')}
    }
    \caption{\textbf{Comparison of randomly generated samples between methods.} The samples are generated by sampling from the prior and decoding them with a randomly selected decoder from the modalities $m_0$, $m_1$, $m_2$.
    % todo give the random generation quality scores achieved for each model
    }


\end{figure}

    \backmatter


\end{document}
