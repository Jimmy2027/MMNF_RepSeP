\chapter{Results}
% First compare mopoe, mopgfm, mogfm
% Then compare iwmopoe, iwmogfm, mogfm

% try other datasets also (mimic,..)
% Train 5 times: mopoe, mopgfm, mogfm

% also compare training times


\section{PolyMNIST} \label{subsec: results polymnist}
Evaluating the separability of the latent representation for 3 modalities, we find that the \mg{mofop}, the \mg{mopgfm} and the \mg{mopoe} perform similarly, yielding on average a linear classification accuracy of \py{boilerplate.get_lr_score(method='mofop')}, \py{boilerplate.get_lr_score(method='mopgfm')} and \py{boilerplate.get_lr_score(method='mopoe')} respectively for all subsets (see \cref{fig:ep comp lr}).

\smallskip

\Cref{tab:lr eval} compares the classification accuracies over multiple subsets.
Overall, we see that the classification accuracy improves when more modalities make up the latent representation which shows that all methods are able to aggregate the modalities.
In particular, we find that the \mg{iwmogfm} method has the best performance when all modalities are given.
%Our results show that the $m0$ modality is the hardest modality to learn from todo

% todo comment on nbr mod comp


\begin{sansmath}
    \py{pytex_fig('thesis/scripts/plots/epoch_comparison_lr.py',
        conf='thesis/main.conf',
        label='ep comp lr',
    caption='
    \\textbf{Linear classification accuracy for different epochs, averaged over all subsets, for 3 modalities.}
    Overall, the \mg{mofop}, the \mg{mopoe} and the \mg{mopgfm} method perform similarly.
    ',
    )}
\end{sansmath}

\py{
    pytex_tab(
    script='thesis/scripts/lr_eval_tab.py',
    options_pre='\\centering \\resizebox{0.99\\textwidth}{!}{',
    options_post='}',
    caption='Linear classification accuracy of latent representations for the Test set.\\
    ',
    label='lr eval',
    )
}

\begin{sansmath}
    \py{pytex_fig('thesis/scripts/plots/nbr_mods_comparison_lr.py',
        conf='thesis/main.conf',
        label='nbr mods comp lr',
        caption='
        \\textbf{Generation classification accuracy for different epochs.}
        Overall, the \mg{mofop}, the \mg{mopoe} and the \mg{mopgfm} method perform similarly.
        ',
        )}
\end{sansmath}


\begin{sansmath}
    \py{pytex_fig('thesis/scripts/plots/epoch_comparison_gen.py',
        conf='thesis/main.conf',
        label='ep comp gen',
        caption='
        \\textbf{Generation classification accuracy for different epochs.}
        Overall, the \mg{mofop}, the \mg{mopoe} and the \mg{mopgfm} method perform similarly.
        ',
        )}
\end{sansmath}



\begin{sansmath}
    \py{pytex_fig('thesis/scripts/plots/nbr_mods_comparison_gen.py',
        conf='thesis/main.conf',
        label='nbr mods comp gen',
        caption='
        \\textbf{Generation classification accuracy for different epochs.}
        Overall, the \mg{mofop}, the \mg{mopoe} and the \mg{mopgfm} method perform similarly.
        ',
        )}
\end{sansmath}



%\begin{sansmath}
%\py{pytex_subfigs(
%[
%{'script':'thesis/scripts/plots/epoch_comparison_lr.py', 'label':'vccv','conf':'thesis/4*2.conf', 'options_pre':'{.48\\textwidth}',
%'options_pre_caption':'\\vspace{-1.5em}\\',
%'options_post':'\\vspace{1em}',
%'caption':''
%,},
%{'script':'thesis/scripts/plots/epoch_comparison_gen.py', 'label':'sccv','conf':'thesis/4*2.conf', 'options_pre':'{.48\\textwidth}',
%'options_pre_caption':'\\vspace{-1.5em}\\',
%'options_post':'\\vspace{1em}',
%'caption':''
%,},
%{'script':'thesis/scripts/plots/nbr_mods_comparison_lr.py', 'label':'vcfb','conf':'thesis/4*2.conf', 'options_pre':'{.48\\textwidth}',
%'options_pre_caption':'\\vspace{-1.5em}\\',
%'options_post':'\\vspace{1em}',
%'caption':''
%,},
%{'script':'thesis/scripts/plots/nbr_mods_comparison_gen.py', 'label':'scfb','conf':'thesis/4*2.conf', 'options_pre':'{.48\\textwidth}',
%'options_pre_caption':'\\vspace{-1.5em}\\',
%'options_post':'\\vspace{1em}',
%'caption':''
%,},
%],
%caption='
%',
%label='fig:vc',
%)}
%\end{sansmath}

\py{
pytex_tab(
script='thesis/scripts/gen_eval_tab.py',
options_pre='\\centering \\resizebox{0.7\\textwidth}{!}{',
options_post='}',
caption='Generation coherence for the Test set.
',
)
}



\py{
pytex_tab(
script='thesis/scripts/prd_tab.py',
options_pre='\\centering \\resizebox{0.7\\textwidth}{!}{',
options_post='}',
caption='Area under the Precision and Recall curve of the PRD score \citep{precision_recall_distributions}.\\
',
)
}