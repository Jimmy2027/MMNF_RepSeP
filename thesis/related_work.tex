\chapter{Related Work}
\section{Generative Modeling}
Generative adversarial networks \citep[GANs]{goodfellow_generative_2014} and variational autoencoders (VAEs, \cref{subsec:vae}) are the two most popular methods for generative modeling.
Both attempt to model the distribution over the data, however while for the VAEs, the resulting posterior approximation is defined explicitly, the learned posterior of GANs is not.
GANs are made of two models that are trained simultaneously, a generative model G that captures the data distribution, and a discriminative model D that estimates the probability that a sample came from the training data rather than G.
The joint optimization of both models D and G can be tricky in practice and GANs are known to suffer from mode collapse since the objective does not require the learned representation to contain all modes of the data.
For images of animals for example, the generator G could learn to generate only images of brown, short haired dogs, so well that the discriminator D will not be able to distinguish them from the true data.
Mode collapse does not happen in VAEs since their objective explicitly requires their learned representation to contain all modes of the data.
In this work, we focus on VAEs and give a more in depth introduction in section \cref{subsec:vae}.

\section{Multi Modal Generative Modeling}

There have been a wide range of approaches for multi-modal generative modeling, however most fall short of expressing the complete range of behaviour that we expect in this setting.

Most prior approaches to generative modelling with multi modal data have targeted modality translation, where the model learns to generate one modality conditioned on another one.
In this case input an output modalities of the model are not interchangeable.
Modality translation has been proposed both as VAE based \citep{pandey2017variational, pu2016variational}, as well as GAN based, for domain translation of images \citep{ledig2017photo, liu2019few}.
% todo
However, we expect our method to be able to generate any modality given any subset of modalities which extends translation between modalities.
It would be possible to train $2^M -1$ modality translation network pairs for $M$ modalities, but this is intractable in practice.

Other prior work has targeted to directly model the joint distribution over the data.
The joint multi modal VAE (JMVAE) from \cite{suzuki2016joint} learns a joint posterior distribution using a joint inference network.
To handle missing data at test time, inference networks need to be trained for every subset of modalities.
While feasible for two modalities, this setup quickly becomes intractable with more data types.
Similarly, the multimodal factorisation model (MFM) from \cite{tsai2018learning} explicitly defines a joint inference network on top of uni modal encoders, however additional decoder networks are needed to generate missing modalities.



