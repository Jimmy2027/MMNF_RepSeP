\documentclass{beamer}
\usepackage[orientation=portrait,size=a0]{beamerposter}
\mode<presentation>{\usetheme{ZH}}
% \usepackage{chemformula}
\usepackage[utf8]{inputenc}
\usepackage{multicol}
% \usepackage[german, english]{babel} % required for rendering German special characters
\usepackage{siunitx} %pretty measurement unit rendering
\usepackage[autoprint=false, gobble=auto]{pythontex} % create figures on-line directly from python!
\usepackage{hyperref} %enable hyperlink for urls
\usepackage{ragged2e}
\usepackage[font=scriptsize,justification=justified]{caption}

% \sisetup{per=frac,fraction=sfrac}
\sisetup{per-mode=symbol}

\begin{pythontexcustomcode}[begin]{py}
import os, sys
almost_this_path = os.path.abspath(os.path.dirname(__file__))
this_path_base, _ = os.path.split(almost_this_path)
this_path = os.path.join(this_path_base,"pythontex")

from pylab import gcf
import matplotlib
import matplotlib.pyplot as plt

pytex.add_dependencies(os.path.join(this_path,'matplotlibrc.conf'))

plt.style.use(os.path.join(this_path,'matplotlibrc.conf'))

# Set the prefix used for figure labels
fig_label_prefix = 'fig'
# Track figure numbers to create unique auto-generated names
fig_count = 0

def figure_by_path(figure_path,textheight_frac=1,caption=None,label=None):
    latex_code = "\\begin{figure}\n"
    latex_code += "\\centering\\includegraphics[width={textheight_frac}\\textheight]{{{figure_path}}}\n".format(textheight_frac=textheight_frac,figure_path=figure_path)
    latex_code += "\\vspace{-2.5em}\n"
    latex_code += "\\caption{{{caption}}}\n".format(caption=caption)
    latex_code += "\\label{{fig:{label}}}\n".format(label=label)
    latex_code += "\\end{figure}\n"
    return latex_code

def save_fig(name='', legend=False, fig=None, ext='.pgf', fig_width=1, fig_height=1):
    '''
    Save the current figure (or `fig`) to file using `plt.save_fig()`.
    If called with no arguments, automatically generate a unique filename.
    Return the filename.
    '''
    # Get name (without extension) and extension
    if not name:
        global fig_count
        # Need underscores or other delimiters between `input_*` variables
        # to ensure uniqueness
        name = 'auto_fig_{}-{}'.format(pytex.id, fig_count)
        fig_count += 1
    else:
        if len(name) > 4 and name[:-4] in ['.pgf', '.svg', '.png', '.jpg']:
            name, ext = name.rsplit('.', 1)

    # Get current figure if figure isn't specified
    if not fig:
        fig = gcf()
    fig.set_size_inches(fig_width,fig_height)
    fig.savefig(name + ext)
    fig.clf()
    return name

def latex_environment(name, content='', option=''):
    '''
    Simple helper function to write the `\begin...\end` LaTeX block.
    '''
    return '\\vspace{-0.25cm}\\begin{%s}%s\n%s\n\\end{%s}' % (name, option, content, name)

def latex_figure(name=None, caption='', label='', width=1):
    ''''
    Auto wrap `name` in a LaTeX figure environment.
    Width is a fraction of `\textwidth`.
    '''
    if not name:
        name = save_fig()
    content = '\\centering\n'
    content += '\\makeatletter\\let\\input@path\\Ginput@path\\makeatother\n'
    content += '\\input{%s.pgf}\n' % name
    if not label:
        label = name
    if caption and not caption.rstrip().endswith('.'):
        caption += '.'
    if caption:
        # `\label` needs to be in `\caption` to avoid issues in some cases
        content += "\\caption{%s\\label{%s:%s}}\n" % (caption, fig_label_prefix, label)
    return latex_environment('figure', content, '[htp]')

pytex.bio_fignum = 0
#global pytex # try without this line
def bio_fig(gdd, fname=None, caption=None, label=None):
#        global pytex # and this one, should work
        if fname is None:
            fname = 'pythontex-files-pres/biopython_fig_{0}-{1}.pdf'.format(pytex.id, pytex.bio_fignum)
        gdd.write(fname, "PDF")
        template = '''
    \\begin{{figure}}
    \\centering
    \\includegraphics{{{fname}}}
    \\caption{{ {label} {caption} }}
    \\end{{figure}}
    '''
        if caption is None:
            caption = ''
        if label is None:
            label = ''
        else:
            if not label.startswith('fig:'):
                label = 'fig:' + label
            label = '\\label{{{0}}}'.format(label)
        template = template.format(fname=fname.rsplit('.', 1)[0], label=label, caption=caption)
        print(template)
        pytex.add_created(fname)
        pytex.bio_fignum += 1
        return template
\end{pythontexcustomcode}
\begin{pythontexcustomcode}[end]{py}
\end{pythontexcustomcode}

\begin{pythontexcustomcode}[begin]{py}
pytex.add_dependencies(os.path.join(this_path,'..','poster/main.conf'))
pytex.add_dependencies(os.path.join(this_path,'..','poster/wide.conf'))
pytex.add_dependencies(os.path.join(this_path,'..','poster/3dplot.conf'))
plt.style.use(os.path.join(this_path,'..','poster/main.conf'))
\end{pythontexcustomcode}
%\begin{pythontexcustomcode}[end]{py}
%plt.style.use(os.path.join(this_path,'..','poster/main.conf'))
%\end{pythontexcustomcode}

\usepackage{array,booktabs,tabularx}
\newcolumntype{Z}{>{\centering\arraybackslash}X} % centered tabularx columns

\title{Python\TeX Minimal Demo using a Poster Template}
% \title{Longitudinal opto-pharmaco-fMRI of Selective\\ Serotonin Reuptake Inhibition}
\author{Horea-Ioan Ioanas$^{1}$}
\institute[ETH]{$^{1}$Institute for Biomedical Engineering, ETH and University of Zurich}
\date{\today}

\newlength{\columnheight}
\setlength{\columnheight}{0.881\textheight}

\begin{document}
\begin{frame}
\begin{columns}
	\begin{column}{.43\textwidth}
		\begin{beamercolorbox}[center]{postercolumn}
			\begin{minipage}{.98\textwidth}  % tweaks the width, makes a new \textwidth
				\parbox[t][\columnheight]{\textwidth}{ % must be some better way to set the the height, width and textwidth simultaneously
					\begin{myblock}{Background}
						Background may go here, and is best kept short.
						Here we create some action via a few lines of provide some Lipsum text.

						\vspace{0.8em}

						Lorem ipsum dolor sit amet, consectetur adipiscing elit.
						Proin malesuada magna ut elit fringilla, eu pretium diam suscipit. Ut id sollicitudin velit.
					\end{myblock}\vfill
					\vspace{-0.3em}
					\begin{myblock}{Animals and Timetables}
						romoter.
						An optic fiber was chronically implanted to allow for longitudinal stimulation.
						\vspace{0.4em}
						Longitudinal designs implemented varying SSRI (fluoxetine) administration routes:
						\begin{itemize}
							\item Intravenous (i.v.) at \SI{10}{\mg\per\kg} daily (in cohorts from).
							\item Intraperitoneal (i.p.) at \SI{10}{\mg\per\kg} daily (cohort timetable shown in).
							\item Drinking water (d.w.) at \SI{\approx 20}{\milli\gram\per\kilo\gram} per day (one cohort timetable shown in figure).
						\end{itemize}

						\vspace{0.5em}
						\py{pytex_fig('scripts/radar.py', label='radar', caption='A radar plot.')}
						\vspace{-1em}
					\end{myblock}\vfill
					\vspace{-0.3em}
					\begin{myblock}{Behaviour}
						Behavioural measurements can help stratify subjects and identify correlated fMRI activation patterns.
						Forced swim test results from the optogenetic responder cohort (figure) spread over half the measurement range, and would thus be suitable to regress fMRI variance.
						Results from all binned cohorts (figure) show a trend indicating that fluoxetine treatment may induce additional behaviour variance in healthy C57BL/6 mice undergoing longitudinal fMRI.
						\vspace{0.4em}
						\py{pytex_fig('scripts/3dplot.py', conf='poster/3dplot.conf', label='3dplot', caption='A 3D plot.')}
						\vspace{-1em}
					\end{myblock}\vfill
					\vspace{-0.3em}
					\begin{myblock}{Outlook}
						\begin{itemize}
							\item More constrained region of interest analysis.
							\item Control group analysis to test for underlying (signal decay) trends.
							\item Population stratification according to forced swim and open field test scores.
							\item DR SNR improvement and true functional connectivity tracking.
						\end{itemize}
					\end{myblock}\vfill
		}\end{minipage}\end{beamercolorbox}
	\end{column}
	\begin{column}{.57\textwidth}
		\begin{beamercolorbox}[center]{postercolumn}
			\begin{minipage}{.98\textwidth} % tweaks the width, makes a new \textwidth
				\parbox[t][\columnheight]{\textwidth}{ % must be some better way to set the the height, width and textwidth simultaneously
					\begin{myblock}{MRI Methods}
						\vspace{0.5em}
						\begin{center}
							\begin{minipage}{.37\textwidth}
							Data Acquisition:
								\begin{itemize}
									\item Bruker PharmaScan (\SI{7}{\tesla}, \SI{16}{\centi\metre} bore)
									\item Implant-compatible in-house T/R coil
									\item Gradient-echo EPI:
										\begin{itemize}
											\item TR=\SI{1000}{\ms}, TE=\SI{5}{\ms}, FA=\SI{60}{\degree}
											\item x($\phi$)=\SI{312.5}{\um}, y($\nu$)=\SI{281.25}{\um}, z(slice)=\SI{500}{\um}
										\end{itemize}
									\item Endorem (Laboratoire Guebet SA)
								\end{itemize}
							\end{minipage}
							\begin{minipage}{.26\textwidth}
								Anesthesia Protocol:
								\begin{itemize}
									\item Free breathing, 1:4 O$_2$/air
									\item Bolus:
										\begin{itemize}
											\item \SI{0.05}{\mg\per\kg} Medetomidine
											\item \SI{1.5}{\percent} Isoflurane
										\end{itemize}
									\item Maintenance:
										\begin{itemize}
											\item \SI{0.1}{\mg\per\kg\per\hour} Medetomidine
											\item \SI{0.5}{\percent} Isoflurane
										\end{itemize}
								\end{itemize}
							\end{minipage}
							\begin{minipage}{.32\textwidth}
								Data Analysis:
								\begin{itemize}
									\item SAMRI (\href{https://github.com/IBT-FMI/SAMRI}{github.com/IBT-FMI/SAMRI}), internally using:
										\begin{multicols}{3}
											\begin{itemize}
												\item AFNI
												\item ANTs
												\item Bru2Nii
												\item FSL
												\item Matplotlib
												\item Nibabel
												\item Nilearn
												\item Nipy
												\item Nipype
												\item Pandas
												\item Seaborn
												\item Statsmodels
											\end{itemize}
										\end{multicols}
								\end{itemize}
							\end{minipage}
						\end{center}
					\end{myblock}\vfill
					\begin{myblock}{opto-fMRI}
						\vspace{0.3em}
						\py{pytex_fig('scripts/bsc_percentage.py', conf='poster/wide.conf', label='bsc_percentage', caption='Percentage of Bachelor’s degrees conferred to women in the U.S.A. by major (1970-2011)')}
						\vspace{0.9em}
			
					\end{myblock}\vfill
		}\end{minipage}\end{beamercolorbox}
	\end{column}
\end{columns}
\end{frame}
\end{document}
