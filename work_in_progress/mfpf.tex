% Preamble
\documentclass[11pt]{article}

% Packages
\usepackage{mathtools}
\usepackage{amsthm}
\usepackage{amsmath}
\usepackage[backend=bibtex,style=authoryear,natbib=true]{biblatex}
\usepackage{amsfonts}
\usepackage{textcomp}

\addbibresource{bib.bib}

\newcommand{\logposterior}{ \log Q(z|X)}

\title{Generalized f-mean with planar flows}

\date{}

% Document
\begin{document}
    \maketitle
    \textbf{Idea}: Can transform \textit{joint\_mu} and \textit{joint\_logvar} independently in order to keep a Gaussian distribution.

    Planar flow: $f_{\psi}(\textbf{z}) = \textbf{z} + \textbf{v}\sigma (\textbf{w}^T\textbf{z} + b)$.

    \vspace{\baselineskip}

    For 3 modalities this would give:
    \begin{equation}
        \mathcal{M}_f(\mu _1,\mu_2, \mu _3) = f^{-1}(\frac{f(\mu _1) + f(\mu _2) + f(\mu _3)}{3})
    \end{equation}

    Problem: when using amortized planar flows, each flow's parameters are dependent on the input \textrightarrow
    hard to compute the inverse.

    \vspace{\baselineskip}

    Instead of making parameters dependent on $h^{(i)}$, could use an aggregation of all $h^{(i)}$'s.


    \printbibliography
\end{document}