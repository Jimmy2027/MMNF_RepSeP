\documentclass[english]{scrartcl}
\usepackage[utf8]{inputenc}
\usepackage[british]{babel} % decent hyphenation, avoiding e.g. anal-ysis
\usepackage[iso]{isodate}
\usepackage{sansmath}
\usepackage{booktabs}
\usepackage{graphicx}
\usepackage{graphviz}
\usepackage{makecell}
\usepackage{minted}
\usepackage{siunitx}
\usepackage{subcaption}
\usepackage[section]{placeins}
\usepackage{amsfonts} % needed for \mathbb{} (Only works on capital letters!)
\usepackage{tikz}
\usetikzlibrary{shapes,snakes}
%\usetikzlibrary{shapes.geometric}

\usepackage{hyperref}
\usepackage{amsmath}
% Needs to be loaded after hyperref and amsmath
\usepackage{cleveref}

% new commands
\newcommand{\xdensity}{\textit{p}_x (\textbf{x})}
\newcommand{\udensity}{\textit{p}_u (\textbf{u})}
\newcommand{\transformer}{\tau(z_i;\textbf{h}_i)}
\newcommand{\conditioner}{c_i(\textbf{z}_{<1})}
\newcommand{\where}{\quad \text{where} \quad}
\newcommand{\xset}{\mathbb{X}}
\newcommand{\xseti}{\mathbb{X}^{(i)}}
\newcommand{\approxdistri}{q_{\phi_i}(\textbf{z}|\xseti)}
\newcommand{\approxdistr}{q_{\phi, \psi}(\textbf{z}|\xset)}
\newcommand{\truedistri}{p_{\theta}(\textbf{z}|\xseti)}
\newcommand{\truedistr}{p_{\theta}(\textbf{z}|\textbf{X})}
\newcommand{\elbo}{\mathcal{L}(\theta, \phi; \xseti)}
\newcommand{\xsubset}{\mathbb{X}_k}
\newcommand{\lmopoe}{\mathcal{L}_{MoPoE}(\theta, \phi; \xset)}
\newcommand{\powerset}{\mathcal{P}(\xset)}
\newcommand{\pFmean}{\mathcal{M}_{f_{\psi}}}
\newcommand{\DklTrueApprox}{D_{KL} \left( \approxdistr || \truedistr \right)}
\newcommand{\Mnfi}{\mathcal{M}_{f_{\psi}}\left( q_{\phi _i}(\textbf{z}|\textbf{x}_i) \right)}




% PythonTeX
\usepackage[autoprint=false, gobble=auto, keeptemps=all, pyfuture=all]{pythontex} % create figures on-line directly from python!
\usepackage{pgf}
\begin{pythontexcustomcode}[begin]{py}
import os, sys
almost_this_path = os.path.abspath(os.path.dirname(__file__))
this_path_base, _ = os.path.split(almost_this_path)
this_path = os.path.join(this_path_base,"pythontex")

from pylab import gcf
import matplotlib
import matplotlib.pyplot as plt

pytex.add_dependencies(os.path.join(this_path,'matplotlibrc.conf'))

plt.style.use(os.path.join(this_path,'matplotlibrc.conf'))

# Set the prefix used for figure labels
fig_label_prefix = 'fig'
# Track figure numbers to create unique auto-generated names
fig_count = 0

def figure_by_path(figure_path,textheight_frac=1,caption=None,label=None):
    latex_code = "\\begin{figure}\n"
    latex_code += "\\centering\\includegraphics[width={textheight_frac}\\textheight]{{{figure_path}}}\n".format(textheight_frac=textheight_frac,figure_path=figure_path)
    latex_code += "\\vspace{-2.5em}\n"
    latex_code += "\\caption{{{caption}}}\n".format(caption=caption)
    latex_code += "\\label{{fig:{label}}}\n".format(label=label)
    latex_code += "\\end{figure}\n"
    return latex_code

def save_fig(name='', legend=False, fig=None, ext='.pgf', fig_width=1, fig_height=1):
    '''
    Save the current figure (or `fig`) to file using `plt.save_fig()`.
    If called with no arguments, automatically generate a unique filename.
    Return the filename.
    '''
    # Get name (without extension) and extension
    if not name:
        global fig_count
        # Need underscores or other delimiters between `input_*` variables
        # to ensure uniqueness
        name = 'auto_fig_{}-{}'.format(pytex.id, fig_count)
        fig_count += 1
    else:
        if len(name) > 4 and name[:-4] in ['.pgf', '.svg', '.png', '.jpg']:
            name, ext = name.rsplit('.', 1)

    # Get current figure if figure isn't specified
    if not fig:
        fig = gcf()
    fig.set_size_inches(fig_width,fig_height)
    fig.savefig(name + ext)
    fig.clf()
    return name

def latex_environment(name, content='', option=''):
    '''
    Simple helper function to write the `\begin...\end` LaTeX block.
    '''
    return '\\vspace{-0.25cm}\\begin{%s}%s\n%s\n\\end{%s}' % (name, option, content, name)

def latex_figure(name=None, caption='', label='', width=1):
    ''''
    Auto wrap `name` in a LaTeX figure environment.
    Width is a fraction of `\textwidth`.
    '''
    if not name:
        name = save_fig()
    content = '\\centering\n'
    content += '\\makeatletter\\let\\input@path\\Ginput@path\\makeatother\n'
    content += '\\input{%s.pgf}\n' % name
    if not label:
        label = name
    if caption and not caption.rstrip().endswith('.'):
        caption += '.'
    if caption:
        # `\label` needs to be in `\caption` to avoid issues in some cases
        content += "\\caption{%s\\label{%s:%s}}\n" % (caption, fig_label_prefix, label)
    return latex_environment('figure', content, '[htp]')

pytex.bio_fignum = 0
#global pytex # try without this line
def bio_fig(gdd, fname=None, caption=None, label=None):
#        global pytex # and this one, should work
        if fname is None:
            fname = 'pythontex-files-pres/biopython_fig_{0}-{1}.pdf'.format(pytex.id, pytex.bio_fignum)
        gdd.write(fname, "PDF")
        template = '''
    \\begin{{figure}}
    \\centering
    \\includegraphics{{{fname}}}
    \\caption{{ {label} {caption} }}
    \\end{{figure}}
    '''
        if caption is None:
            caption = ''
        if label is None:
            label = ''
        else:
            if not label.startswith('fig:'):
                label = 'fig:' + label
            label = '\\label{{{0}}}'.format(label)
        template = template.format(fname=fname.rsplit('.', 1)[0], label=label, caption=caption)
        print(template)
        pytex.add_created(fname)
        pytex.bio_fignum += 1
        return template
\end{pythontexcustomcode}
\begin{pythontexcustomcode}[end]{py}
\end{pythontexcustomcode}

\begin{pythontexcustomcode}[begin]{py}
pytex.add_dependencies(
	'lib/utils.py',
	'lib/categorical.py',
	)
\end{pythontexcustomcode}
% Single-session PythonTeX codeblocks
\newcounter{pysessioncounter}
\newcommand{\sessionpy}{%
          \edef\sessionpysession{session\arabic{pysessioncounter}}%
            \stepcounter{pysessioncounter}%
              \expandafter\py\expandafter[\sessionpysession]}

% SIunitx customizations detect-all will use the current font for typesetting
\sisetup{per-mode=symbol, detect-all, range-units = single}
\newcommand\SIci[5]{\SI{#1}{#2}, {#3}CI: \SIrange{#4}{#5}{#2}}

% Fix for matplotlib PGF wonkiness which isn't interpreted correctly by pdflatex
\DeclareUnicodeCharacter{2212}{-}

\usepackage{bm}

\usepackage[backend=bibtex,style=authoryear,natbib=true]{biblatex}
\usepackage{mathtools}
\usepackage{amsthm}
%\theoremstyle{definition}
\newtheorem{definition}{Definition}

\addbibresource{bib.bib}


\title{Multi Modal Generative Learning with Normalizing Flows}
\subtitle{Master Thesis Project Proposal}
\author{Hendrik Klug}

%\date{}
\begin{document}
    \maketitle


    \section{Introduction}
    The availability of multiple data types provides a rich source of information and holds promise for learning representations that generalise well across multiple modalities \parencite{baltrusaitis_multimodal_2019}.
    Multimodal data naturally grants additional self-supervision in the form of shared information connecting the different data types.
    Further, the understanding of different modalities and the interplay between data types are non-trivial research questions and longstanding goals in machine learning research.
    While fully-supervised approaches have been applied successfully \parencite{karpathy_deep_2015,tsai_learning_2018}, the labeling of multiple data types remains time-consuming and expensive.
    Therefore, it requires models that efficiently learn from multiple data types in a self-supervised fashion.
    Self-supervised, generative models are suitable for learning the joint distribution of multiple data types without supervision.
    We focus on VAEs \parencite{kingma_auto-encoding_2014,rezende_stochastic_2014} which are able to jointly infer representations and generate new observations.
    Despite their success on unimodal datasets, there are additional challenges associated with multimodal data \parencite{suzuki_joint_2016, vedantam_generative_2018}.
    In particular, multimodal generative models need to represent both modality-specific and shared factors and generate semantically coherent samples across modalities.
    Semantically coherent samples are connected by the information which is shared between data types \parencite{shi_variational_2019}.
    These requirements are not inherent to the objective: the evidence lower bound (ELBO) of uni-modal VAEs.
    Hence, adaptions to the original formulation are required to cater to and benefit from multiple data types.

    Normalizing flows represent an approach for specifying flexible, arbitrarily complex and scalable approximate posterior distributions \parencite{papamakarios_normalizing_2019,rezende_variational_2016}.
    They consist of transforming a simple initial density into a more complex one by applying a sequence of invertible transformations until a desired level of complexity is attained.
    For this project, I propose to use normalizing flows as a parameterisable function $f_\psi$ which combines the latent variables of multiple data types.
    Normalizing flows are optimal for this task since they can transform simple initial densities into arbitrarily complex ones and vice versa.

    Normalizing flows became very popular in the past years, and many methods already exist for different applications, however they have not yet been used to create a joint latent representation from multiple data types for VAEs.
    The task in this project is to find an adequate implementation of normalizing flows, i.e. the best trade-off between the complexity of the resulting joint latent representation and the computational feasibility, and evaluate it on existing multimodal datasets.
    To this end, I would first compare simpler methods, with a less good approximation of the true joint posterior but less computationally expensive, with more complex methods on toy datasets.
    The ultimate goal being to construct a method that can learn a separable \footnote{separable in the sense that the joint latent representation can be separated into the different classes that span the dataset} joint latent representation from which coherent \footnote{coherent in the sense that all generated samples belong to the same class} samples can be generated on the challenging MIMIC-CXR \parencite{johnson_mimic-cxr-jpg_2019} database.


    \section{Background}

    \subsection{Previous work on Multimodal Generative Learning} \label{subsec:prevwork_mopoe}
    Previous work on self-supervised multimodal generative models has shown that the class of generalised f-means provides a good tool for combining multiple modalities in a dataset.
    In \citep[PoE]{wu_multimodal_2018}, the Product of Experts uses of a geometric mean and in \citep[MoE]{shi_variational_2019}, the Mixture of Experts uses an arithmetic mean.
    More recently, the authors of \citep[MoPoE]{sutter_generalized_2020} improved on previous results by using a combination of a geometric mean and an arithmetic mean.
    Their Mixture of Product of Experts (MoPoE) computes the joint posterior approximation in two steps:\\
    By first computing the individual posterior approximations for each subset of the powerset $\powerset$:
    \begin{equation}
        \tilde{q}_{\phi}(\textbf{z}|\xsubset)=PoE(\{q_{\phi_j}(\textbf{z}|\textbf{x}_j) \forall \textbf{x}_j \in \xsubset\}) \propto \prod _{\textbf{x}_j \in \xsubset}q_{\phi_j}(\textbf{z}|\textbf{x}_j)
    \end{equation}
    which are then fused to get the joint posterior:
    \begin{equation}
        q_{\phi}(\textbf{z}|\mathbb{X}) = \frac{1}{2^3} \sum _{\textbf{x}_k \in \mathbb{X}} \tilde{q}_{\phi} (\textbf{z}|\mathbb{X}_k),
    \end{equation}
    The MoPoE achieves state of the art results on the datasets PolyMNIST (introduced in \citet{sutter_multimodal_2020}) and the celebA faces dataset \citep{liu_deep_2015}.

    \subsection{Previous work on Normalizing Flows}\label{subsec:prev_work_nf}
    In practice, flow-based models are typically constructed by implementing a diffeomorphic transformation T (or $T^{-1}$) with a neural network.
    Because invertible and differentiable transforms are composable, complex transformations can be built by composing multiple instances of simpler ones: $T=T_K \circ \cdots \circ T_1$.
    A flow-based model can be sampled from, and it's density can be evaluated, such that it can be easily integrated into the bottleneck of a VAE.
    The sampling can be achieved with:
    \begin{equation}
        \textbf{x} = T(\textbf{u}) \where \textbf{u} \sim \udensity
    \end{equation}
    And the density can be obtained by a change of variables \parencite{bogachev2007measure}:
    \begin{equation}
        \label{eq:density}
        \xdensity = \udensity |\det J_T(\textbf{u})|^{-1} \quad \text{where} \quad \textbf{u} = T^{-1}(x)
    \end{equation}

    Both the sampling and the evaluation of the density have different computational requirements.
    Sampling from the model requires the ability to sample from $\udensity$ and to compute the forward transformation $T$.
    Evaluating the model's density requires computing the inverse transformation $T^{-1}$ and its Jacobian determinant, as well as evaluating the density $\udensity$.

    In \cite{papamakarios_normalizing_2019}, the authors give an overview of existing methods and applications of normalizing flows and discuss the computational trade-offs for each.
    Papamakarios et al. separate the field of finite compositions of flows into three categories: Autoregressive flows, Linear flows and Residual flows.

    \subsubsection{Autoregressive flows}
    The class of \textit{autoregressive flows} are a direct implementation of flows that are built such that the Jacobian in \cref{eq:density} is triangular.
    In particular, $f_{\theta}$ is specified to have the following form (as described by \cite{papamakarios_normalizing_2019}):
    \begin{equation}
        z_i^{\prime} = \transformer \where \textbf{h}_i = \conditioner
    \end{equation}

    where $\tau$ is termed the \textit{transformer} and $c_i$ the \textit{i}-th \textit{conditioner}.
    The transformer is a strictly monotonic function of $z_i$ (and therefore invertible), is parameterized by $\textbf{h}_i$, and specifies how the flow acts on $z_i$ in order to output $z_i^{\prime}$.
    The conditioner determines the parameters of the transformer, and in turn, can modify the transformer's behaviour.
    The triangularity of the Jacobian is achieved by making the \textit{i}th conditioner only dependent on previous conditioners, with indices less than \textit{i}.
    Autoregressive flows are universal approximators provided the transformer and the conditioner are flexible enough to represent any function arbitrarily well.
    However, one drawback of autoregressive flows is that in the inverse computation $z_i = \tau ^{-1} (z_i ^{\prime}; \textbf{h}_i)$, all $\textbf{z}_{<i}$ need to have been computed before $z_i$, so that $\textbf{z}_{<i}$ is available to the conditioner for computing $\textbf{h}_i$.

    State of the art methods that are using autoregressive flows include \citep[][NICE, Flow++, Glow, PixelCNN]{dinh_nice_2015, ho_flow_2019, kingma_glow_2018, oord_conditional_2016}.

%\newline
%    \textbf{The transformer}
%\newline
%    The \textbf{affine transformer} controls the location with $\alpha_i$ and the scale with $\beta_i$:
%    \begin{equation}
%        \transformer = \alpha_i z_i +\beta _i \where \textbf{h}_i = {\alpha_1, \beta_i}
%    \end{equation}
%    Invertibility can be easily guaranteed by making sure that $\alpha _i \neq 0$ by taking $\alpha _i = \exp \tilde{\alpha _i}$.
%    Also the log absolute Jacobian determinant is easily computable with:
%    \begin{equation}
%        \log |\det J_{f_\phi}(\textbf{z})| = \sum _{i=1} ^{D} \log |\alpha _i| = \sum _{i=1} ^{D} \tilde{\alpha _i}.
%    \end{equation}
%    Because of their simplicity, affine autoregressive flows are very popular in the literature.
%    However, their expressiveness is limited, and it is unknown whether affine autoregressive flows with multiple layers are universal approximators or not.
%
%    \textbf{Non-affine neural transformers} present a way to construct the transformer as a multi-layer perceptron, by taking conic combinations in addition to compositions of transformations:
%    \begin{equation}
%        \label{eq:neural_tf}
%        \tau (z) = \sum _{k=1} ^{K} \omega_k \tau _k(z) \where\omega_k > 0\ \forall\ k
%    \end{equation}
%    When constructing the transformer using a conic combination of monotonically increasing activation functions (such as the sigmoid, tanh, ReLu, \ldots), \cref{eq:neural_tf} becomes a single-layer perceptron.
%    By repeatedly combining and composing monotonic activation functions, a multi-layer perceptron that is monotonic can be constructed, with the restriction that all its weights are strictly positive.
%    Because of the universal-approximation capabilities of multi-layer perceptrons, non-affine neural transformers can represent any function arbitrarily well \parencite{huang_neural_2018}.
%    However, non-affine neural transformers cannot be inverted analytically, and can be inverted only iteratively with a bijection search for example.
%    \newline
%    \textbf{The conditioner}
%    \newline
%    In theory, the conditioner $\conditioner$ can be implemented by any function of $\textbf{z}_{<i}$.
%    However, the main limitation being the scaling, it is not feasible to build each $h_i$ with a neural network.
%    In practice this problem is addressed by sharing parameters across the conditioners $\conditioner$.
%    One way to do that is via a recurrent neural network (RNN), such as a LSTM \parencite{hochreiter_long_1997} for example.
%    The main disadvantage of RNNs however is, that they turn the calculation of the transformations into a sequential operation, such that they can no longer be parallelized.
%
%    Another way to share parameters between the conditioners is via \textbf{masked autoregressive flows}, where the conditioners are computed at once by one neural network.
%    This can be achieved with a network that takes as input $\textbf{z}$ and outputs the entire sequence ($\textbf{h}_1, \ldotp, \textbf{h}_i$).
%    To keep the autoregressive structure, one can mask all connections that make one output dependent on all inputs $\textbf{z}_{\geq i}$.
%    \Cite{papamakarios_normalizing_2019} point out two main advantages of masked autoregressive flows: they are efficient to evaluate and they are universal approximators.
%    One disadvantage is however, that they create an imbalance in the efficiency of the evaluation and the inversion: they are not as efficient to invert as to evaluate.
%    This is because all $z_i$'s need to be computed sequentially.
%
%    A way to overcome this limitation is given by \textbf{coupling layers}, by making parameters ($\textbf{h}_1, \ldotp, \textbf{h}_d$) constants.
%    The rest of the parameters a calculated with a neural network like for the masked autoregressive flows:
%    \begin{equation}
%    (\textbf{h}_{d+1}, \ldotp, \textbf{h}_D)
%        = NN(\textbf{z}_{\leq d})
%    \end{equation}
%    This gives the following special structure of the Jacobian:
%    \begin{equation}
%        J_{f_{\phi}} = \begin{bmatrix}
%                           \textbf{I} & \textbf{0}\\
%                           \textbf{A} & \textbf{D}
%        \end{bmatrix}
%    \end{equation}
%    This structure makes the Jacobian easily computable, it is simply equal to the product of derivatives of the
%    transformers $\textbf{z}^{\prime}_{d+1:D}$.
%    The efficiency of coupling layers comes at the cost of reduced expressive power: a single coupling layer can not represent
%    any autoregressive transformation, regardless of how expressive the neural network is.
%    However the expressivity of the flow can be, in theory, arbitrarily increased by composing multiple coupling layers.
%    Coupling flows are very popular in the literature and used by state of the art models such as \citep[NICE, Glow, WaveGlow, Flow++][]{dinh_nice_2015,kingma_glow_2018, prenger_waveglow_2019, ho_flow_2019}.
%    Showing that it's possible to obtain an universal approximator by composing strictly fewer than $O(D)$ coupling layers is still an open problem.

%    \subsubsection{Linear Flows}
%    A linear flow is an invertible linear transformation of the form:
%    \begin{equation}
%        \textbf{z}^{\prime} = \textbf{W} \textbf{z},
%    \end{equation}
%    where $\textbf{W}$ is a $D \times D$ invertible matrix that parameterises the transformation.
%    Linear flows are essential for autoregressive flows because in order to increase their flexibility, it is needed to permute the order of the input variables $\textbf{z}_{\leq i}$.
%    This permutation can be achieved by setting $\textbf{W}$ to a permutation matrix (i.e. a binary matrix with exactly one entry of 1 in each row and column and 0's everywhere else).
%    In practice, $\textbf{W}$ is a parameterised such that it is guaranteed to be invertible.

    \subsubsection{Residual Flows}
    Residual flows are similar to residual neural networks, between each transformation the identity function is given and any learned transformation is added to the input:
    \begin{equation}
        \label{eq:res_flow_eq}
        \textbf{z}^{\prime} = \textbf{z} + g_{\phi} (\textbf{z})
    \end{equation}
    $g_{\phi} (\textbf{z})$ can be implemented by a neural network with parameters $\phi$ and that outputs a $D$-dimensional translation vector.

    Unlike autoregressive flows, which are based on constraining the Jacobian to be sparse, residual flows have a dense Jacobian in general, which allows all input variables to affect all output variables.
    As a result, residual flows can be very flexible and have demonstrated good results in practice \citep{chen_residual_2020}.
    Sylvester flows \citep{berg_sylvester_2019} in particular provide a method to have $\mathcal{O}(D)$ Jacobians and they can be made invertible by suitably restricting their parameters.
    However, each individual transformation is fairly simple, and it's not clear how the flexibility of the flow can be increased other than by increasing the number of transformations.
    For these reasons, these types of flow have mostly been used to approximate posteriors for variational autoencoders and rarely as generative models in their own right.


    \section{Methods}

    \subsection{Implementation of the Normalizing Flow}
    In order to arrive to a model that can efficiently and accurately approximate the joint posterior distribution of multimodal data, I propose to start from the codebase used in \citet{sutter_generalized_2020}.
    However, where the "Mixture-of-Products-of-Experts-VAE" (MoPoE-VAE) computes the joint posterior approximation in two steps as shown in \cref{subsec:prevwork_mopoe}, the task of this project is to merge the posterior approximations of each of the M modalities $q_{\phi_i}(\textbf{z}|\{\xseti\}_{i=1}^M)$ using a parameterisable function.
    A possible implementation of the fusion of the posterior approximations are generalised f-means \citep{niculescu_convex_2018}.
    In previous work, the arithmetic mean and the geometric mean have been used  \citep[PoE,MoE,MoPoE]{wu_multimodal_2018,shi_variational_2019,sutter_generalized_2020}, which belong to the class of generalised f-means.
    In this sense, using a parameterisable generalised f-means extends and generalises previous work.
    \begin{definition}
        If f is both continuous and injective, the f-mean $\mathcal{M}_f$ is defined as:
        \begin{equation}
             \mathcal{M}_{f}\left( \textbf{x} \right) = f^{-1}\left( \frac{1}{N} \sum ^N _{i=1} f(\textbf{x}_i)) \right)
        \end{equation}
    \end{definition}

    Applying a f-mean with parameters $\psi$ to the unimodal posterior approximations $q_{\phi _i}(\textbf{z}|\textbf{x}_i)$ gives:

    \begin{equation}\label{eq:merge_function}
        q_{\phi, \psi}(\textbf{z} | \xset) = \mathcal{M}_{f_{\psi}}\left( \left\{ q_{\phi _i}(\textbf{z}|\textbf{x}_i)\ \forall\ \textbf{x}_i \in \xseti \right\} \right) = f_{\psi}^{-1}\left( \frac{1}{N} \sum ^N _{i=1} f_{\psi}(q_{\phi _i}(\textbf{z}|\textbf{x}_i)) \right)
    \end{equation}

    Because $f_{\psi}$ needs to be both differentiable, invertible and transform the individual latent representations into a complex enough joint representation, normalizing flows present themselves as an evident choice:
    they are both invertible and differentiable, and can form arbitrarily complex distributions, depending on their implementation.
    During this project, it will be necessary to find a suitable implementation of normalizing flows, that provides a good tradeoff between computational complexity and quality of the joint latent distribution.
    Appealing choices are the coupling layer's implementation of Autoregressive Flows or the Sylvester flow's implementation of Residual Flows (see \Cref{subsec:prev_work_nf}).
    Both trade efficiency against expressiveness of the transformation, which appears like a good tradeoff supposing that the unimodal distributions in \Cref{eq:merge_function} only need to be merged and not changed substantially.

    It follows that the evidence lower bound (ELBO) on the marginal log-likelihood will be of the form:
    \begin{equation}
        \label{eq:elbo}
        \mathcal{L}(\theta, \phi, \psi; \xset) \coloneqq \mathbb{E}_{\approxdistr}[\log p_{\theta}(\xset|\textbf{z})] - D_{KL} \left( \approxdistr ||p_{\theta}(\textbf{z})\right)
    \end{equation}
    In order to successfully apply \Cref{eq:elbo}, several problems need to be addressed.
    For one, it will be necessary to find ways to efficiently calculate the KL-divergence between the joint posterior approximation and the prior.
    In \citet{papamakarios_normalizing_2019}, the authors show that the KL-divergence in \Cref{eq:elbo} can be rewritten as:
    \begin{equation}
        \label{eq:Dkl}
        \begin{split}
            \mathcal{L}_{D_{KL}}(\theta, \phi, \psi; \xset) & =D_{KL} \left( \approxdistr || \truedistr \right)\\
            & = \mathbb{E}_{\approxdistr}[\log \approxdistr - \log \truedistr]\\
            & = \mathbb{E}_{p_{u,\phi}(\textbf{u})}[\log p_{u,\phi} \left(\textbf{u} \right) - \log |\det J_{T}(\textbf{u}; \boldsymbol{\psi})|- \log p_{\theta}(T(\textbf{u}; \boldsymbol{\psi}))].
        \end{split}
    \end{equation}
    In the last step, $T(\textbf{u},\boldsymbol{\psi})$ is the transformation from $p_{u,\phi}(\textbf{u})$ to $\approxdistr$, where $p_{u,\phi}(\textbf{u})$ is the distribution given by the encoder of the VAE.

    The parameters $\boldsymbol{\phi}, \boldsymbol{\psi}$ can be optimised iteratively with stochastic gradient descent.
    An unbiased estimate of the gradient of \Cref{eq:Dkl} can be obtained with respect to the parameters with:
    \begin{equation}\label{eq:grad_Dkl}
        \nabla _{\psi} \mathcal{L}_{D_{KL}}(\phi, \psi) \approx - \frac{1}{N} \sum ^N _{n=1}   \nabla _{\psi}\log |\det J_{T}(\textbf{u}_n; \boldsymbol{\psi})| + \nabla _{\psi} \log p_{\theta}(T(\textbf{u}; \boldsymbol{\psi})).
    \end{equation}
    
    In order to apply normalising flows to multimodal data, it will be necessary to derive \cref{eq:Dkl} and \cref{eq:grad_Dkl} with $\textbf{u}$ being multimodal: $p_{u,\phi}(\textbf{u}) = \left\{ q_{\phi _i}(\textbf{z}|\textbf{x}_i)\ \forall\ \textbf{x}_i \in \xseti \right\}_{i=1}^M$.

    \subsection{Evaluation}
    To evaluate the method, I propose to apply it to the toy dataset MNIST-SVHN-TEXT introduced in \citet{sutter_generalized_2020}.
    This allows for an easy comparison between the new method and previous work on multi modal data generation \citep[MoPoE, MoE, PoE][]{sutter_generalized_2020, shi_variational_2019, wu_multimodal_2018}.
    In a second step, the new method can then be applied to the challenging medical dataset MIMIC-CXR \citep{johnson_mimic-cxr-jpg_2019} in order to evaluate it on real world applications.

    For comparison to previous work, I propose to evaluate the separability of the latent representation, the generation quality, and the generation coherence.


    \section{Objective}
    This Master Project will consist of deriving a theoretical foundation as well as establishing a codebase in order to successfully create and evaluate a multimodal VAE model that makes use of normalizing flows.
    The theoretical foundation will need to be sound, and the codebase will need to be well documented and structured such that the results can be easily reproduced, verified and improved in future work.

    \newpage
    \printbibliography

%    \newpage
%    \begin{equation}
%        \textbf{z}_{Sylvester}^\prime =f(\textbf{z})=  \textbf{z} + \textbf{A}  h(\textbf{B}\textbf{z} + \textbf{b})
%    \end{equation}
%    \begin{equation}
%        \det \left( \frac{d \textbf{z}_{Sylvester}^\prime}{d \textbf{z}}  \right)= \det(\textbf{I}_M + \text{diag} (h^\prime (\textbf{B} \textbf{z} + \textbf{b}))\textbf{B}\textbf{A})
%    \end{equation}
%    What I want:
%    \begin{equation}
%        \textbf{z}^\prime =f(\textbf{z}^{(0)}, \ldots, \textbf{z}^{(m)})= \sum _{i=0} ^m \textbf{z}^{(i)} + \textbf{A}^{(i)}  h^{(i)}(\textbf{B}^{(i)}\textbf{z}^{(i)} + \textbf{b}^{(i)})
%    \end{equation}
%    with $\textbf{z}^{(i)}, \textbf{z}^\prime \in \mathbb{R}^{D}, \textbf{A}^{(i)} \in \mathbb{R}^{D\times M},   \textbf{B}^{(i)} \in \mathbb{R}^{M\times D}, \textbf{b}^{(i)} \in \mathbb{R}^{M}$ and $M\leq D$.
%
%
%
%    $\rightarrow J(\textbf{z}^\prime) \in \mathbb{R}^{M\times M} $ not diagonal.
%    \newline
%    What if $\textbf{z}^{(i)}, \textbf{z}^\prime \in \mathbb{R}^{m*D}$?
\end{document}
