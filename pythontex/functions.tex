\begin{pythontexcustomcode}[begin]{py}
import os, sys
almost_this_path = os.path.abspath(os.path.dirname(__file__))
this_path_base, _ = os.path.split(almost_this_path)
this_path = os.path.join(this_path_base,"pythontex")

from pylab import *
import matplotlib
import matplotlib.pyplot as plt
from matplotlib import rcParams

pytex.add_dependencies(os.path.join(this_path,'matplotlibrc.conf'))
sys.path.append(this_path)

plt.style.use(u'seaborn-darkgrid')
plt.style.use(u'ggplot')
plt.style.use(os.path.join(this_path,'matplotlibrc.conf'))

from scipy.stats import ttest_rel, ttest_ind, ttest_1samp

# Set the prefix used for figure labels
fig_label_prefix = 'fig'
# Track figure numbers to create unique auto-generated names
fig_count = 0

def figure_by_path(figure_path,textheight_frac=1,caption=None,label=None):
    latex_code = "\\begin{figure}\n"
    latex_code += "\\centering\\includegraphics[width={textheight_frac}\\textheight]{{{figure_path}}}\n".format(textheight_frac=textheight_frac,figure_path=figure_path)
    latex_code += "\\vspace{-2.5em}\n"
    latex_code += "\\caption{{{caption}}}\n".format(caption=caption)
    latex_code += "\\label{{fig:{label}}}\n".format(label=label)
    latex_code += "\\end{figure}\n"
    return latex_code

def save_fig(name='', legend=False, fig=None, ext='.pgf', fig_width=1, fig_height=1):
    '''
    Save the current figure (or `fig`) to file using `plt.save_fig()`.
    If called with no arguments, automatically generate a unique filename.
    Return the filename.
    '''
    # Get name (without extension) and extension
    if not name:
        global fig_count
        # Need underscores or other delimiters between `input_*` variables
        # to ensure uniqueness
        name = 'auto_fig_{}-{}'.format(pytex.id, fig_count)
        fig_count += 1
    else:
        if len(name) > 4 and name[:-4] in ['.pgf', '.svg', '.png', '.jpg']:
            name, ext = name.rsplit('.', 1)

    # Get current figure if figure isn't specified
    if not fig:
        fig = gcf()
    fig.set_size_inches(fig_width,fig_height)
    fig.savefig(name + ext)
    fig.clf()
    return name

def latex_environment(name, content='', option=''):
    '''
    Simple helper function to write the `\begin...\end` LaTeX block.
    '''
    return '\\vspace{-0.25cm}\\begin{%s}%s\n%s\n\\end{%s}' % (name, option, content, name)

def latex_figure(name=None, caption='', label='', width=1):
    ''''
    Auto wrap `name` in a LaTeX figure environment.
    Width is a fraction of `\textwidth`.
    '''
    if not name:
        name = save_fig()
    content = '\\centering\n'
    content += '\\makeatletter\\let\\input@path\\Ginput@path\\makeatother\n'
    content += '\\input{%s.pgf}\n' % name
    if not label:
        label = name
    if caption and not caption.rstrip().endswith('.'):
        caption += '.'
    if caption:
        # `\label` needs to be in `\caption` to avoid issues in some cases
        content += "\\caption{%s\\label{%s:%s}}\n" % (caption, fig_label_prefix, label)
    return latex_environment('figure', content, '[htp]')

pytex.bio_fignum = 0
#global pytex # try without this line
def bio_fig(gdd, fname=None, caption=None, label=None):
#        global pytex # and this one, should work
        if fname is None:
            fname = 'pythontex-files-pres/biopython_fig_{0}-{1}.pdf'.format(pytex.id, pytex.bio_fignum)
        gdd.write(fname, "PDF")
        template = '''
    \\begin{{figure}}
    \\centering
    \\includegraphics{{{fname}}}
    \\caption{{ {label} {caption} }}
    \\end{{figure}}
    '''
        if caption is None:
            caption = ''
        if label is None:
            label = ''
        else:
            if not label.startswith('fig:'):
                label = 'fig:' + label
            label = '\\label{{{0}}}'.format(label)
        template = template.format(fname=fname.rsplit('.', 1)[0], label=label, caption=caption)
        print(template)
        pytex.add_created(fname)
        pytex.bio_fignum += 1
        return template
\end{pythontexcustomcode}
\begin{pythontexcustomcode}[end]{py}
\end{pythontexcustomcode}
