\begin{pythontexcustomcode}[begin]{py}
import os, sys

import matplotlib.pyplot as plt
from pylab import gcf

# Set the prefix used for figure labels
fig_label_prefix = 'fig'
# Track figure numbers to create unique auto-generated names
fig_count = 0

def pytex_tab(script, caption='', label='', options='', data=''):
    import sys
    try:
        from StringIO import StringIO
    except ImportError:
        from io import StringIO
    import contextlib

    if data:
        pytex.add_dependencies(data)

    @contextlib.contextmanager
    def stdoutIO(stdout=None):
        old = sys.stdout
        if stdout is None:
            stdout = StringIO()
        sys.stdout = stdout
        yield stdout
        sys.stdout = old

    with stdoutIO() as s:
        exec(open(script).read(), locals())

    tab = latex_table(s.getvalue(), caption=caption, label=label, options=options)
    return tab

def pytex_fig(script, conf=[], caption='', label='', multicol=False):
    '''
    Executes a Python script while applying the custom style.

    Notes
    -----
    We go about this in a somewhat roundabout fashion - always applying `DOC_STYLE`, and then additionally applying a context style which may be only `[DOC_STYLE]` in case no other config file is specified.
    This is because scripts need to be executed in a while statement lest rcParams become persistent between figures.
    Conversely, the text engine part of the configuration needs to be applied outside of the context statement, because it will not work inside it.
    '''
    pytex.add_dependencies(script)
    try:
        document_style = DOC_STYLE
    except NameError:
        document_style = 'pythontex/minimal.conf'
    #for some reason this needs to be applied here, in addition to the context application
    #below (where it can be omitted, but is needed to make sure the `figre_styles` variable is populated).
    plt.style.use(document_style)
    try:
        if isinstance(conf, basestring):
            conf = [conf]
    except NameError:
        if isinstance(conf, str):
            conf = [conf]
    figure_styles = [document_style]+conf
    pytex.add_dependencies(*figure_styles)
    with plt.style.context(figure_styles):
        exec(open(script).read())
    if multicol:
        fig = latex_figure(save_fig(), "figure*", caption=caption, label=label)
    else:
        fig = latex_figure(save_fig(), "figure", caption=caption, label=label)
    return fig

def figure_by_path(figure_path,textheight_frac=1,caption=None,label=None):
    latex_code = "\\begin{figure}\n"
    latex_code += "\\centering\\includegraphics[width={textheight_frac}\\textheight]{{{figure_path}}}\n".format(textheight_frac=textheight_frac,figure_path=figure_path)
    latex_code += "\\vspace{-.5em}\n"
    latex_code += "\\caption{{{caption}}}\n".format(caption=caption)
    latex_code += "\\label{{fig:{label}}}\n".format(label=label)
    latex_code += "\\end{figure}\n"
    return latex_code

def save_fig(name='', legend=False, fig=None, ext='.pgf'):
    '''
    Save the current figure (or `fig`) to file using `plt.save_fig()`.
    If called with no arguments, automatically generate a unique filename.
    Return the filename.
    '''
    # Get name (without extension) and extension
    if not name:
        global fig_count
        # Need underscores or other delimiters between `input_*` variables
        # to ensure uniqueness
        name = 'auto_fig_{}-{}'.format(pytex.id, fig_count)
        fig_count += 1
    else:
        if len(name) > 4 and name[:-4] in ['.pgf', '.svg', '.png', '.jpg']:
            name, ext = name.rsplit('.', 1)

    # Get current figure if figure isn't specified
    if not fig:
        fig = gcf()
    fig.savefig(name + ext)
    fig.clear()
    plt.cla()
    plt.clf()
    plt.close()
    plt.close('all')
    return name

def latex_environment(name, content='', option=''):
    '''
    Simple helper function to write the `\begin...\end` LaTeX block.
    '''
    return '\\begin{%s}%s\n%s\n\\end{%s}' % (name, option, content, name)

def latex_table(table, caption='', label='', options=''):
    content = table
    content += "\\caption{%s\\label{%s:%s}}\n" % (caption, "tab", label)
    return latex_environment("table", content=content, option=options)

def latex_figure(name, figure, caption='', label='', width=1):
    ''''
    Auto wrap `name` in a LaTeX figure environment.
    Width is a fraction of `\textwidth`.
    '''
    if not name:
        name = save_fig()
    content = '\\centering\n'
    content += '\\makeatletter\\let\\input@path\\Ginput@path\\makeatother\n'
    content += '\\input{%s.pgf}\n' % name
    if not label:
        label = name
    if caption and not caption.rstrip().endswith('.'):
        caption += '.'
    if caption:
        # `\label` needs to be in `\caption` to avoid issues in some cases
        content += "\\caption{%s\\label{%s:%s}}\n" % (caption, fig_label_prefix, label)
    return latex_environment(figure, content, '[htp]')

pytex.bio_fignum = 0
#global pytex # try without this line
def bio_fig(gdd, fname=None, caption=None, label=None):
#        global pytex # and this one, should work
        if fname is None:
            fname = 'pythontex-files-pres/biopython_fig_{0}-{1}.pdf'.format(pytex.id, pytex.bio_fignum)
        gdd.write(fname, "PDF")
        template = '''
    \\begin{{figure}}
    \\centering
    \\includegraphics{{{fname}}}
    \\caption{{ {label} {caption} }}
    \\end{{figure}}
    '''
        if caption is None:
            caption = ''
        if label is None:
            label = ''
        else:
            if not label.startswith('fig:'):
                label = 'fig:' + label
            label = '\\label{{{0}}}'.format(label)
        template = template.format(fname=fname.rsplit('.', 1)[0], label=label, caption=caption)
        print(template)
        pytex.add_created(fname)
        pytex.bio_fignum += 1
        return template
\end{pythontexcustomcode}
\begin{pythontexcustomcode}[end]{py}
\end{pythontexcustomcode}
